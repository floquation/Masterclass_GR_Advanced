\section{Datatype: Boolean}


\begin{frame}
\frametitle{Wat is een Boolean?}

\begin{itemize}
  \item<1-> Een Boolean datatype beschrijft een waarheid,
  			of of er aan een conditie voldaan wordt: ``Juist'' of ``Onjuist''
  \begin{itemize}
    \item<2-> In het Engels, ``True'' of ``False''
  \end{itemize}
  \item<3-> Een Boolean kan dus slechts twee waarden aannemen: \\ Hij is ``Binair'',
  			en is dus ook te beschouwen als een Binair getal: 1 (juist) of 0 (onjuist).
  \item<4-> De Boolean wordt ook gebruikt in de Logica om op een formele manier
  			te redeneren over de juistheid van argumenten.		
\end{itemize}

\end{frame}





\begin{frame}
\frametitle{Wat is het nut van een Boolean?}

Booleans kunnen gebruikt worden om je programma te conditioneren:

Onder bepaalde voorwaarden (condities) heeft je programma een andere output.

\vspace{0.5cm}
\pause %1
Neem bijvoorbeeld de ABC formule:

\pause %2
Afhankelijk van
\only<5->{\underline{het teken van de discriminant}}\only<1-4>{het teken van de discriminant}
zijn er nul, een of twee oplossingen.

$D=b^2-4ac$

\vspace{0.3cm}
\pause %3
Wat is hier de conditie?
\pause %4

\end{frame}





\begin{frame}
\frametitle{Voorbeelden van statements}

Condities zijn statements zoals:

\inlineticalc{C=2} \inlineticalc{\pi\!3} \inlineticalc{A>6} \inlineticalc{A<B} \inlineticalc{X\ge42} \inlineticalc{0\le A}

\uncover<2->{Vraag: Wanneer zijn deze 6 condities ``true''?}

\visible<3->{
\begin{itemize}
  \item \tifonttxt{=} Gelijk aan.
  \item \tifonttxt{\!} Ongelijk aan.
  \item \tifonttxt{>} Groter dan.
  \item \tifonttxt{<} Kleiner dan.
  \item \tifonttxt{\ge} Groter dan of gelijk aan.
  \item \tifonttxt{\le} Kleiner dan of gelijk aan.
\end{itemize}
}

\end{frame}












% EOF