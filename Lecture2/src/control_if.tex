\section{If-statements}


\begin{frame}
\frametitle{Hoe kunnen we condities gebruiken in een programma?}

Allereerst\ldots Hoe gebruiken we condities in ons taalgebruik?

\vspace{15pt}
\visible<2->{``Als het regent, dan blijf ik binnen.''}

\visible<3->{``Als ik later groot ben, dan wordt ik een brandweerman.''}

\visible<4->{``Als ik die toets haal, dan ga ik over, of anders blijf ik zitten.''}

\vspace{15pt}
\visible<5->{In het algemeen:\\
``\textbf{Als} <<CONDITIE>>, \textbf{dan} <<ACTIE>>, \textbf{of anders} <<ACTIE>>''}

\end{frame}




\begin{frame}
\frametitle{Hoe kunnen we condities gebruiken in een programma?}

\visible<1->{``\textbf{Als} <<CONDITIE>>, \textbf{dan} <<ACTIE>>, \textbf{of anders} <<ACTIE>>. \textbf{Klaar}.''}

\vspace{3pt}
\visible<2->{``\textbf{If} <<CONDITION>>, \textbf{then} <<ACTION>>, \textbf{else} <<ACTION>>. \textbf{End}.''}

\vspace{15pt}
\visible<3->{
\begin{minipage}{0.43\textwidth}%
	\begin{ticalc}[4.3cm]
		PROGRAM\:FIRST IF\\%
		\:If\,SOME\,CONDITION\\%
		\:Then\\%
		\:DO\,SOMETHING\\%
		\:Else\\%
		\:DO\,OTHER\,THING\\%
		\:End
	\end{ticalc}
\end{minipage}%
~
\visible<4->{
\begin{minipage}{0.53\textwidth}%
	\begin{ticalc}[5.3cm]
		PROGRAM\:FIRST IF\\%
		\:If\,R=0\\%
		\:Then\\%
		\:Disp\,\qt IT\,IS\,DRY\,OUTSIDE\qt\\%
		\:Else\\%
		\:Disp\,\qt IT\,IS\,RAINY\,TODAY\qt\\%
		\:End
	\end{ticalc}
\end{minipage}%	
}
}

\vspace{5pt}
\visible<5->{Er ontstaan nu twee paden in het programma:\\}
\visible<6->{Slechts de statement(s) tussen \tifonttxt{Then} en \tifonttxt{Else} wordt uitgevoerd,
of de statement(s) tussen \tifonttxt{Else} en \tifonttxt{End}.}
\visible<7->{Daarna gaat het programma verder met statements die onder \tifonttxt{End} staan.}

\end{frame}





\begin{frame}
\frametitle{Waar staat het op de GR?}

\visible<1-2,6->{\ticalcfig{\ticalcfigCircle{\ticalcfigCircleColThree}{0.615}}} % prgm
\visible<3>{\ticalcfig{}} % nothing
\visible<4-5>{\ticalcfig{\ticalcfigCircleSecond\ticalcfigCircle{\ticalcfigCircleColOne}{0.64}}} % test menu

\begin{itemize}
  \item<1-> Maak een nieuw programma: \inlineticalc{PROGRAM\:FIRSTIF}.
  \item<2-> Druk op \tiPRGM\,en kies \tifonttxt{If}.
  \item<3-> Type een letter.
  \item<4-> Druk op \tiTEST=\tiSecond\tiMATH\,en kies \tifonttxt{=}.
  \item<5-> Type \tifonttxt{0} gevolgd door \tiENTER.
  \item<6-> Druk op \tiPRGM\,en kies \tifonttxt{Then}, daarna \tiENTER.
  \item<7-> Druk op \tiPRGM\,en kies \tifonttxt{Else}, daarna \tiENTER.
  \item<8-> Druk op \tiPRGM\,en kies \tifonttxt{End}, daarna \tiENTER.
  \item<9-> Dit is de structuur van het \tifonttxt{if}-statement.
\end{itemize}


\begin{tikzpicture}[overlay,remember picture]
	\node[] (BL) at (current page.south east){ };
	\node [anchor=south east,xshift=0.195cm,yshift=0.6cm] at (BL)
	{%
	\only<1,3,5,9>{%
		\begin{ticalc}
			PROGRAM\:FIRSTIF\\%
			\:\only<1>{\tiCursor}\visible<3->{If\,A\only<3>{\tiCursor}\visible<5->{=0\\%
			\:\only<5>{\tiCursor}}}\visible<9->{Then\\%
			\:Else\\%
			\:End}
		\end{ticalc}
	}%
	\only<2>{%
		\begin{ticalc}
			\select{CTL}\,I/O\,EXEC \\
			\selectitem{1\+\:}If \\
			2\:Then \\
			3\:Else \\
			4\:For( \\
			5\:While \\
			6\:Repeat( \\
			7\arrowdown End
		\end{ticalc}
	}%
	\only<4>{%
		\begin{ticalc}
			\select{TEST}\,LOGIC \\
			\selectitem{1\+\:}= \\
			2\:\! \\
			3\:> \\
			4\:\ge \\
			5\:< \\
			6\:\le
		\end{ticalc}
	}%
	\only<6>{%
		\begin{ticalc}
			\select{CTL}\,I/O\,EXEC \\
			1\+\:If \\
			\selectitem{2\:}Then \\
			3\:Else \\
			4\:For( \\
			5\:While \\
			6\:Repeat( \\
			7\arrowdown End
		\end{ticalc}
	}%
	\only<7>{%
		\begin{ticalc}
			\select{CTL}\,I/O\,EXEC \\
			1\+\:If \\
			2\:Then \\
			\selectitem{3\:}Else \\
			4\:For( \\
			5\:While \\
			6\:Repeat( \\
			7\arrowdown End
		\end{ticalc}
	}%
	\only<8>{%
		\begin{ticalc}
			\select{CTL}\,I/O\,EXEC \\
			1\+\:If \\
			2\:Then \\
			3\:Else \\
			4\:For( \\
			5\:While \\
			6\:Repeat( \\
			\selectitem{7\arrowdown} End
		\end{ticalc}
	}%
	};
\end{tikzpicture}

\end{frame}





\begin{frame}
\frametitle{Intermezzo: Invoegen/Insert}

\begin{ticalc}
	PROGRAM\:FIRSTIF\\%
	\:If\,A=0\\%
	\:Then\\%
	\:Else\\%
	\:End
\end{ticalc}

\end{frame}






