\section{Exercises}

\begin{frame}
\frametitle{Exercises}

\begin{itemize}
  \item Redeneer m.b.v. formele logica:
	\begin{enumerate}
      \item Het regent. Ik heb een paraplu. Regent het \tifonttxt{and} heb ik een paraplu?
      \item Het is bewolkt. Het is droog. Regent het \tifonttxt{or} is het bewolkt?
      \item Het is bewolkt. Het regent. Regent het \tifonttxt{xor} is het bewolkt?
      \item Het regent. Regent het niet niet niet?
	\end{enumerate}
  \item Maak de pseudocodes en implementeer alle algoritmes van slide \ref{frame:pseudo_exercises}
	(dat zijn de opdrachten aan het einde van ``pseudocode \& algoritmes''), voor in hoeverre je kunt.
  \item De pokeropdracht is optioneel. Ik raad aan om w\'el de pseudocode te maken, maar het niet te implementeren.
	Dat is gigantisch veel werk op je GR.
% Tip voor de pokeropdracht: Gebruik twee variabelen per kaart (dus 10 in totaal).
% \'E\'en voor het nummer (1 tot 13), en \'e\'en voor de kleur (harten, ruiten, schoppen, klaveren).
  \item Breid je \tifonttxt{ABCPD} programma uit voor de drie situaties van \tifonttxt{D}.
\end{itemize}

\end{frame}

%
% \begin{enumerate} 
%   \item Bepaal of iemand genoeg geld, $g$, op zijn rekening heeft staan, indien hij een product voor $x$ euro wil kopen
%   	en $r$ euro in het rood mag staan.
%   \item Gegeven de huidige datum, $(y_t, m_t, d_t)$, en iemand's geboortedatum, $(y_g, m_g, d_g)$, bepaal zijn/haar leeftijd.
%   \item Uit een set van 5 kaarten, bepaal wat de pokerset is.
%   \begin{itemize}
%     \item Bijvoorbeeld: ``pair of $5$'s, $K$ high'' or ``straight, $9$ high''
%   \end{itemize}
%   \item Er zijn vier spelers die elk punten hebben. Bepaal de winnaar, i.e. degene met de meeste punten.
%   \item Bereken de `rest' van $y/x$.
%   \begin{itemize}
%     \item Bijvoorbeeld: $5/2 = 2$ `rest' $1$
%     \item Tip: Kijk naar \tiMATH\inlineticalc{NUM}\inlineticalc{int(} of \tiMATH\inlineticalc{NUM}\inlineticalc{fPart(}
%   \end{itemize}
% \end{enumerate}
%

\begin{frame}
\frametitle{Antwoorden}

\begin{enumerate}
  \item Het regent. Ik heb een paraplu. Regent het \tifonttxt{and} heb ik een paraplu?
	\begin{itemize}
	  \item \inlineticalc{R=1}\inlineticalc{P=1} Gevraagd: \inlineticalc{R\,and\,P}
	  \item Dit geeft \tifonttxt{1} (true)!
	\end{itemize}
  \item Het is bewolkt. Het is droog. Regent het \tifonttxt{or} is het bewolkt?
	\begin{itemize}
	  \item \inlineticalc{B=1}\inlineticalc{R=0} Gevraagd: \inlineticalc{B\,or\,R}
	  \item Dit geeft \tifonttxt{1} (true)!
	\end{itemize}
  \item Het is bewolkt. Het regent. Regent het \tifonttxt{xor} is het bewolkt?
	\begin{itemize}
	  \item \inlineticalc{B=1}\inlineticalc{R=1} Gevraagd: \inlineticalc{B\,xor\,R}
	  \item Dit geeft \tifonttxt{0} (false)!
	\end{itemize}
  \item Het regent. Regent het niet niet niet? 
	\begin{itemize}
	  \item \inlineticalc{R=1} Gevraagd: \inlineticalc{not(not(not(R)))}
	  \item Redeneer van binnen naar buiten: \inlineticalc{not(R)=0} \inlineticalc{not(not(R))=1}
	  	en dus is het antwoord \tifonttxt{0} (false)!
	\end{itemize}
\end{enumerate}

\end{frame}

\begin{frame}
\frametitle{Antwoorden}

\begin{minipage}{0.4\textwidth}
\begin{ticalc}
	PROGRAM\:NUFMONEY\\%
	\:Input\,\qt HOEVEEL\,GELD?\,\qt\comma G\\%
	\:Input\,\qt HOEVEEL\,KOST\,HET?\,\qt\comma X\\%
	\:Input\,\qt HOEVEEL\,KREDIET?\,\qt\comma R\\%
	\:\\%
	\:If\,G-X\ge0\:Then\\%
	\:Disp\,\qt JE\,KUNT\,HET\,BETALEN\qt\\%
	\:Else\\%
	\:If\,X-G>R\:Then\\%
	\:Disp\,\qt TE\,DUUR\qt\\%
	\:Else\\%
	\:Disp\,\qt OK,\,KOST\,KREDIET\qt\\%
	\:End
\end{ticalc}
\end{minipage}
\begin{minipage}{0.4\textwidth}
\begin{ticalc}[5cm]
	PROGRAM\:BIRTHDAY\\%
	\:Input\,\qt YOU\,Y\qt\comma Y\\%
	\:Input\,\qt YOU\,M\qt\comma M\\%
	\:Input\,\qt YOU\,D\qt\comma D\\%
	\:Input\,\qt CUR\,Y\qt\comma Z\\%
	\:Input\,\qt CUR\,M\qt\comma N\\%
	\:Input\,\qt CUR\,D\qt\comma E\\%
	\:\\%
	\:Z-Y\>A\\%
	\:If\,M>N\:Then\\%
	\:A+1\>A\\%
	\:Else\\%
	\:If\,M=N\,and\,D>E\:Then\\%
	\:A-1\>A\\%
	\:End\\%
	\:End\\%
	\:Disp\,A
\end{ticalc}
\end{minipage}

\end{frame}



\begin{frame}
\frametitle{Antwoorden}

\begin{minipage}{0.48\textwidth}
Dit negeert gelijkspel\ldots En is slecht leesbaar.
Later meer over dit soort gevallen (m.b.v. Lists).

\vspace{0.2cm}
\begin{ticalc}
	PROGRAM\:WIEWINT\\%
	\:Prompt\,A\\%
	\:Prompt\,B\\%
	\:Prompt\,C\\%
	\:Prompt\,D\\%
	\:\\%
	\:If\,A>B\:Then\\%
		\:If\,A>C\:Then\\% A or D win
			\:If\,A>D\:Then\\%
				\:Disp\,\qt A\,WINS\qt\\%
			\:Else\\%
				\:Disp\,\qt D\,WINS\qt\\%
			\:End\\%
		\:Else\\% C or D win
			\:If\,C>D\:Then\\%
				\:Disp\,\qt C\,WINS\qt\\%
\end{ticalc}
\end{minipage}
~
\begin{minipage}{0.48\textwidth}
\begin{ticalc}	
			\:Else\\%
				\:Disp\,\qt D\,WINS\qt\\%
			\:End\\%		
		\:End\\%
	\:Else\\% A does not win
		\:If\,B>C\:Then\\% B or D win
			\:If\,B>D\:Then\\%
				\:Disp\,\qt B\,WINS\qt\\%
			\:Else\\%
				\:Disp\,\qt D\,WINS\qt\\%
			\:End\\%
		\:Else\\% C or D win
			\:If\,C>D\:Then\\%
				\:Disp\,\qt C\,WINS\qt\\%
			\:Else\\%
				\:Disp\,\qt D\,WINS\qt\\%
			\:End\\%
		\:End\\%
	\:End
\end{ticalc}
\end{minipage}

\end{frame}




% \begin{frame}
% \frametitle{Antwoorden}
% 
% \begin{minipage}{0.4\textwidth}
% \begin{ticalc}
% 	PROGRAM\:POKER\\%
% 	\,Input\,\qt NUM1\qt\comma A\\%
% 	\,Input\,\qt COL1\qt\comma B\\%
% 	\,Input\,\qt NUM2\qt\comma C\\%
% 	\,Input\,\qt COL2\qt\comma D\\%
% 	\,Input\,\qt NUM3\qt\comma E\\%
% 	\,Input\,\qt COL3\qt\comma F\\%
% 	\,Input\,\qt NUM4\qt\comma G\\%
% 	\,Input\,\qt COL4\qt\comma H\\%
% 	\,Input\,\qt NUM5\qt\comma I\\%
% 	\,Input\,\qt COL5\qt\comma J\\%
% \end{ticalc}
% \end{minipage}
% \begin{minipage}{0.4\textwidth}
% \begin{ticalc}
% \end{ticalc}
% \end{minipage}
% 
% \end{frame}



\begin{frame}
\frametitle{Antwoorden}

\begin{minipage}{0.48\textwidth}
\begin{ticalc}[4.5cm]
	PROGRAM\:RESTYDX\\%
	\:Disp\,\qt REST(Y/X)\:\qt\\%
	\:Input\,\qt Y=\qt\comma Y\\%
	\:Input\,\qt X=\qt\comma X\\%
	\:If\,X=0\\%
	\:Then\\%
	\:Disp\,\qt X\,CANNOT\,BE\,0\qt\\%
	\:Stop\\%
	\:End\\%
	\:Disp\,fPart(Y/X)*X\\%
	\:Disp\,(Y/X-int(Y/X))*X%
\end{ticalc}
\end{minipage}
\begin{minipage}{0.48\textwidth}
\begin{ticalc}[5.5cm]
	PROGRAM\:ABC\\%
	\:Disp\,\qt SOLVING\,AX\sq+BX+C=0\qt \\%
	\:Prompt\,A\comma B\comma C \\%
	\:B\sq-4AC\>D \\%
	\:If\,D>0\:Then\\%
	\:Disp\,\qt THERE\,ARE\,TWO\,SOLUTIONS\:\qt \\%
	\:(\min B+\sqrt(D))/(2A)\>X \\%
	\:Disp\,X \\%
	\:(\min B-\sqrt(D))/(2A)\>X \\%
	\:Disp\,X\\%
	\:Else\:If\,D=0\:Then\\%
	\:Disp\,\qt THERE\,IS\,ONE\,SOLUTION\:\qt \\%
	\:(\min B)/(2A)\>X \\%
	\:Disp\,X \\%
	\:Else\\%
	\:Disp\,\qt THERE\,IS\,NO\,SOLUTION!\qt \\%	
	\:End\\%
	\:End
\end{ticalc}
\end{minipage}




\end{frame}





