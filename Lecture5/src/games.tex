\section{Games!}

\subsection{Game Loop}

% Game Loop

\begin{frame}
\frametitle{Game Loop}

\begin{itemize}
  \item<1-> In tegenstelling tot veel van het voorgaande, wil je dat een game blijft spelen tot de gebruiker het afsluit.
  \item<2-> Dit kun je bereiken met een `infinite loop'.
  \item<3-> Daarin moet een aantal dingen gebeuren:
  \begin{enumerate}
    \item<4-> \visible<4->{Input (handle key presses)}
    \item<5-> \visible<5->{Update (rekenen)}
    \item<6-> \visible<6->{Draw (tekenen)} 
  \end{enumerate}
  \item<7-> Precieze vorm is afhankelijk van gametype:\\%
  			turn-based, real-time
\end{itemize}

\vspace{2.5cm}

\begin{tikzpicture}[overlay,remember picture]
	\node[yshift=0.6cm] (BR) at (current page.south east){ };
	\node [anchor=south east,xshift=0.195cm] at (BR)
	{%
	\only<3->{%
		\begin{ticalc}
			PROGRAM\:GAMELOOP\\%
			\:\qt INIT\,GAME\\%
			\:While\,1\\%
			\:\visible<4->{\qt INPUT\\%
			\:getKey\>K\\%
			\:If\,K=\ldots\\%
			\:\ldots}\\%
			\:\visible<5->{\qt UPDATE\\%
			\:L\sub{1}+UT(\E6+\E2)+\\%
				VT(\E4+1)\>L\sub{1}}\\%
			\:\visible<6->{\qt DRAW\\%
			\:L\sub{1}\:prgm\theta DRAWSPT}\\%
			\:End
		\end{ticalc}
	}%
	};
	\node [anchor=south east,xshift=-3.5cm] at (BR)
	{%
	\only<2->{%
		\begin{ticalc}
			PROGRAM\:INFILOOP\\%
			\:While\,1\\%
			\:do stuff\\%
			\:End
		\end{ticalc}
	}%
	};
\end{tikzpicture}

\end{frame}


\begin{frame}
\frametitle{Real-Time Games}

\hspace{-1.2cm}
\begin{minipage}{\textwidth}
\begin{itemize}
  \item \lenitem[0.45\linewidth]{In een real-time game gaat de game door, of je nou op een knop drukt of niet.}
\end{itemize}
\end{minipage}

\vspace{2.5cm}

\begin{tikzpicture}[overlay,remember picture]
	\node[yshift=0.6cm] (BR) at (current page.south east){ };
	\node [anchor=south east,xshift=0.195cm] at (BR)
	{%
	\only<1->{%
		\begin{ticalc}
			\:If\,K=105\:Then\\%
			\:Goto\,X\\%
			\:\qt UPDATE\\%
			\:Lbl\,UD
			\:End\\%
			\:L\sub{1}+U(\E6+\E2)+\\%
				V(\E4+1)\>L\sub{1}\\%
			\:\qt DRAW\\%
			\:L\sub{1}\:prgm\theta DRAWSPT\\%
			\:End\\%
			\:Lbl\,X\\%
			\:\only<1>{Clean stuff up here, reset graph axis etc.}\only<2->{\select{Clean stuff up}\\\select{here, reset graph}\\\select{axis etc.}}
		\end{ticalc}
	}%
	};
	\node [anchor=south east,xshift=-3.5cm] at (BR)
	{%
	\only<1->{%
		\begin{ticalc}
			PROGRAM\:REALTIME\\%
			\:\qt INIT\,GAME\\%
			\:While\,1\\%
			\:\qt INPUT\\%
			\:getKey\>K\\%
			\:\select{If\,K=0}\:Then\\%
			\:Goto\,UD
			\:End\\%
			\:If\,K=25\:Then\\%
			\:V+1\>V\\%
			\:Goto\,UD
			\:End\\%
			\:If\,K=34\:Then\\%
			\:V-1\>V\\%
			\:Goto\,UD
			\:End\\%
			\:If\,K=24\,or\,K=26\:Then\\%
			\:U-25+K\>U\\%
			\:Goto\,UD
			\:End\\%
		\end{ticalc}
	}%
	};
\end{tikzpicture}

\end{frame}


\begin{frame}
\frametitle{Turn-Based Games}

\hspace{-1.2cm}
\begin{minipage}{\textwidth}
\begin{itemize}
  \item \lenitem[0.45\linewidth]{In een turn-based game gebeurt er alleen iets nieuws wanneer er op de knop wordt gedrukt.}
\end{itemize}
\end{minipage}

\vspace{2.5cm}

\begin{tikzpicture}[overlay,remember picture]
	\node[yshift=0.6cm] (BR) at (current page.south east){ };
	\node [anchor=south east,xshift=0.195cm] at (BR)
	{%
	\only<1->{%
		\begin{ticalc}
			\:If\,K=105\:Then\\%
			\:Goto\,X\\%
			\:\qt UPDATE\\%
			\:Lbl\,UD
			\:End\\%
			\:L\sub{1}+U(\E6+\E2)+\\%
				V(\E4+1)\>L\sub{1}\\%
			\:0\>U\:0\>V\\%
			\:\qt DRAW\\%
			\:L\sub{1}\:prgm\theta DRAWSPT\\%
			\:End\\%
			\:Lbl\,X\\%
			\:\only<1>{Clean stuff up here, reset graph axis etc.}\only<2->{\select{Clean stuff up}\\\select{here, reset graph}\\\select{axis etc.}}
		\end{ticalc}
	}%
	};
	\node [anchor=south east,xshift=-3.5cm] at (BR)
	{%
	\only<1->{%
		\begin{ticalc}
			PROGRAM\:TURNGAME\\%
			\:\qt INIT\,GAME\\%
			\:While\,1\\%
			\:\qt INPUT\\%
			\:\select{Repeat\,Ans}\\%
			\:getKey\>K\\%
			\:\select{end}\\%
			\:If\,K=25\:Then\\%
			\:1\>V\\%
			\:Goto\,UD
			\:End\\%
			\:If\,K=34\:Then\\%
			\:\min\one\>V\\%
			\:Goto\,UD
			\:End\\%
			\:If\,K=24\,or\,K=26\:Then\\%
			\:K-25\>U\\%
			\:Goto\,UD
			\:End\\%
		\end{ticalc}
	}%
	};
\end{tikzpicture}

\end{frame}





\begin{frame}
\frametitle{Turn-Based Games with Animations}

\hspace{-1.2cm}
\begin{minipage}{\textwidth}
\begin{itemize}
  \item \lenitem[0.45\linewidth]{Indien animaties door moeten gaan, moet je iets slimmer zijn\ldots}
\end{itemize}
\end{minipage}

\vspace{2.5cm}

\begin{tikzpicture}[overlay,remember picture]
	\node[yshift=0.6cm] (BR) at (current page.south east){ };
	\node [anchor=south east,xshift=0.195cm] at (BR)
	{%
	\only<1->{%
		\begin{ticalc}
			\:If\,K=105\:Then\\%
			\:Goto\,X\\%
			\:Lbl\,UD
			\:End\\%
			\:\qt PROCESS\,TURN\\%
			\:L\sub{1}+U(\E6+\E2)+\\%
				V(\E4+1)\>L\sub{1}\\%
			\:0\>U\:0\>V\\%
			\:\select{End}\\%
			\:\qt ANIMATE\\%
			\:Update sprite animation here\\%
			\:\qt DRAW\\%
			\:L\sub{1}\:prgm\theta DRAWSPT\\%
			\:End\\%
			\:Lbl\,X\\%
			\:Clean stuff up here, reset graph axis etc.
		\end{ticalc}
	}%
	};
	\node [anchor=south east,xshift=-3.5cm] at (BR)
	{%
	\only<1->{%
		\begin{ticalc}
			PROGRAM\:TURNGAM2\\%
			\:\qt INIT\,GAME\\%
			\:While\,1\\%
			\:\qt INPUT\\%
			\:getKey\>K\\%
			\:\select{If\,K\!0\:Then}\\%		
			\:If\,K=25\:Then\\%
			\:1\>V\\%
			\:Goto\,UD
			\:End\\%
			\:If\,K=34\:Then\\%
			\:\min\one\>V\\%
			\:Goto\,UD
			\:End\\%
			\:If\,K=24\,or\,K=26\:Then\\%
			\:K-25\>U\\%
			\:Goto\,UD
			\:End\\%
		\end{ticalc}
	}%
	};
\end{tikzpicture}

\end{frame}



\begin{frame}
\frametitle{Turn-Based Games: Wait for Animation}

\hspace{-1.2cm}
\begin{minipage}{\textwidth}
\begin{itemize}
  \item<1-> \lenitem[0.45\linewidth]{Het kan logisch zijn om te wachten tot een animatie klaar is in een turn-based game, voordat de volgende zet gedaan mag worden.}
  \item<2-> \lenitem[0.45\linewidth]{Dit voorbeeld negeert de input (\tifonttxt{K}) totdat de animatie (\tifonttxt{A}) klaar is.}
\end{itemize}
\end{minipage}

\vspace{2.5cm}

\begin{tikzpicture}[overlay,remember picture]
	\node[yshift=0.6cm] (BR) at (current page.south east){ };
	\node [anchor=south east,xshift=0.195cm] at (BR)
	{%
	\only<1->{%
		\begin{ticalc}
			\:If\,K=105\:Then\\%
			\:Goto\,X\\%
			\:Lbl\,UD
			\:End\\%
			\:\qt PROCESS\,TURN\\%
			\:L\sub{1}+U(\E6+\E2)+\\%
				V(\E4+1)\>L\sub{1}\\%
			\:0\>U\:0\>V\\%
			\:\select{End}\\%
			\:\qt ANIMATE\\%
			\:Update sprite animation here\\%
			\:\visible<2->{\select{If\,animation}\\\select{finished\:0\>A}}\\%
			\:\qt DRAW\\%
			\:L\sub{1}\:prgm\theta DRAWSPT\\%
			\:End\\%
			\:Lbl\,X\\%
			\:Clean stuff up here, reset graph axis etc.
		\end{ticalc}
	}%
	};
	\node [anchor=south east,xshift=-3.5cm] at (BR)
	{%
	\only<1->{%
		\begin{ticalc}
			PROGRAM\:TURNGAM3\\%
			\:\qt INIT\,GAME\\%
			\:While\,1\\%
			\:\qt INPUT\\%
			\:getKey\>K\\%
			\:\visible<2->{\select{If\,A=1\:K=0}}\\%
			\:\select{If\,K\!0\:Then}\\%
			\:\visible<2->{\select{1\>A}}\\%
			\:If\,K=25\:Then\\%
			\:1\>V\\%
			\:Goto\,UD
			\:End\\%
			\:If\,K=34\:Then\\%
			\:\min\one\>V\\%
			\:Goto\,UD
			\:End\\%
			\:If\,K=24\,or\,K=26\:Then\\%
			\:K-25\>U\\%
			\:Goto\,UD
			\:End\\%
		\end{ticalc}
	}%
	};
\end{tikzpicture}

\end{frame}




\subsection{Towards multiplayer\ldots}

% Kort hoe ze verder kunnen om games multiplayer te maken:
% - Belangrijkste overwegingen
% - Limitaties


\begin{frame}
\frametitle{Wat is een multiplayer game?}

\begin{itemize}
  \item<1-> Meerdere spelers spelen hetzelfde spel:
  \begin{itemize}
    \item<2-> Tegen elkaar (PvP)
    \item<2-> Samen met elkaar (coop)
  \end{itemize}
  \item<3-> Meerdere manieren:
  \begin{itemize}
    \item<4-> One Device
    \item<4-> Two Devices
  \end{itemize}
  \item<5-> Ik geef slechtst een korte overview. Details staan hier:
  \begin{itemize}
    \item<5-> http://tibasicdev.wikidot.com/multiplayer
  \end{itemize}  
\end{itemize}

\end{frame}





\begin{frame}
\frametitle{Multiplayer: One Device}

\begin{itemize}
  \item<1-> Dit is vrijwel hetzelfde als een single player game, maar nu:
  \begin{itemize}
    \item<2-> Andere knoppen besturen andere sprites: meer \inlineticalc{If\,K=\ldots} statements.
  \end{itemize}
  \item<3-> BELANGRIJK: Slechts 1 knop kan tegelijk ingedrukt worden.
  \begin{itemize}
    \item<4-> PROBLEEM: Indien je \tifonttxt{prgm} langzaam is, dan komen niet beide spelers aan de beurt; alleen degene die het snelste drukt.
    \item<5-> PAS OP: Pijltoetsen kunnen ingedrukt worden gehouden. Gebruik die knoppen dus NIET.
    \item<6-> Geen probleem voor turn-based games!\\
    		  Turn-based is een aanrader voor multiplayer!
  \end{itemize}
\end{itemize}

\end{frame}






\begin{frame}
\frametitle{Multiplayer: Two Devices}

\begin{itemize}
  \item<1-> Het principe is simpel:
  \begin{itemize}
    \item<2-> Connect een linkabel.
    \item<3-> Beide spelers starten hetzelfde \tifonttxt{prgm}.
    \item<4-> Beide spelers kunnen hun eigen knoppen gebruiken.
  \end{itemize}
  \item<5-> Linken van variabelen gaat via de \tifonttxt{GetCalc(varname)} functie.
  \item<6-> PROBLEEM: Linken kan alleen wanneer \'e\'en rekenmachine niets doet, bijvoorbeeld door \tifonttxt{Pause}.
  \item<7-> Daarom \underline{alleen mogelijk voor turn-based games}.
  \begin{itemize}
    \item<8-> Maar dat kan net zo goed op \'e\'en device\ldots
  \end{itemize}  
\end{itemize}

\end{frame}












