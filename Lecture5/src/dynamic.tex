\section{Dynamic input}

\subsection{getKey}

% getKey command
\begin{frame}
\frametitle{Check for key presses}

\visible<1>{\ticalcfig{\ticalcfigCircle{\ticalcfigCircleColThree}{0.615}}} % prgm

\begin{itemize}
  \item<1-> \lenitem[0.75\linewidth]{Met \tifonttxt{getKey} kun je de \emph{laatste} knop die ingedrukt is opvragen.}
  \item<2-> \lenitem[0.75\linewidth]{Je krijgt een getal, welk naar een knop verwijst (zie figuur).}
\end{itemize}



\vspace{2.5cm}



\begin{tikzpicture}[overlay,remember picture]
	\node[yshift=0.6cm] (BR) at (current page.south east){ };
	\node [anchor=south east,xshift=0.195cm] at (BR)
	{%
	\only<2->{%
		\begin{tikzonimage}[width=3.2cm]{./TI84_getkey.jpg}
			\only<6>{\ticalcfigCircle{0.822}{0.113}}
		\end{tikzonimage}
	}
	};
	\node [anchor=south east,xshift=-3.5cm] at (BR)
	{%
	\only<1->{%
		\begin{ticalc}
			CTL\,\select{I/O}\,EXEC \\
			1\+\:Input \\
			2\:Prompt \\
			3\:Disp \\
			4\:DispGraph \\
			5\:DispTable \\
			6\:Output( \\
			\selectitem{7\arrowdown} getKey
		\end{ticalc}
	}%
	};
	\node [anchor=south east,xshift=-7.195cm] at (BR)
	{%
	\only<3->{%
		\begin{ticalc}
			getKey\\%
			\hfill0\\%
			\visible<4->{getKey\\%
			\hfill0\\}%
			\visible<5->{getKey\\%
			\hfill0\\}%
			\visible<6->{getKey\\%
			\hfill105}
		\end{ticalc}
	}%
	};
\end{tikzpicture}

\end{frame}









\subsection{wait for input}

% use getKey, freeze prgm until button press


\begin{frame}
\frametitle{Wait for keypress}


\begin{itemize}
  \item<1-> \lenitem[0.75\linewidth]{\tifonttxt{prgmGETKEY1} wacht tot een knop ingedrukt wordt.}
  \item<2-> \lenitem[0.75\linewidth]{\tifonttxt{prgmGETKEY2} blijft loopen totdat er op \tiENTER\,gedrukt wordt.}
\end{itemize}



\vspace{3cm}

\begin{tikzpicture}[overlay,remember picture]
	\node[yshift=0.6cm] (BR) at (current page.south east){ };
	\node [anchor=south east,xshift=0.195cm] at (BR)
	{%
	\only<1->{%
		\begin{tikzonimage}[width=3.2cm]{./TI84_getkey.jpg}
			\only<2>{\ticalcfigCircle{0.822}{0.113}}
		\end{tikzonimage}
	}
	};
	\node [anchor=south east,xshift=-3.5cm] at (BR)
	{%
	\only<1->{%
		\begin{ticalc}
			PROGRAM\:GETKEY1\\%
			\:Repeat\,Ans\\%
			\:getKey\>X\\%
			\:Disp\,X\\%
			\:End
		\end{ticalc}
	}%
	};
	\node [anchor=south east,xshift=-7.195cm] at (BR)
	{%
	\only<2->{%
		\begin{ticalc}
			PROGRAM\:GETKEY2\\%
			\:0\>K\\%
			\:While\,K\!105\\%
			\:getKey\>K\\%
			\:If\,K\!0
			\:Disp\,K\\%
			\:End
		\end{ticalc}
	}%
	};
\end{tikzpicture}

\end{frame}




% \subsection{event --> action}


\begin{frame}
\frametitle{Hoe reageer je op specifieke input?}

\vspace{-1cm}

\begin{itemize}
  \item \lenitem[0.7\linewidth]{Hoe kunnen we onze kennis van \tifonttxt{getKey} gebruiken om te reageren op specifieke knoppen?}
  \begin{itemize}
    \item Bijv. reageer op pijltoetsen voor game/menu.
  \end{itemize}
\end{itemize}

\vspace{3cm}

\begin{tikzpicture}[overlay,remember picture]
	\node[yshift=0.6cm] (BR) at (current page.south east){ };
	\node [anchor=south east,xshift=0.195cm] at (BR)
	{%
	\only<3->{%
		\begin{tikzonimage}[width=3.2cm]{./TI84_getkey.jpg}
			\ticalcfigCircle{0.75}{0.568} % Up
			\ticalcfigCircle{0.628}{0.53} % Left
% 			\ticalcfigCircle{0.822}{0.113} % Enter
		\end{tikzonimage}
	}
	};
	\node [anchor=south east,xshift=-3.5cm] at (BR)
	{%
	\only<2->{%
		\begin{ticalc}
			PROGRAM\:AROWKEY1\\%
			\:0\>K\:Repeat\,K\\%
			\:getKey\>X\\%
			\:If\,K=25\:Then\\%
			\:Disp\,\qt MOVE\,UP\\%
			\:End\\%
			\:If\,K=24\:Then\\%
			\:Disp\,\qt MOVE\,<-\\%
			\:End\\%
			\:End
		\end{ticalc}
	}%
	};
	\node [anchor=south east,xshift=-7.195cm] at (BR)
	{%
	\only<4->{%
		\begin{ticalc}
			PROGRAM\:AROWKEY2\\%
			\:Lbl\,0
			\:0\>K\\%
			\:Repeat\,K\\%
			\:getKey\>K\\%
			\:End\\%
			\:If\,K=25\:Then\\%
			\:Disp\,\qt MOVE\,UP\\%
			\:Else\\%
			\:If\,K=24\:Then\\%
			\:Disp\,\qt MOVE\,<-\\%
			\:End\:End\\%
			\:Goto\,0
		\end{ticalc}
	}%
	};
\end{tikzpicture}

\end{frame}





\begin{frame}
\frametitle{Event \tifonttxt{\>} Action programming}

\vspace{-1cm}

\begin{itemize}
  \item<2-> \lenitem[0.7\linewidth]{Een betere manier is \tifonttxt{PROGRAM\:AROWKEY3}.}
  \item<3-> Merk op hoe raar de \tifonttxt{End}'s staan
  \begin{itemize}
    \item<4-> \tifonttxt{Goto} springt in het programma, dus klopt wel!
  \end{itemize}
\end{itemize}

\vspace{3cm}

\begin{tikzpicture}[overlay,remember picture]
	\node[yshift=0.6cm] (BR) at (current page.south east){ };
	\node [anchor=south east,xshift=0.195cm] at (BR)
	{%
	\only<1->{%
		\begin{tikzonimage}[width=3.2cm]{./TI84_getkey.jpg}
			\ticalcfigCircle{0.75}{0.568} % Up
			\ticalcfigCircle{0.628}{0.53} % Left
 			\only<2->{\ticalcfigCircle{0.822}{0.113}} % Enter
		\end{tikzonimage}
	}
	};
	\node [anchor=south east,xshift=-3.5cm] at (BR)
	{%
	\only<1->{%
		\begin{ticalc}
			PROGRAM\:AROWKEY1\\%
			\:0\>K\:Repeat\,K\\%
			\:getKey\>X\\%
			\:If\,K=25\:Then\\%
			\:Disp\,\qt MOVE\,UP\\%
			\:End\\%
			\:If\,K=24\:Then\\%
			\:Disp\,\qt MOVE\,<-\\%
			\:End\\%
			\:End
		\end{ticalc}
	}%
	};
	\node [anchor=south east,xshift=-7.195cm] at (BR)
	{%
	\only<2->{%
		\begin{ticalc}
			PROGRAM\:AROWKEY3\\%
			\:If\,1=1\:Then\\%
			\:\only<-3>{Lbl\,0}\only<4->{\select{Lbl\,0}}
			\:\only<-2>{End}\only<3->{\select{End}}\\%
			\:getKey\>K\\%
			\:If\,K=0\:Then\\%
			\:Goto\,0
			\:End\\%
			\:\only<-3>{If\,K=25\:Then}\only<4->{\select{If\,K=25}\:\select{Then}}\\%
			\:Disp\,\qt MOVE\,UP\\%
			\:\only<-3>{Goto\,0}\only<4->{\select{Goto\,0}}
			\:End\\%
			\:If\,K=24\:Then\\%
			\:Disp\,\qt MOVE\,<-\\%
			\:Goto\,0
			\:End\\%
			\:If\,K\!105\:Then\\%
			\:Goto\,0
		\end{ticalc}
	}%
	};
\end{tikzpicture}

\end{frame}


\subsection{Memory Leaks}


\begin{frame}
\frametitle{Intermezzo: Memory Leaks}

\vspace{-1cm}
\hspace{-1cm}
\begin{minipage}{\textwidth}
\begin{itemize}
  \item<2-> \lenitem[\linewidth]{Als je in een block statement (zoals \tifonttxt{If} of \tifonttxt{While}) gaat, onthoud de rekenmachine hoe diep je bent: dit heet de `stack'.}
  \item<3-> \lenitem[0.75\linewidth]{Wanneer je met \tifonttxt{Goto} naar buiten springt, ga je steeds dieper en dieper.}
  \item<4-> \lenitem[0.75\linewidth]{Dit kost memory\ldots\visible<5->{Vertraagt je \tifonttxt{prgm}\ldots}\\\visible<6->{En uiteindelijk crashed je \tifonttxt{prgm}\ldots}}
  \item<7-> \lenitem[0.8\linewidth]{De memory leak verdwijnt pas zodra \tifonttxt{prgm} eindigt.}
  \item<8-> \lenitem[0.75\linewidth]{http://tibasicdev.wikidot.com/memory-leaks}
\end{itemize}
\end{minipage}


\begin{tikzpicture}[overlay,remember picture]
	\node[yshift=0.6cm] (BR) at (current page.south east){ };
	\node [anchor=south east,xshift=0.195cm] at (BR)
	{%
% 	\only<1->{%
% 		\begin{tikzonimage}[width=3.2cm]{./TI84_getkey.jpg}
% 			\ticalcfigCircle{0.75}{0.568} % Up
% 			\ticalcfigCircle{0.628}{0.53} % Left
%  			\only<2->{\ticalcfigCircle{0.822}{0.113}} % Enter
% 		\end{tikzonimage}
% 	}
	\only<1->{%
		\begin{ticalc}
			PROGRAM\:AROWKEY3\\%
			\:If\,1=1\:Then\\%
			\:Lbl\,0
			\:\select{End}\\%
			\:getKey\>K\\%
			\:If\,K=0\:Then\\%
			\:Goto\,0
			\:End\\%
			\:If\,K=25\:Then\\%
			\:Disp\,\qt MOVE\,UP\\%
			\:Goto\,0
			\:End\\%
			\:If\,K=24\:Then\\%
			\:Disp\,\qt MOVE\,<-\\%
			\:Goto\,0
			\:End\\%
			\:If\,K\!105\:Then\\%
			\:Goto\,0
		\end{ticalc}
	}%
	};
	\node [anchor=south east,xshift=-3.5cm] at (BR)
	{%
	\only<3->{%
		\begin{ticalc}
			\:PROGRAM\:MEMLEAK\\%
			\:Lbl\,A\\%
			\:While\,1\\%
			\:Goto\,A\\%
			\:End
		\end{ticalc}
	}%
	};
	\node [anchor=south east,xshift=-7.195cm] at (BR)
	{%
	\only<4->{%
		\begin{ticalc}
			prgmMEMLEAK\\%
			\visible<6->{ERR\:MEMORY}
		\end{ticalc}
	}%
	};
\end{tikzpicture}



\end{frame}






\begin{frame}
\frametitle{Exercise}

\lenitem[0.7\linewidth]{Maak een ``custom menu'': een menu getekent met \tiDRAW-functies bestuurt via \tifonttxt{getKey}.}

\lenitem[0.7\linewidth]{\flushleft{Maak het zo fancy als je wilt (cursor met animatie :)?), maar \underline{begin simpel}.}}


\begin{tikzpicture}[overlay,remember picture]
	\node[yshift=0.6cm] (BR) at (current page.south east){ };
	\node [anchor=south east,xshift=0.195cm] at (BR)
	{%
	\only<1->{%
		\begin{tikzonimage}[width=3.2cm]{./TI84_getkey.jpg}
			\ticalcfigCircle{0.75}{0.568} % Up
			\ticalcfigCircle{0.75}{0.488} % Down
 			\ticalcfigCircle{0.822}{0.113} % Enter
		\end{tikzonimage}
	}
	};
	\node [anchor=south east,xshift=-3.5cm] at (BR)
	{%
	\only<1->{%
		\begin{ticalc}
			\tiny{WHAT\,PIE?}\\%
			\tiny{\visible<1>{-}\,APPLE}\\%
			\tiny{\visible<2>{-}\,BANANA\,CREAM}\\%
			\tiny{\visible<3>{-}\,CHERRY}
		\end{ticalc}
	}%
	};
% 	\node [anchor=south east,xshift=-7.195cm] at (BR)
% 	{%
% 	\only<4->{%
% 		\begin{ticalc}
% 			prgmMEMLEAK\\%
% 			\visible<6->{ERR\:MEMORY}
% 		\end{ticalc}
% 	}%
% 	};
\end{tikzpicture}



\end{frame}



\begin{frame}
\frametitle{Answer}

\hspace{-1cm}
\begin{minipage}{0.44\textwidth}
	\begin{ticalc}[6.7cm]
		PROGRAM\:CUSTMENU\\%
		\:ClrDraw\:0\>I\\%
		\:Text(10\comma20\comma\qt APPLE\\%
		\:Text(20\comma20\comma\qt BANANA\,CREAM\\%
		\:Text(30\comma20\comma\qt CHERRY\\%
		\:Text(40\comma20\comma\qt DERBY\\%
		\:Text(50\comma20\comma\qt EMPANADA\\%
		\:0\>K\:Repeat\,K=105\\%
		\:getKey\>K\\%
		\:I\>J\\%
		\:If\,K=25\:I-1\>I\\%
		\:If\,K=34\:I+1\>I\\%
		\:min(4\comma max(0\comma I\>I\\%
		\:\\%
		\:Text(10+10I\comma10\comma\qt-\\%
		\:If\,I\!J\:Text(10+10J\comma10\comma\qt\tiSpace\tiSpace\tiSpace\\%
		\:End\\%
		\:Disp\,\qt YOU\,CHOSE\qt\comma I+1
	\end{ticalc}
\end{minipage}
\fcolorbox{black}{white}{%
\begin{minipage}{0.52\textwidth}
	Met simpele animatie (8 frames):
	
	\begin{ticalc}[5.3cm]
		\:\ldots\\%
		\:ClrDraw\:0\>I\:0\>A\\%
		\:\ldots\\%
		\:8fPart((A+1)/8\>A\\%
		\:Text(10+10I\comma10\comma\qt-\\%
		\:Text(10+10I\comma6+A\comma\qt\tiSpace\tiSpace\\%
		\:\ldots
	\end{ticalc}
\end{minipage}
}



\end{frame}


% 	\node [anchor=south east,xshift=-7.195cm] at (BR)
% 	{%
% 	\only<4->{%
% 		\begin{ticalc}
% 			PROGRAM\:AROWKEY2\\%
% 			\:Lbl\,0
% 			\:0\>X\\%
% 			\:Repeat\,X\\%
% 			\:getKey\>X\\%
% 			\:End\\%
% 			\:If\,X=25\:Then\\%
% 			\:Disp\,\qt MOVE UP\qt\\%
% 			\:Else\\%
% 			\:If\,X=105\:Then\\%
% 			\:Disp\,\qt END\qt\\%
% 			\:End
% 			\:End\\%
% 			\:Goto\,0
% 		\end{ticalc}
% 	}%
% 	};

% use getKey, continue prgm and process input if buton was pressed
% pseudocode!



% 1: in een real-time game wil je niet wachten op button press
%    je wilt game afspelen en iets speciaals doen (action) wanneer er een button ingedrukt wordt (event)

% 2: code




% Intermediate exercise: make a custom menu!


