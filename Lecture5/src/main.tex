%% LaTeX Beamer presentation template (requires beamer package)
%% see http://bitbucket.org/rivanvx/beamer/wiki/Home
%% idea contributed by H. Turgut Uyar
%% template based on a template by Till Tantau
%% this template is still evolving - it might differ in future releases!

\documentclass{beamer}

\mode<presentation>
{
\usetheme{Warsaw}

\setbeamercovered{transparent}
}

\usepackage{../../Common/sty/TI84} %TODO: How to get the link working...?



\title{Masterclass programmeren op de GR TI-84 (les 5)}

%\subtitle{}

% - Use the \inst{?} command only if the authors have different
%   affiliation.
%\author{F.~Author\inst{1} \and S.~Another\inst{2}}
\author{Kevin van As}

% - Use the \inst command only if there are several affiliations.
% - Keep it simple, no one is interested in your street address.
% \institute[Universities of]
% {
% \inst{1}%
% Department of Computer Science\\
% Univ of S
% \and
% \inst{2}%
% Department of Theoretical Philosophy\\
% Univ of E}

\date{\today}


% This is only inserted into the PDF information catalog. Can be left
% out.
\subject{Masterclass GR TI-84 programmeren (les 5)}



% If you have a file called "university-logo-filename.xxx", where xxx
% is a graphic format that can be processed by latex or pdflatex,
% resp., then you can add a logo as follows:

% \pgfdeclareimage[height=0.5cm]{university-logo}{university-logo-filename}
% \logo{\pgfuseimage{university-logo}}



% Delete this, if you do not want the table of contents to pop up at
% the beginning of each subsection:
\AtBeginSubsection[]
{
\begin{frame}<beamer>
\frametitle{Outline}
\tableofcontents[currentsection,currentsubsection]
\end{frame}
}

% If you wish to uncover everything in a step-wise fashion, uncomment
% the following command:

%\beamerdefaultoverlayspecification{<+->}

\begin{document}

\begin{frame}
\titlepage
\end{frame}


\begin{frame}
\frametitle{Recap!}


We hebben gekeken naar:
\begin{itemize}
	\item<2-> Advanced I/O
		\begin{itemize}
		  \item<3-> \tifonttxt{Menu}s
		  \item<4-> \tifonttxt{Output}
		\end{itemize}
	\item<6-> Precisie en afrondingsfouten
	\item<7-> Debugging
	\item<8-> Eigen functions definieren
\end{itemize}
\end{frame}

\begin{frame}
\frametitle{Vooruitzicht!}

Vandaag zullen we kijken naar:
\begin{itemize}
	\item<2-> Dynamische \tifonttxt{prgm}s:
		\begin{itemize}
		  \item<3-> Real-Time input: \tifonttxt{getKey}
		  \item<4-> Linking calculators: \tifonttxt{Get} \& \tifonttxt{Send}
		\end{itemize}
	\item<5-> Graphics
	\item<6-> Introductie Games
\end{itemize}

\end{frame}


%% END %%

\begin{frame}
\frametitle{Outline}
\tableofcontents
% You might wish to add the option [pausesections]
\end{frame}

% The core pages
\section{Graphics}

\subsection{Friendly window}

\begin{frame}
\frametitle{Friendly window}

\visible<5>{\ticalcfig{\ticalcfigCircle{\ticalcfigCircleColThree}{0.944}}} % zoom
\visible<6->{\ticalcfig{\ticalcfigCircle{\ticalcfigCircleColFour}{0.62}}} % vars

\begin{itemize}
  \item<1-> Om iets te tekenen, hebben we coordinaten nodig.
  \item<2-> Een mooi coordinatenstelsel is handig!
  \item<3-> Kies zodat elke pixel ``1'' waard is.
  \begin{itemize}
    \item<4-> Coordinaten $x\in[0,94]$ en $y\in[0,62]$
  \end{itemize}
  \item<7-> See also: \\
  	http://tibasicdev.wikidot.com/friendly-window
\end{itemize}

\vspace{3cm}

\begin{tikzpicture}[overlay,remember picture]
	\node[yshift=0.6cm] (BR) at (current page.south east){ };
	\node [anchor=south east,xshift=0.195cm] at (BR)
	{%
	\only<4->{%
		\begin{ticalc}
			\:ZStandard\\%
			\:84\>Xmin\\%
			\:72\>Ymax\\%
			\:ZInteger
		\end{ticalc}
	}%
	};
	\node [anchor=south east,xshift=-3.5cm] at (BR)
	{%
	\only<5>{%
		\begin{ticalc}
			\select{ZOOM}\,MEMORY\\%
			3\arrowup Zoom\,Out\\%
			4\:ZDecimal\\%
			5\:ZSquare\\%
			\selectitem{6\:}ZStandard\\%
			7\:ZTrig\\%
			\selectitem{8\:}ZInteger\\%
			9\arrowdown ZoomStat
		\end{ticalc}
	}%
	\only<6->{%
		\begin{ticalc}
			\select{VARS}\,Y-VARS\\%
			\selectitem{\one\:}Window\ldots\\%
			2\:Zoom\ldots\\%
			3\:GDB\ldots\\%
			4\:Picture\ldots\\%
			5\:Statistics\ldots\\%
			6\:Table\ldots\\%
			7\:String\ldots
		\end{ticalc}
	}%
	};
	\node [anchor=south east,xshift=-7.195cm] at (BR)
	{%
	\only<6->{%
		\begin{ticalc}
			\select{X/Y}\,T/\theta\,U/V/W\\%
			\selectitem{\one\:}Xmin\\%
			2\:Xmax\\%
			3\:Xscl\\%
			4\:Ymin\\%
			\selectitem{5\:}Ymax\\%
			6\:Yscl\\%
			7\arrowdown Xres
		\end{ticalc}
	}%
	};
\end{tikzpicture}


\end{frame}




\begin{frame}
\frametitle{\tifonttxt{prgm\theta FWINDOW}}

\visible<1->{\ticalcfig{\ticalcfigCircleSecond\ticalcfigCircle{\ticalcfigCircleColThree}{0.944}}} % format

\begin{itemize}
  \item \lenitem{Het is handig om een subprogramma te maken om een friendly window te activeren.}
  \item \lenitem{Maak dit \tifonttxt{prgm} na!}
\end{itemize}

\vspace{3cm}


\begin{tikzpicture}[overlay,remember picture]
	\node[yshift=0.6cm] (BR) at (current page.south east){ };
	\node [anchor=south east,xshift=0.195cm] at (BR)
	{%
	\only<1->{%
		\begin{ticalc}[3.5cm]
			PROGRAM\:\theta FWINDOW\\%
			\:FnOff\\%
			\:GridOff\\%
			\:AxesOff\\%
			\:ZStandard\\%
			\:84\>Xmin\\%
			\:72\>Ymax\\%
			\:ZInteger\\%
			\:ClrDraw
		\end{ticalc}
	}%
	};
\end{tikzpicture}

\end{frame}



\subsection{Basic drawing functions}

% 1) shapes (line,circle,..)
% 2) pixel on/off

\begin{frame}
\frametitle{Schermgrootte \& Draw line}

\visible<3->{\ticalcfig{\ticalcfigCircleSecond\ticalcfigCircle{\ticalcfigCircleColThree}{0.615}}} % draw

\begin{itemize}
  \item<1-> TI-84: 96x64 pixels
	  \begin{itemize}
	  	\item<2-> \lenitem{Om te tekenen: $x\in[0,94]$ en $y\in[0,62]$}
	  \end{itemize}
  \item<3-> In het \tiDRAW-menu staan functies om te tekenen.
  \item<4-> \tifonttxt{prgmOUTLINE} kleurt alle buitenste pixels.
\end{itemize}

\vspace{2cm}


\begin{tikzpicture}[overlay,remember picture]
	\node[yshift=0.6cm] (BR) at (current page.south east){ };
	\node [anchor=south east,xshift=0.195cm] at (BR)
	{%
	\only<3->{%
		\begin{ticalc}
			\select{DRAW}\,POINTS\,STO\\%
			\one\:ClrDraw\\%
			\selectitem{2\:}Line(\\%
			3\:Horizontal\\%
			4\:Vertical\\%
			5\:Tangent(\\%
			6\:DrawF\\%
			7\arrowdown Shade(
		\end{ticalc}
	}%
	};
	\node [anchor=south east,xshift=-3.5cm] at (BR)
	{%
	\only<4->{%
		\begin{ticalc}[3.5cm]
			PROGRAM\:OUTLINE
			\:ZStandard\\%
			\:84\>Xmin\\%
			\:72\>Ymax\\%
			\:ZInteger\\%
			\:Line(0\comma0\comma0\comma62)\\%
			\:Line(0\comma0\comma94\comma0)\\%
			\:Line(0\comma62\comma94\comma62)\\%
			\:Line(94\comma0\comma94\comma62)
		\end{ticalc}
	}%
	};
\end{tikzpicture}

\end{frame}



\begin{frame}
\frametitle{Let's draw!}

\visible<1->{\ticalcfig{\ticalcfigCircleSecond\ticalcfigCircle{\ticalcfigCircleColThree}{0.615}}} % draw

\vspace{-1.8cm}

\begin{itemize}
  \item<1-> Alles om te tekenen staat in het \tiDRAW-menu.
  \item<2-> Teken eens:
  \begin{enumerate}
    \item Een lijn \tifonttxt{\>} een driehoek
    \item Een cirkel
    \item Een smily
    \item Wat je wilt!
  \end{enumerate}
  \item<3-> Let op hoe snel/handig alle functies zijn.
  \item<4-> \lenitem{Dit hoeft niet in een \tifonttxt{prgm}: kan gewoon op het hoofdscherm.}
\end{itemize}

\vspace{1cm}


\begin{tikzpicture}[overlay,remember picture]
	\node[yshift=0.6cm] (BR) at (current page.south east){ };
	\node [anchor=south east,xshift=0.195cm] at (BR)
	{%
	\only<1->{%
		\begin{ticalc}
			\select{DRAW}\,POINTS\,STO\\%
			\one\:ClrDraw\\%
			2\:Line(\\%
			3\:Horizontal\\%
			4\:Vertical\\%
			5\:Tangent(\\%
			6\:DrawF\\%
			7\arrowdown Shade(
		\end{ticalc}
	}%
	};
	\node [anchor=south east,xshift=-3.5cm] at (BR)
	{%
	\only<2->{%
		\begin{ticalc}[4.7cm]
			\:Line(5\comma5\comma35\comma35)\\%
			\:Circle(20\comma20\comma10)\\%
			\:DrawF\,X\^1.1\\%
			\:Shade(X\comma X\sq\comma0\comma10\comma1\comma1)\\%
			\:Shade(X\comma X\sq\comma10\comma20\comma1\comma2)\\%
			\:Shade(X\comma X\sq\comma20\comma30\comma1\comma4)\\%
			\:Shade(X\comma X\sq\comma30\comma40\comma2\comma4)\\%
			\:Shade(X\comma X\sq\comma40\comma99\comma3\comma7)
		\end{ticalc}
	}%
	};
	\node [anchor=south east,xshift=-8.645cm] at (BR)
	{%
	\only<1->{%
		\begin{ticalc}
			\:ZStandard\\%
			\:84\>Xmin\\%
			\:72\>Ymax\\%
			\:ZInteger
		\end{ticalc}
	}%
	};
\end{tikzpicture}
  
\end{frame}


\begin{frame}
\frametitle{\tiDRAW}

\visible<1->{\ticalcfig{\ticalcfigCircleSecond\ticalcfigCircle{\ticalcfigCircleColThree}{0.615}}} % draw

\begin{itemize}
  \item<1-> Alles om te tekenen staat in het \tiDRAW-menu.
  \item<2-> Een lijn tekenen is vrij snel\ldots
  \item<3-> Maar de \tifonttxt{Circle(} functie is zeer langzaam!
  \begin{itemize}
    \item<4-> Ongeschikt voor games!
  \end{itemize}
\end{itemize}

\vspace{2cm}


\begin{tikzpicture}[overlay,remember picture]
	\node[yshift=0.6cm] (BR) at (current page.south east){ };
	\node [anchor=south east,xshift=0.195cm] at (BR)
	{%
	\only<1->{%
		\begin{ticalc}
			\select{DRAW}\,POINTS\,STO\\%
			\one\:ClrDraw\\%
			2\:Line(\\%
			3\:Horizontal\\%
			4\:Vertical\\%
			5\:Tangent(\\%
			6\:DrawF\\%
			7\arrowdown Shade(
		\end{ticalc}
	}%
	};
\end{tikzpicture}

\end{frame}





\begin{frame}
\frametitle{Per pixel drawing}

\visible<1->{\ticalcfig{\ticalcfigCircleSecond\ticalcfigCircle{\ticalcfigCircleColThree}{0.615}}} % draw

\begin{itemize}
  \item<1-> \lenitem{Je kunt ook pixels/punten individueel aan en uit zetten.}
  \item<2-> Dit bevind zich onder \inlineticalc{POINTS} in het \tiDRAW-menu.
  \item<3-> Waarom je dit ooit zou doen?
  \visible<4->{\begin{itemize}
    \item<4-> Veel sneller!! Je wilt een real-time game!
  \end{itemize}}
  \item<5-> Maar kost veel geheugen om iets te tekenen\ldots
\end{itemize}


\begin{tikzpicture}[overlay,remember picture]
	\node[yshift=0.6cm] (BR) at (current page.south east){ };
	\node [anchor=south east,xshift=0.195cm] at (BR)
	{%
	\only<1>{%
		\begin{ticalc}
			\select{DRAW}\,POINTS\,STO\\%
			\selectitem{\one\:}ClrDraw\\%
			2\:Line(\\%
			3\:Horizontal\\%
			4\:Vertical\\%
			5\:Tangent(\\%
			6\:DrawF\\%
			7\arrowdown Shade(
		\end{ticalc}
	}%
	\only<2->{%
		\begin{ticalc}
			DRAW\,\select{POINTS}\,STO\\%
			\selectitem{\one\:}Pt-On(\\%
			2\:Pt-Off(\\%
			3\:Pt-Change(\\%
			4\:Pxl-On(\\%
			5\:Pxl-Off(\\%
			6\:Pxl-Change(\\%
			7\:pxl-Test(
		\end{ticalc}
	}%
	};
\end{tikzpicture}


\end{frame}





\begin{frame}
\frametitle{Per pixel drawing}

\visible<1->{\ticalcfig{\ticalcfigCircleSecond\ticalcfigCircle{\ticalcfigCircleColThree}{0.615}}} % draw

\begin{itemize}
  \item<1-> Punten (x,y) beginnen linksonderin.
  \item<2-> Pixels (R,C) beginnen linksbovenin.
  \begin{itemize}
    \item R (rij) telt verticaal \underline{van} boven \underline{naar} beneden!
    \item C (kolom) telt horizontaal
  \end{itemize}
  \item<3-> Bij welk punt hoort pixel (R,C)? (Tip: $y_{max}=62$)
  \visible<5->{\begin{itemize}
    \item Punt (C,62-R)
  \end{itemize}}
  \item<4-> En bij welke pixel hoort punt (x,y)? 
  \visible<5->{\begin{itemize}
    \item Pixel (62-y,x)
  \end{itemize}}
\end{itemize}


\begin{tikzpicture}[overlay,remember picture]
	\node[yshift=0.6cm] (BR) at (current page.south east){ };
	\node [anchor=south east,xshift=0.195cm] at (BR)
	{%
	\only<1->{%
		\begin{ticalc}
			DRAW\,\select{POINTS}\,STO\\%
			\only<2-3,5>{\one\:}\only<1,4>{\selectitem{\one\:}}Pt-On(\\%
			2\:Pt-Off(\\%
			3\:Pt-Change(\\%
			\only<1,4,5>{4\:}\only<2-3>{\selectitem{4\:}}Pxl-On(\\%
			5\:Pxl-Off(\\%
			6\:Pxl-Change(\\%
			7\:pxl-Test(
		\end{ticalc}
	}%
	};
\end{tikzpicture}


\end{frame}




\subsection{Sprites}

% Background image: StorePic & RecallPic
% Plot sprites
% 2*4 digits = line
% Animating a sprite



\begin{frame}
\frametitle{Sprite: Meerdere lijnen}

\begin{itemize}
  \item<1-> In een game wil je vaak een object tekenen, zoals pacman.
  \item<2-> Zo'n object heet een ``sprite''.
  \item<3-> Op de TI84 is het een combinatie van meerdere pixels of lijnen.
\end{itemize}

\vspace{3cm}


\begin{tikzpicture}[overlay,remember picture]
	\node[yshift=0.6cm] (BR) at (current page.south east){ };
	\node [anchor=south east,xshift=0.195cm] at (BR)
	{%
	\only<4->{%
		\begin{ticalc}[3.5cm]
			PROGRAM\:SPRITE\\%
			\:prgm\theta FWINDOW\\%
			\:Line(10\comma10\comma20\comma20\\%
			\:Line(20\comma20\comma30\comma10\\%
			\:Line(30\comma10\comma20\comma15\\%
			\:Line(20\comma15\comma10\comma10
		\end{ticalc}
	}%
	};
	\node [anchor=south east,xshift=-3.75cm] at (BR)
	{%
	\only<4->{%
		\begin{ticalc}[3.5cm]
			PROGRAM\:\theta FWINDOW\\%
			\:FnOff\\%
			\:GridOff\\%
			\:AxesOff\\%
			\:ZStandard\\%
			\:84\>Xmin\\%
			\:72\>Ymax\\%
			\:ZInteger\\%
			\:ClrDraw
		\end{ticalc}
	}%
	};
\end{tikzpicture}

\end{frame}


\begin{frame}
\frametitle{Plot Sprites}

\begin{itemize}
  \item<1-> Hoe kan dit effici\"enter?
  \visible<2->{\item<2-> Store alle getallen in een list!
  \item<3-> Met ``Plot'' kun je \textbf{maximaal 3} sprites maken.
  \item<4-> Voordeel: Sprite makkelijk te bewerken en bewegen.
  \item<5-> Nadeel: \tifonttxt{DispGraph} overschrijft het hele scherm.}
\end{itemize}

\vspace{2.5cm}

\begin{tikzpicture}[overlay,remember picture]
	\node[yshift=0.6cm] (BR) at (current page.south east){ };
	\node [anchor=south east,xshift=0.195cm] at (BR)
	{%
	\only<1->{%
		\begin{ticalc}[3.5cm]
			PROGRAM\:SPRITE\\%
			\:prgm\theta FWINDOW\\%
			\:Line(10\comma10\comma20\comma20\\%
			\:Line(20\comma20\comma30\comma10\\%
			\:Line(30\comma10\comma20\comma15\\%
			\:Line(20\comma15\comma10\comma10
		\end{ticalc}
	}%
	};
	\node [anchor=south east,xshift=-3.7cm] at (BR)
	{%
	\only<2->{%
		\begin{ticalc}[4cm]
			PROGRAM\:SPRITEPL\\%
			\:prgm\theta FWINDOW\\%
			\:\{10\comma20\comma30\comma20\comma10\>L\sub{1}\\%
			\:\{10\comma20\comma10\comma15\comma10\>L\sub{2}\\%
			\visible<3->{\:Plot1(xyLine\comma L\sub{1}\comma L\sub{2}\\%
			\:DispGraph}
		\end{ticalc}
	}%
	};
	\node [anchor=south east,xshift=-8.1cm] at (BR)
	{%
	\only<4->{%
		\begin{ticalc}[4cm]
			\qt TRANSLEER\,X\:\\%
			L\sub{1}+5\>L\sub{1}\\%
			\qt ROTEER\:\\%
			L\sub{1}cos(\theta)-L\sub{2}sin(\theta)\>L\sub{3}\\%
			L\sub{1}sin(\theta)+L\sub{2}cos(\theta)\>L\sub{2}\\%
			L\sub{3}\>L\sub{1}
		\end{ticalc}
	}%
	};
\end{tikzpicture}

\end{frame}



\begin{frame}
\frametitle{Looping Line Sprites}

\begin{itemize}
  \item<2-> We kunnen ook zelf met \tifonttxt{Line} over een list loopen!
  \item<3-> Voordeel: Cleared het scherm niet, unlike \tifonttxt{DispGraph}.
  \item<3-> Nadeel: Tekent langzamer\ldots
  \item<4-> Voor kleine sprites kun je ook met pixels tekenen.
\end{itemize}

\vspace{2.5cm}

\begin{tikzpicture}[overlay,remember picture]
	\node[yshift=0.6cm] (BR) at (current page.south east){ };
	\node [anchor=south east,xshift=0.195cm] at (BR)
	{%
	\only<1->{%
		\begin{ticalc}[3.5cm]
			PROGRAM\:SPRITE\\%
			\:prgm\theta FWINDOW\\%
			\:Line(10\comma10\comma20\comma20\\%
			\:Line(20\comma20\comma30\comma10\\%
			\:Line(30\comma10\comma20\comma15\\%
			\:Line(20\comma15\comma10\comma10
		\end{ticalc}
	}%
	};
	\node [anchor=south east,xshift=-3.75cm] at (BR)
	{%
	\only<2->{%
		\begin{ticalc}[3.8cm]
			PROGRAM\:SPRITELL\\% loop line
			\:prgm\theta FWINDOW\\%
			\:\{\one0\comma20\comma30\comma20\comma\one0\>L\sub{\one}\\%
			\:\{\one0\comma20\comma\one0\comma\one5\comma\one0\>L\sub{2}\\%
			\:For(X\comma1\comma dim(L\sub{1})-1\\%
			\:Line(L\sub{1}(X)\comma L\sub{2}(X)\comma\\%
				L\sub{1}(X+1)\comma L\sub{2}(X+1\\%
			\:End
		\end{ticalc}
	}%
	};
	\node [anchor=south east,xshift=-8cm] at (BR)
	{%
	\only<4->{%
		\begin{ticalc}[3.8cm]
			PROGRAM\:SPRITELP\\% loop point
			\:prgm\theta FWINDOW\\%
			\:\{\one0\comma20\comma30\comma20\>L\sub{\one}\\%
			\:\{\one0\comma20\comma\one0\comma\one5\>L\sub{2}\\%
			\:For(X\comma1\comma dim(L\sub{1})-1\\%
			\:Pt-On(L\sub{1}(X)\comma L\sub{2}(X
			\:End
		\end{ticalc}
	}%
	};
\end{tikzpicture}


\end{frame}



\begin{frame}
\frametitle{Compressed Lines}

\begin{itemize}
  \item<2-> In al het voorgaande werd een sprite getekend met aaneengesloten lijnen.
  \item<3-> Alternatief: compressie van een lijn in een enkel getal.
  \item<4-> Voordeel: flexibel en slechts 1 list.
  \item<4-> Nadeel: lastiger om mee te werken.
\end{itemize}

\vspace{2.5cm}

\begin{tikzpicture}[overlay,remember picture]
	\node[yshift=0.6cm] (BR) at (current page.south east){ };
	\node [anchor=south east,xshift=0.195cm] at (BR)
	{%
	\only<3->{%
		\begin{ticalc}[4cm]
			PROGRAM\:SPRITES1\\%
			\:prgm\theta FWINDOW\\%
			\:\{10102020\comma20203010
				\comma30102015\comma20151010\\%
			\:For(X\comma1\comma dim(Ans\\%
			\:Line(iPart(Ans(X)\\%
				/\E6),iPart(\E2fPart(\\%
				Ans(X)/\E6)),iPart(\E2\\%
				fPart(Ans(X)/\E4)),\E2\\%
				fPart(Ans(X)/\E2\\%
			\:End
		\end{ticalc}
	}%
	};
	\node [anchor=south east,xshift=-4.25cm] at (BR)
	{%
	\only<1->{%
		\begin{ticalc}[3.8cm]
			PROGRAM\:SPRITELL\\%
			\:prgm\theta FWINDOW\\%
			\:\{\one0\comma20\comma30\comma20\comma\one0\>L\sub{\one}\\%
			\:\{\one0\comma20\comma\one0\comma\one5\comma\one0\>L\sub{2}\\%
			\:For(X\comma1\comma dim(L\sub{1})-1\\%
			\:Line(L\sub{1}(X)\comma L\sub{2}(X)\comma\\%
				L\sub{1}(X+1)\comma L\sub{2}(X+1\\%
			\:End
		\end{ticalc}
	}%
	};
	\node [anchor=south east,xshift=-8.5cm] at (BR)
	{%
	\only<1->{%
		\begin{ticalc}[3.5cm]
			PROGRAM\:SPRITE\\%
			\:prgm\theta FWINDOW\\%
			\:Line(10\comma10\comma20\comma20\\%
			\:Line(20\comma20\comma30\comma10\\%
			\:Line(30\comma10\comma20\comma15\\%
			\:Line(20\comma15\comma10\comma10
		\end{ticalc}
	}%
	};
\end{tikzpicture}

\end{frame}


\begin{frame}
\frametitle{Compressed Lines}

\begin{itemize}
  \item<1-> Dezelfde \tifonttxt{For}-loop kan gebruikt worden voor \textit{elke} sprite!
\end{itemize}

\vspace{4cm}

\begin{tikzpicture}[overlay,remember picture]
	\node[yshift=0.6cm] (BR) at (current page.south east){ };
	\node [anchor=south east,xshift=0.195cm] at (BR)
	{%
	\only<1->{%
		\begin{ticalc}[4cm]
			PROGRAM\:SPRITES1\only<2->{a}\\%
			\:prgm\theta FWINDOW\\%
			\:\{10102020\comma20203010
				\comma30102015\comma20151010\\%
			\:For(X\comma1\comma dim(Ans\\%
			\:Line(iPart(Ans(X)\\%
				/\E6),iPart(\E2fPart(\\%
				Ans(X)/\E6)),iPart(\E2\\%
				fPart(Ans(X)/\E4)),\E2\\%
				fPart(Ans(X)/\E2\\%
			\:End
		\end{ticalc}
	}%
	};
	\node [anchor=south east,xshift=-4.25cm] at (BR)
	{%
	\only<2->{%
		\begin{ticalc}[4cm]
			PROGRAM\:SPRITES1\only<2->{b}\\%
			\:prgm\theta FWINDOW\\%
			\:\{10102020\comma20203010
				\comma30102015\comma20151010\\%
			\:prgm\theta DRAWSPT
		\end{ticalc}
	}%
	};
	\node [anchor=south east,xshift=-4.25cm,yshift=1.85cm] at (BR)
	{%
	\only<2->{%
		\begin{ticalc}[4cm]
			PROGRAM\:\theta DRAWSPT\\%
			\:For(X\comma1\comma dim(Ans\\%
			\:Line(iPart(Ans(X)\\%
				/\E6),iPart(\E2fPart(\\%
				Ans(X)/\E6)),iPart(\E2\\%
				fPart(Ans(X)/\E4)),\E2\\%
				fPart(Ans(X)/\E2\\%
			\:End
		\end{ticalc}
	}%
	};
\end{tikzpicture}

\end{frame}




\begin{frame}
\frametitle{Animation 1}

\begin{itemize}
  \item<1-> Een animatie is niets anders dan steeds een andere sprite tekenen (of andere positie/rotatie/\ldots).
  \item<2-> Wat doet deze animatie?
  \item<3-> \visible<3->{PROBLEEM: Je ziet alle vorige sprites ook\ldots}
\end{itemize}

\vspace{2.5cm}

\begin{tikzpicture}[overlay,remember picture]
	\node[yshift=0.6cm] (BR) at (current page.south east){ };
	\node [anchor=south east,xshift=0.195cm] at (BR)
	{%
	\only<2->{%
		\begin{ticalc}[4cm]
			PROGRAM\:SPRITES2\\%
			\:prgm\theta FWINDOW\\%
			\:\{10102020\comma20203010
				\comma30102015\comma20151010\>L\sub{1}\\%
			\:For(I\comma\one\comma10\\%
			\:\qt UPDATE\\%
			\:L\sub{1}+2(\E4+1)\>L\sub{1}\\%
			\:\qt DRAW\\%
			\:L\sub{1}\:prgm\theta DRAWSPT\\%
			\:End
		\end{ticalc}
	}%
	};
	\node [anchor=south east,xshift=-4.25cm] at (BR)
	{%
	\only<1->{%
		\begin{ticalc}[4cm]
			PROGRAM\:\theta DRAWSPT\\%
			\:For(X\comma1\comma dim(Ans\\%
			\:Line(iPart(Ans(X)\\%
				/\E6),iPart(\E2fPart(\\%
				Ans(X)/\E6)),iPart(\E2\\%
				fPart(Ans(X)/\E4)),\E2\\%
				fPart(Ans(X)/\E2\\%
			\:End
		\end{ticalc}
	}%
	};
\end{tikzpicture}

\end{frame}


\begin{frame}
\frametitle{Animation 2}

\hspace{-1cm}
\begin{minipage}{\textwidth}
\begin{itemize}
  \item<2-> OPLOSSING: Gum in een lijn
  \item<3-> PROBLEEM: Overschrijft de achtergrond
\end{itemize}
\end{minipage}

\vspace{2.5cm}

\begin{tikzpicture}[overlay,remember picture]
	\node[yshift=0.6cm] (BR) at (current page.south east){ };
	\node [anchor=south east,xshift=0.195cm] at (BR)
	{%
	\only<1->{%
		\begin{ticalc}[4cm]
			PROGRAM\:SPRITES2\\%
			\:prgm\theta FWINDOW\\%
			\:\{10102020\comma20203010
				\comma30102015\comma20151010\>L\sub{1}\\%
			\:For(I\comma\one\comma10\\%
			\:\qt UPDATE\\%
			\:L\sub{1}+2(\E4+1)\>L\sub{1}\\%
			\:\qt DRAW\\%
			\:L\sub{1}\:prgm\theta DRAWSPT\\%
			\:End
		\end{ticalc}
	}%
	};
	\node [anchor=south east,xshift=-4.25cm] at (BR)
	{%
	\only<1->{%
		\begin{ticalc}[4cm]
			PROGRAM\:\theta DRAWSPT\\%
			\:For(X\comma1\comma dim(Ans\\%
			\:Line(iPart(Ans(X)\\%
				/\E6),iPart(\E2fPart(\\%
				Ans(X)/\E6)),iPart(\E2\\%
				fPart(Ans(X)/\E4)),\E2\\%
				fPart(Ans(X)/\E2\select{)\comma B}\\%
			\:End
		\end{ticalc}
	}%
	};
	\node [anchor=south east,xshift=0.195cm,yshift=3.55cm] at (BR)
	{%
	\only<2->{%
		\begin{ticalc}[4cm]
			PROGRAM\:SPRITES3\\%
			\:prgm\theta FWINDOW\\%
			\:\{10102020\comma20203010
				\comma30102015\comma20151010\>L\sub{1}\\%
			\:For(I\comma\one\comma10\\%
			\:\qt UPDATE\\%
			\:\select{0\>B}\:L\sub{1}\:prgm\theta DRAWSPT\\%
			\:L\sub{1}+2(\E4+1)\>L\sub{1}\\%
			\:\qt DRAW\\%
			\:\select{1\>B}\:L\sub{1}\:prgm\theta DRAWSPT\\%
			\:End
		\end{ticalc}
	}%
	};
\end{tikzpicture}

\end{frame}






\begin{frame}
\frametitle{Background image: \tifonttxt{StorePic}}

\visible<2->{\ticalcfig{\ticalcfigCircleSecond\ticalcfigCircle{\ticalcfigCircleColThree}{0.615}}} % draw

\begin{itemize}
  \item<1-> \lenitem{Indien je iets stilstaands hebt, zoals een title screen of een background, dan wil je vaak dat het niet overschreven kan worden.}
  \item<2-> \lenitem{Met \tifonttxt{StorePic} kun je het opslaan in een plaatje (alleen mogelijk met het \textit{hele} scherm).}
  \item<3-> \lenitem{Met \tifonttxt{RecallPic} kun je dat plaatje weer tekenen (bovenop wat er al op je scherm staat).}
  \item<4-> \lenitem{Je kunt maximaal 10 \tifonttxt{Pic}s opslaan.}
\end{itemize}

\vspace{2.5cm}

\begin{tikzpicture}[overlay,remember picture]
	\node[yshift=0.6cm] (BR) at (current page.south east){ };
	\node [anchor=south east,xshift=0.195cm] at (BR)
	{%
	\only<2->{%
		\begin{ticalc}
			DRAW\,POINTS\,\select{STO}\\%
			\only<3->{\one\:}\only<2>{\selectitem{\one\:}}StorePic\\%
			\only<2>{2\:}\only<3->{\selectitem{2\:}}RecallPic\\%
			3\:StoreGDB(\\%
			4\:RecallGDB
		\end{ticalc}
	}%
	};
	\node [anchor=south east,xshift=-3.5cm] at (BR)
	{%
	\only<2->{%
		\begin{ticalc}[3.5cm]
			PROGRAM\:SPRITEBG\\%
			\:prgm\theta FWINDOW\\%
			\:Circle(10\comma10\comma30\\%
			\:StorePic\,\only<4>{\select{4}}\only<-3>{4}\\%
			\visible<3->{\:\ldots do\,stuff\ldots\\%
			\:RecallPic\,\only<4>{\select{4}}\only<-3>{4}}
		\end{ticalc}
	}%
	};
\end{tikzpicture}

\end{frame}




\begin{frame}
\frametitle{Caching}

\hspace{-1cm}
\begin{minipage}{\textwidth}
\begin{itemize}
  \item<1-> `Caching' is het opslaan van een plaatje voor later gebruik.
  \item<2-> Stel dat je sprite over het scherm beweegt:
  \begin{itemize}
    \item<3-> Indien je blijft tekenen zie je een heel spoor waar je sprite geweest is.
    \item<4-> Zijn vorige positie moeten we clearen!
  \end{itemize}
  \item<5-> Dit kan met `caching':\\
  			\tifonttxt{StorePic} en \tifonttxt{RecallPic}.
\end{itemize}
\end{minipage}


\vspace{2.5cm}

\begin{tikzpicture}[overlay,remember picture]
	\node[yshift=0.6cm] (BR) at (current page.south east){ };
	\node [anchor=south east,xshift=0.195cm] at (BR)
	{%
	\only<1->{%
		\begin{ticalc}[4cm]
			PROGRAM\:SPRITES4\\%
			\:prgm\theta FWINDOW\\%
			\:Circle(10\comma10\comma30\\%
			\:\only<-4>{StorePic\,4}\only<5->{\select{StorePic\,4}}\\%
			\:\{10102020\comma20203010
				\comma30102015\comma20151010\>L\sub{1}\\%
			\:For(I\comma\one\comma10\\%
			\:\qt UPDATE\\%
			\:L\sub{1}+2(\E4+1)\>L\sub{1}\\%
			\:\qt DRAW\\%
			\visible<5->{\:\select{ClrDraw\:RecallPic\,4}\\}%
			\:L\sub{1}\:prgm\theta DRAWSPT\\%
			\:End
		\end{ticalc}
	}%
	};
\end{tikzpicture}

\end{frame}







\begin{frame}
\frametitle{Animation 3: Clear small sprites}

\begin{itemize}
  \item<1-> Small sprites kunnen gecleared worden door spaties bovenop de sprite te tekenen.
  \item<2-> PROBLEEM: Werkt alleen makkelijk\\
  			voor kleine sprites (max. 5 pixels hoog):\\
  			Waardeloos voor dit voorbeeld!
  \item<3-> PROBLEEM: Achtergrond verdwijnt.
\end{itemize}

\vspace{2.5cm}



\begin{tikzpicture}[overlay,remember picture]
	\node[yshift=0.6cm] (BR) at (current page.south east){ };
	\node [anchor=south east,xshift=0.195cm] at (BR)
	{%
	\only<1->{%
		\begin{ticalc}[4cm]
			PROGRAM\:SPRITES5\\%
			\:prgm\theta FWINDOW\\%
			\:Circle(10\comma10\comma30\\%
			\:\{10102020\comma20203010
				\comma30102015\comma20151010\>L\sub{1}\\%
			\:For(I\comma\one\comma10\\%
			\:\qt UPDATE\\%
			\:L\sub{1}+2(\E4+1)\>L\sub{1}\\%
			\:\qt DRAW\\%
			\:Text(62-10\comma10\comma\\%
				\qt\tiSpace\tiSpace\tiSpace\tiSpace\tiSpace\tiSpace\tiSpace\tiSpace\tiSpace\\%
			\:L\sub{1}\:prgm\theta DRAWSPT\\%
			\:End
		\end{ticalc}
	}%
	};
\end{tikzpicture}



\end{frame}




\begin{frame}
\frametitle{Exercise}


Maak zelf een simpele sprite die langzaam van links naar rechts beweegt.


\end{frame}






\section{Dynamic input}

\subsection{getKey}

% getKey command
\begin{frame}
\frametitle{Check for key presses}

\visible<1>{\ticalcfig{\ticalcfigCircle{\ticalcfigCircleColThree}{0.615}}} % prgm

\begin{itemize}
  \item<1-> \lenitem[0.75\linewidth]{Met \tifonttxt{getKey} kun je de \emph{laatste} knop die ingedrukt is opvragen.}
  \item<2-> \lenitem[0.75\linewidth]{Je krijgt een getal, welk naar een knop verwijst (zie figuur).}
\end{itemize}



\vspace{2.5cm}



\begin{tikzpicture}[overlay,remember picture]
	\node[yshift=0.6cm] (BR) at (current page.south east){ };
	\node [anchor=south east,xshift=0.195cm] at (BR)
	{%
	\only<2->{%
		\begin{tikzonimage}[width=3.2cm]{./TI84_getkey.jpg}
			\only<6>{\ticalcfigCircle{0.822}{0.113}}
		\end{tikzonimage}
	}
	};
	\node [anchor=south east,xshift=-3.5cm] at (BR)
	{%
	\only<1->{%
		\begin{ticalc}
			CTL\,\select{I/O}\,EXEC \\
			1\+\:Input \\
			2\:Prompt \\
			3\:Disp \\
			4\:DispGraph \\
			5\:DispTable \\
			6\:Output( \\
			\selectitem{7\arrowdown} getKey
		\end{ticalc}
	}%
	};
	\node [anchor=south east,xshift=-7.195cm] at (BR)
	{%
	\only<3->{%
		\begin{ticalc}
			getKey\\%
			\hfill0\\%
			\visible<4->{getKey\\%
			\hfill0\\}%
			\visible<5->{getKey\\%
			\hfill0\\}%
			\visible<6->{getKey\\%
			\hfill105}
		\end{ticalc}
	}%
	};
\end{tikzpicture}

\end{frame}









\subsection{wait for input}

% use getKey, freeze prgm until button press


\begin{frame}
\frametitle{Wait for keypress}


\begin{itemize}
  \item<1-> \lenitem[0.75\linewidth]{\tifonttxt{prgmGETKEY1} wacht tot een knop ingedrukt wordt.}
  \item<2-> \lenitem[0.75\linewidth]{\tifonttxt{prgmGETKEY2} blijft loopen totdat er op \tiENTER\,gedrukt wordt.}
\end{itemize}



\vspace{3cm}

\begin{tikzpicture}[overlay,remember picture]
	\node[yshift=0.6cm] (BR) at (current page.south east){ };
	\node [anchor=south east,xshift=0.195cm] at (BR)
	{%
	\only<1->{%
		\begin{tikzonimage}[width=3.2cm]{./TI84_getkey.jpg}
			\only<2>{\ticalcfigCircle{0.822}{0.113}}
		\end{tikzonimage}
	}
	};
	\node [anchor=south east,xshift=-3.5cm] at (BR)
	{%
	\only<1->{%
		\begin{ticalc}
			PROGRAM\:GETKEY1\\%
			\:Repeat\,Ans\\%
			\:getKey\>X\\%
			\:Disp\,X\\%
			\:End
		\end{ticalc}
	}%
	};
	\node [anchor=south east,xshift=-7.195cm] at (BR)
	{%
	\only<2->{%
		\begin{ticalc}
			PROGRAM\:GETKEY2\\%
			\:0\>K\\%
			\:While\,K\!105\\%
			\:getKey\>K\\%
			\:If\,K\!0
			\:Disp\,K\\%
			\:End
		\end{ticalc}
	}%
	};
\end{tikzpicture}

\end{frame}




% \subsection{event --> action}


\begin{frame}
\frametitle{Hoe reageer je op specifieke input?}

\vspace{-1cm}

\begin{itemize}
  \item \lenitem[0.7\linewidth]{Hoe kunnen we onze kennis van \tifonttxt{getKey} gebruiken om te reageren op specifieke knoppen?}
  \begin{itemize}
    \item Bijv. reageer op pijltoetsen voor game/menu.
  \end{itemize}
\end{itemize}

\vspace{3cm}

\begin{tikzpicture}[overlay,remember picture]
	\node[yshift=0.6cm] (BR) at (current page.south east){ };
	\node [anchor=south east,xshift=0.195cm] at (BR)
	{%
	\only<3->{%
		\begin{tikzonimage}[width=3.2cm]{./TI84_getkey.jpg}
			\ticalcfigCircle{0.75}{0.568} % Up
			\ticalcfigCircle{0.628}{0.53} % Left
% 			\ticalcfigCircle{0.822}{0.113} % Enter
		\end{tikzonimage}
	}
	};
	\node [anchor=south east,xshift=-3.5cm] at (BR)
	{%
	\only<2->{%
		\begin{ticalc}
			PROGRAM\:AROWKEY1\\%
			\:0\>K\:Repeat\,K\\%
			\:getKey\>X\\%
			\:If\,K=25\:Then\\%
			\:Disp\,\qt MOVE\,UP\\%
			\:End\\%
			\:If\,K=24\:Then\\%
			\:Disp\,\qt MOVE\,<-\\%
			\:End\\%
			\:End
		\end{ticalc}
	}%
	};
	\node [anchor=south east,xshift=-7.195cm] at (BR)
	{%
	\only<4->{%
		\begin{ticalc}
			PROGRAM\:AROWKEY2\\%
			\:Lbl\,0
			\:0\>K\\%
			\:Repeat\,K\\%
			\:getKey\>K\\%
			\:End\\%
			\:If\,K=25\:Then\\%
			\:Disp\,\qt MOVE\,UP\\%
			\:Else\\%
			\:If\,K=24\:Then\\%
			\:Disp\,\qt MOVE\,<-\\%
			\:End\:End\\%
			\:Goto\,0
		\end{ticalc}
	}%
	};
\end{tikzpicture}

\end{frame}





\begin{frame}
\frametitle{Event \tifonttxt{\>} Action programming}

\vspace{-1cm}

\begin{itemize}
  \item<2-> \lenitem[0.7\linewidth]{Een betere manier is \tifonttxt{PROGRAM\:AROWKEY3}.}
  \item<3-> Merk op hoe raar de \tifonttxt{End}'s staan
  \begin{itemize}
    \item<4-> \tifonttxt{Goto} springt in het programma, dus klopt wel!
  \end{itemize}
\end{itemize}

\vspace{3cm}

\begin{tikzpicture}[overlay,remember picture]
	\node[yshift=0.6cm] (BR) at (current page.south east){ };
	\node [anchor=south east,xshift=0.195cm] at (BR)
	{%
	\only<1->{%
		\begin{tikzonimage}[width=3.2cm]{./TI84_getkey.jpg}
			\ticalcfigCircle{0.75}{0.568} % Up
			\ticalcfigCircle{0.628}{0.53} % Left
 			\only<2->{\ticalcfigCircle{0.822}{0.113}} % Enter
		\end{tikzonimage}
	}
	};
	\node [anchor=south east,xshift=-3.5cm] at (BR)
	{%
	\only<1->{%
		\begin{ticalc}
			PROGRAM\:AROWKEY1\\%
			\:0\>K\:Repeat\,K\\%
			\:getKey\>X\\%
			\:If\,K=25\:Then\\%
			\:Disp\,\qt MOVE\,UP\\%
			\:End\\%
			\:If\,K=24\:Then\\%
			\:Disp\,\qt MOVE\,<-\\%
			\:End\\%
			\:End
		\end{ticalc}
	}%
	};
	\node [anchor=south east,xshift=-7.195cm] at (BR)
	{%
	\only<2->{%
		\begin{ticalc}
			PROGRAM\:AROWKEY3\\%
			\:If\,1=1\:Then\\%
			\:\only<-3>{Lbl\,0}\only<4->{\select{Lbl\,0}}
			\:\only<-2>{End}\only<3->{\select{End}}\\%
			\:getKey\>K\\%
			\:If\,K=0\:Then\\%
			\:Goto\,0
			\:End\\%
			\:\only<-3>{If\,K=25\:Then}\only<4->{\select{If\,K=25}\:\select{Then}}\\%
			\:Disp\,\qt MOVE\,UP\\%
			\:\only<-3>{Goto\,0}\only<4->{\select{Goto\,0}}
			\:End\\%
			\:If\,K=24\:Then\\%
			\:Disp\,\qt MOVE\,<-\\%
			\:Goto\,0
			\:End\\%
			\:If\,K\!105\:Then\\%
			\:Goto\,0
		\end{ticalc}
	}%
	};
\end{tikzpicture}

\end{frame}


\subsection{Memory Leaks}


\begin{frame}
\frametitle{Intermezzo: Memory Leaks}

\vspace{-1cm}
\hspace{-1cm}
\begin{minipage}{\textwidth}
\begin{itemize}
  \item<2-> \lenitem[\linewidth]{Als je in een block statement (zoals \tifonttxt{If} of \tifonttxt{While}) gaat, onthoud de rekenmachine hoe diep je bent: dit heet de `stack'.}
  \item<3-> \lenitem[0.75\linewidth]{Wanneer je met \tifonttxt{Goto} naar buiten springt, ga je steeds dieper en dieper.}
  \item<4-> \lenitem[0.75\linewidth]{Dit kost memory\ldots\visible<5->{Vertraagt je \tifonttxt{prgm}\ldots}\\\visible<6->{En uiteindelijk crashed je \tifonttxt{prgm}\ldots}}
  \item<7-> \lenitem[0.8\linewidth]{De memory leak verdwijnt pas zodra \tifonttxt{prgm} eindigt.}
  \item<8-> \lenitem[0.75\linewidth]{http://tibasicdev.wikidot.com/memory-leaks}
\end{itemize}
\end{minipage}


\begin{tikzpicture}[overlay,remember picture]
	\node[yshift=0.6cm] (BR) at (current page.south east){ };
	\node [anchor=south east,xshift=0.195cm] at (BR)
	{%
% 	\only<1->{%
% 		\begin{tikzonimage}[width=3.2cm]{./TI84_getkey.jpg}
% 			\ticalcfigCircle{0.75}{0.568} % Up
% 			\ticalcfigCircle{0.628}{0.53} % Left
%  			\only<2->{\ticalcfigCircle{0.822}{0.113}} % Enter
% 		\end{tikzonimage}
% 	}
	\only<1->{%
		\begin{ticalc}
			PROGRAM\:AROWKEY3\\%
			\:If\,1=1\:Then\\%
			\:Lbl\,0
			\:\select{End}\\%
			\:getKey\>K\\%
			\:If\,K=0\:Then\\%
			\:Goto\,0
			\:End\\%
			\:If\,K=25\:Then\\%
			\:Disp\,\qt MOVE\,UP\\%
			\:Goto\,0
			\:End\\%
			\:If\,K=24\:Then\\%
			\:Disp\,\qt MOVE\,<-\\%
			\:Goto\,0
			\:End\\%
			\:If\,K\!105\:Then\\%
			\:Goto\,0
		\end{ticalc}
	}%
	};
	\node [anchor=south east,xshift=-3.5cm] at (BR)
	{%
	\only<3->{%
		\begin{ticalc}
			\:PROGRAM\:MEMLEAK\\%
			\:Lbl\,A\\%
			\:While\,1\\%
			\:Goto\,A\\%
			\:End
		\end{ticalc}
	}%
	};
	\node [anchor=south east,xshift=-7.195cm] at (BR)
	{%
	\only<4->{%
		\begin{ticalc}
			prgmMEMLEAK\\%
			\visible<6->{ERR\:MEMORY}
		\end{ticalc}
	}%
	};
\end{tikzpicture}



\end{frame}






\begin{frame}
\frametitle{Exercise}

\lenitem[0.7\linewidth]{Maak een ``custom menu'': een menu getekent met \tiDRAW-functies bestuurt via \tifonttxt{getKey}.}

\lenitem[0.7\linewidth]{\flushleft{Maak het zo fancy als je wilt (cursor met animatie :)?), maar \underline{begin simpel}.}}


\begin{tikzpicture}[overlay,remember picture]
	\node[yshift=0.6cm] (BR) at (current page.south east){ };
	\node [anchor=south east,xshift=0.195cm] at (BR)
	{%
	\only<1->{%
		\begin{tikzonimage}[width=3.2cm]{./TI84_getkey.jpg}
			\ticalcfigCircle{0.75}{0.568} % Up
			\ticalcfigCircle{0.75}{0.488} % Down
 			\ticalcfigCircle{0.822}{0.113} % Enter
		\end{tikzonimage}
	}
	};
	\node [anchor=south east,xshift=-3.5cm] at (BR)
	{%
	\only<1->{%
		\begin{ticalc}
			\tiny{WHAT\,PIE?}\\%
			\tiny{\visible<1>{-}\,APPLE}\\%
			\tiny{\visible<2>{-}\,BANANA\,CREAM}\\%
			\tiny{\visible<3>{-}\,CHERRY}
		\end{ticalc}
	}%
	};
% 	\node [anchor=south east,xshift=-7.195cm] at (BR)
% 	{%
% 	\only<4->{%
% 		\begin{ticalc}
% 			prgmMEMLEAK\\%
% 			\visible<6->{ERR\:MEMORY}
% 		\end{ticalc}
% 	}%
% 	};
\end{tikzpicture}



\end{frame}



\begin{frame}
\frametitle{Answer}

\hspace{-1cm}
\begin{minipage}{0.44\textwidth}
	\begin{ticalc}[6.7cm]
		PROGRAM\:CUSTMENU\\%
		\:ClrDraw\:0\>I\\%
		\:Text(10\comma20\comma\qt APPLE\\%
		\:Text(20\comma20\comma\qt BANANA\,CREAM\\%
		\:Text(30\comma20\comma\qt CHERRY\\%
		\:Text(40\comma20\comma\qt DERBY\\%
		\:Text(50\comma20\comma\qt EMPANADA\\%
		\:0\>K\:Repeat\,K=105\\%
		\:getKey\>K\\%
		\:I\>J\\%
		\:If\,K=25\:I-1\>I\\%
		\:If\,K=34\:I+1\>I\\%
		\:min(4\comma max(0\comma I\>I\\%
		\:\\%
		\:Text(10+10I\comma10\comma\qt-\\%
		\:If\,I\!J\:Text(10+10J\comma10\comma\qt\tiSpace\tiSpace\tiSpace\\%
		\:End\\%
		\:Disp\,\qt YOU\,CHOSE\qt\comma I+1
	\end{ticalc}
\end{minipage}
\fcolorbox{black}{white}{%
\begin{minipage}{0.52\textwidth}
	Met simpele animatie (8 frames):
	
	\begin{ticalc}[5.3cm]
		\:\ldots\\%
		\:ClrDraw\:0\>I\:0\>A\\%
		\:\ldots\\%
		\:8fPart((A+1)/8\>A\\%
		\:Text(10+10I\comma10\comma\qt-\\%
		\:Text(10+10I\comma6+A\comma\qt\tiSpace\tiSpace\\%
		\:\ldots
	\end{ticalc}
\end{minipage}
}



\end{frame}


% 	\node [anchor=south east,xshift=-7.195cm] at (BR)
% 	{%
% 	\only<4->{%
% 		\begin{ticalc}
% 			PROGRAM\:AROWKEY2\\%
% 			\:Lbl\,0
% 			\:0\>X\\%
% 			\:Repeat\,X\\%
% 			\:getKey\>X\\%
% 			\:End\\%
% 			\:If\,X=25\:Then\\%
% 			\:Disp\,\qt MOVE UP\qt\\%
% 			\:Else\\%
% 			\:If\,X=105\:Then\\%
% 			\:Disp\,\qt END\qt\\%
% 			\:End
% 			\:End\\%
% 			\:Goto\,0
% 		\end{ticalc}
% 	}%
% 	};

% use getKey, continue prgm and process input if buton was pressed
% pseudocode!



% 1: in een real-time game wil je niet wachten op button press
%    je wilt game afspelen en iets speciaals doen (action) wanneer er een button ingedrukt wordt (event)

% 2: code




% Intermediate exercise: make a custom menu!



\section{Games!}

\subsection{Game Loop}

% Game Loop

\begin{frame}
\frametitle{Game Loop}

\begin{itemize}
  \item<1-> In tegenstelling tot veel van het voorgaande, wil je dat een game blijft spelen tot de gebruiker het afsluit.
  \item<2-> Dit kun je bereiken met een `infinite loop'.
  \item<3-> Daarin moet een aantal dingen gebeuren:
  \begin{enumerate}
    \item<4-> \visible<4->{Input (handle key presses)}
    \item<5-> \visible<5->{Update (rekenen)}
    \item<6-> \visible<6->{Draw (tekenen)} 
  \end{enumerate}
  \item<7-> Precieze vorm is afhankelijk van gametype:\\%
  			turn-based, real-time
\end{itemize}

\vspace{2.5cm}

\begin{tikzpicture}[overlay,remember picture]
	\node[yshift=0.6cm] (BR) at (current page.south east){ };
	\node [anchor=south east,xshift=0.195cm] at (BR)
	{%
	\only<3->{%
		\begin{ticalc}
			PROGRAM\:GAMELOOP\\%
			\:\qt INIT\,GAME\\%
			\:While\,1\\%
			\:\visible<4->{\qt INPUT\\%
			\:getKey\>K\\%
			\:If\,K=\ldots\\%
			\:\ldots}\\%
			\:\visible<5->{\qt UPDATE\\%
			\:L\sub{1}+UT(\E6+\E2)+\\%
				VT(\E4+1)\>L\sub{1}}\\%
			\:\visible<6->{\qt DRAW\\%
			\:L\sub{1}\:prgm\theta DRAWSPT}\\%
			\:End
		\end{ticalc}
	}%
	};
	\node [anchor=south east,xshift=-3.5cm] at (BR)
	{%
	\only<2->{%
		\begin{ticalc}
			PROGRAM\:INFILOOP\\%
			\:While\,1\\%
			\:do stuff\\%
			\:End
		\end{ticalc}
	}%
	};
\end{tikzpicture}

\end{frame}


\begin{frame}
\frametitle{Real-Time Games}

\hspace{-1.2cm}
\begin{minipage}{\textwidth}
\begin{itemize}
  \item \lenitem[0.45\linewidth]{In een real-time game gaat de game door, of je nou op een knop drukt of niet.}
\end{itemize}
\end{minipage}

\vspace{2.5cm}

\begin{tikzpicture}[overlay,remember picture]
	\node[yshift=0.6cm] (BR) at (current page.south east){ };
	\node [anchor=south east,xshift=0.195cm] at (BR)
	{%
	\only<1->{%
		\begin{ticalc}
			\:If\,K=105\:Then\\%
			\:Goto\,X\\%
			\:\qt UPDATE\\%
			\:Lbl\,UD
			\:End\\%
			\:L\sub{1}+U(\E6+\E2)+\\%
				V(\E4+1)\>L\sub{1}\\%
			\:\qt DRAW\\%
			\:L\sub{1}\:prgm\theta DRAWSPT\\%
			\:End\\%
			\:Lbl\,X\\%
			\:\only<1>{Clean stuff up here, reset graph axis etc.}\only<2->{\select{Clean stuff up}\\\select{here, reset graph}\\\select{axis etc.}}
		\end{ticalc}
	}%
	};
	\node [anchor=south east,xshift=-3.5cm] at (BR)
	{%
	\only<1->{%
		\begin{ticalc}
			PROGRAM\:REALTIME\\%
			\:\qt INIT\,GAME\\%
			\:While\,1\\%
			\:\qt INPUT\\%
			\:getKey\>K\\%
			\:\select{If\,K=0}\:Then\\%
			\:Goto\,UD
			\:End\\%
			\:If\,K=25\:Then\\%
			\:V+1\>V\\%
			\:Goto\,UD
			\:End\\%
			\:If\,K=34\:Then\\%
			\:V-1\>V\\%
			\:Goto\,UD
			\:End\\%
			\:If\,K=24\,or\,K=26\:Then\\%
			\:U-25+K\>U\\%
			\:Goto\,UD
			\:End\\%
		\end{ticalc}
	}%
	};
\end{tikzpicture}

\end{frame}


\begin{frame}
\frametitle{Turn-Based Games}

\hspace{-1.2cm}
\begin{minipage}{\textwidth}
\begin{itemize}
  \item \lenitem[0.45\linewidth]{In een turn-based game gebeurt er alleen iets nieuws wanneer er op de knop wordt gedrukt.}
\end{itemize}
\end{minipage}

\vspace{2.5cm}

\begin{tikzpicture}[overlay,remember picture]
	\node[yshift=0.6cm] (BR) at (current page.south east){ };
	\node [anchor=south east,xshift=0.195cm] at (BR)
	{%
	\only<1->{%
		\begin{ticalc}
			\:If\,K=105\:Then\\%
			\:Goto\,X\\%
			\:\qt UPDATE\\%
			\:Lbl\,UD
			\:End\\%
			\:L\sub{1}+U(\E6+\E2)+\\%
				V(\E4+1)\>L\sub{1}\\%
			\:0\>U\:0\>V\\%
			\:\qt DRAW\\%
			\:L\sub{1}\:prgm\theta DRAWSPT\\%
			\:End\\%
			\:Lbl\,X\\%
			\:\only<1>{Clean stuff up here, reset graph axis etc.}\only<2->{\select{Clean stuff up}\\\select{here, reset graph}\\\select{axis etc.}}
		\end{ticalc}
	}%
	};
	\node [anchor=south east,xshift=-3.5cm] at (BR)
	{%
	\only<1->{%
		\begin{ticalc}
			PROGRAM\:TURNGAME\\%
			\:\qt INIT\,GAME\\%
			\:While\,1\\%
			\:\qt INPUT\\%
			\:\select{Repeat\,Ans}\\%
			\:getKey\>K\\%
			\:\select{end}\\%
			\:If\,K=25\:Then\\%
			\:1\>V\\%
			\:Goto\,UD
			\:End\\%
			\:If\,K=34\:Then\\%
			\:\min\one\>V\\%
			\:Goto\,UD
			\:End\\%
			\:If\,K=24\,or\,K=26\:Then\\%
			\:K-25\>U\\%
			\:Goto\,UD
			\:End\\%
		\end{ticalc}
	}%
	};
\end{tikzpicture}

\end{frame}





\begin{frame}
\frametitle{Turn-Based Games with Animations}

\hspace{-1.2cm}
\begin{minipage}{\textwidth}
\begin{itemize}
  \item \lenitem[0.45\linewidth]{Indien animaties door moeten gaan, moet je iets slimmer zijn\ldots}
\end{itemize}
\end{minipage}

\vspace{2.5cm}

\begin{tikzpicture}[overlay,remember picture]
	\node[yshift=0.6cm] (BR) at (current page.south east){ };
	\node [anchor=south east,xshift=0.195cm] at (BR)
	{%
	\only<1->{%
		\begin{ticalc}
			\:If\,K=105\:Then\\%
			\:Goto\,X\\%
			\:Lbl\,UD
			\:End\\%
			\:\qt PROCESS\,TURN\\%
			\:L\sub{1}+U(\E6+\E2)+\\%
				V(\E4+1)\>L\sub{1}\\%
			\:0\>U\:0\>V\\%
			\:\select{End}\\%
			\:\qt ANIMATE\\%
			\:Update sprite animation here\\%
			\:\qt DRAW\\%
			\:L\sub{1}\:prgm\theta DRAWSPT\\%
			\:End\\%
			\:Lbl\,X\\%
			\:Clean stuff up here, reset graph axis etc.
		\end{ticalc}
	}%
	};
	\node [anchor=south east,xshift=-3.5cm] at (BR)
	{%
	\only<1->{%
		\begin{ticalc}
			PROGRAM\:TURNGAM2\\%
			\:\qt INIT\,GAME\\%
			\:While\,1\\%
			\:\qt INPUT\\%
			\:getKey\>K\\%
			\:\select{If\,K\!0\:Then}\\%		
			\:If\,K=25\:Then\\%
			\:1\>V\\%
			\:Goto\,UD
			\:End\\%
			\:If\,K=34\:Then\\%
			\:\min\one\>V\\%
			\:Goto\,UD
			\:End\\%
			\:If\,K=24\,or\,K=26\:Then\\%
			\:K-25\>U\\%
			\:Goto\,UD
			\:End\\%
		\end{ticalc}
	}%
	};
\end{tikzpicture}

\end{frame}



\begin{frame}
\frametitle{Turn-Based Games: Wait for Animation}

\hspace{-1.2cm}
\begin{minipage}{\textwidth}
\begin{itemize}
  \item<1-> \lenitem[0.45\linewidth]{Het kan logisch zijn om te wachten tot een animatie klaar is in een turn-based game, voordat de volgende zet gedaan mag worden.}
  \item<2-> \lenitem[0.45\linewidth]{Dit voorbeeld negeert de input (\tifonttxt{K}) totdat de animatie (\tifonttxt{A}) klaar is.}
\end{itemize}
\end{minipage}

\vspace{2.5cm}

\begin{tikzpicture}[overlay,remember picture]
	\node[yshift=0.6cm] (BR) at (current page.south east){ };
	\node [anchor=south east,xshift=0.195cm] at (BR)
	{%
	\only<1->{%
		\begin{ticalc}
			\:If\,K=105\:Then\\%
			\:Goto\,X\\%
			\:Lbl\,UD
			\:End\\%
			\:\qt PROCESS\,TURN\\%
			\:L\sub{1}+U(\E6+\E2)+\\%
				V(\E4+1)\>L\sub{1}\\%
			\:0\>U\:0\>V\\%
			\:\select{End}\\%
			\:\qt ANIMATE\\%
			\:Update sprite animation here\\%
			\:\visible<2->{\select{If\,animation}\\\select{finished\:0\>A}}\\%
			\:\qt DRAW\\%
			\:L\sub{1}\:prgm\theta DRAWSPT\\%
			\:End\\%
			\:Lbl\,X\\%
			\:Clean stuff up here, reset graph axis etc.
		\end{ticalc}
	}%
	};
	\node [anchor=south east,xshift=-3.5cm] at (BR)
	{%
	\only<1->{%
		\begin{ticalc}
			PROGRAM\:TURNGAM3\\%
			\:\qt INIT\,GAME\\%
			\:While\,1\\%
			\:\qt INPUT\\%
			\:getKey\>K\\%
			\:\visible<2->{\select{If\,A=1\:K=0}}\\%
			\:\select{If\,K\!0\:Then}\\%
			\:\visible<2->{\select{1\>A}}\\%
			\:If\,K=25\:Then\\%
			\:1\>V\\%
			\:Goto\,UD
			\:End\\%
			\:If\,K=34\:Then\\%
			\:\min\one\>V\\%
			\:Goto\,UD
			\:End\\%
			\:If\,K=24\,or\,K=26\:Then\\%
			\:K-25\>U\\%
			\:Goto\,UD
			\:End\\%
		\end{ticalc}
	}%
	};
\end{tikzpicture}

\end{frame}




\subsection{Towards multiplayer\ldots}

% Kort hoe ze verder kunnen om games multiplayer te maken:
% - Belangrijkste overwegingen
% - Limitaties


\begin{frame}
\frametitle{Wat is een multiplayer game?}

\begin{itemize}
  \item<1-> Meerdere spelers spelen hetzelfde spel:
  \begin{itemize}
    \item<2-> Tegen elkaar (PvP)
    \item<2-> Samen met elkaar (coop)
  \end{itemize}
  \item<3-> Meerdere manieren:
  \begin{itemize}
    \item<4-> One Device
    \item<4-> Two Devices
  \end{itemize}
  \item<5-> Ik geef slechtst een korte overview. Details staan hier:
  \begin{itemize}
    \item<5-> http://tibasicdev.wikidot.com/multiplayer
  \end{itemize}  
\end{itemize}

\end{frame}





\begin{frame}
\frametitle{Multiplayer: One Device}

\begin{itemize}
  \item<1-> Dit is vrijwel hetzelfde als een single player game, maar nu:
  \begin{itemize}
    \item<2-> Andere knoppen besturen andere sprites: meer \inlineticalc{If\,K=\ldots} statements.
  \end{itemize}
  \item<3-> BELANGRIJK: Slechts 1 knop kan tegelijk ingedrukt worden.
  \begin{itemize}
    \item<4-> PROBLEEM: Indien je \tifonttxt{prgm} langzaam is, dan komen niet beide spelers aan de beurt; alleen degene die het snelste drukt.
    \item<5-> PAS OP: Pijltoetsen kunnen ingedrukt worden gehouden. Gebruik die knoppen dus NIET.
    \item<6-> Geen probleem voor turn-based games!\\
    		  Turn-based is een aanrader voor multiplayer!
  \end{itemize}
\end{itemize}

\end{frame}






\begin{frame}
\frametitle{Multiplayer: Two Devices}

\begin{itemize}
  \item<1-> Het principe is simpel:
  \begin{itemize}
    \item<2-> Connect een linkabel.
    \item<3-> Beide spelers starten hetzelfde \tifonttxt{prgm}.
    \item<4-> Beide spelers kunnen hun eigen knoppen gebruiken.
  \end{itemize}
  \item<5-> Linken van variabelen gaat via de \tifonttxt{GetCalc(varname)} functie.
  \item<6-> PROBLEEM: Linken kan alleen wanneer \'e\'en rekenmachine niets doet, bijvoorbeeld door \tifonttxt{Pause}.
  \item<7-> Daarom \underline{alleen mogelijk voor turn-based games}.
  \begin{itemize}
    \item<8-> Maar dat kan net zo goed op \'e\'en device\ldots
  \end{itemize}  
\end{itemize}

\end{frame}













\section{Exercises}
\subsection{Exercises}
% 
% 
\begin{frame}
\frametitle{Exercises\ldots?}

Nu heb je geen exercises meer nodig, toch?

\begin{enumerate}
  \item Bedenk een idee voor een spel.
  \item Versimpel je idee A LOT.
  \begin{itemize}
    \item Je idee was geweldig!, maar als je het niet eerst versimpelt gaat het je nooit lukken.
    \item One step at a time! Begin \underline{altijd} simpel.
  \end{itemize}
  \item Maak pseudocode.
  \item Maak de game loop.
  \item Maak een simpele sprite om mee te beginnen.
  \item Implementeer een \underline{minimaal-werkende} game.
  \item Breid de game nu langzaam uit met meer features.
\end{enumerate}

\end{frame}




\subsection{Further Reading}

\begin{frame}
\frametitle{Further Reading}

\begin{itemize}
  \item Op deze website staat veel informatie over alles wat je maar wilt weten:
  \begin{itemize}
    \item http://tibasicdev.wikidot.com/home
    \item http://tibasicdev.wikidot.com/graphics
    \item http://tibasicdev.wikidot.com/maps
    \item http://tibasicdev.wikidot.com/cryptography
    \item http://tibasicdev.wikidot.com/multiplayer
  \end{itemize}
  \item Of stel je vraag op het offici\"ele forum:
  \begin{itemize}
    \item http://tibasicdev.wikidot.com/forum:home
  \end{itemize}
\end{itemize}

\end{frame}


% TOPIC:
% GetCalc/Get/Send (?)
% 
% Graphics:
%  - Show that drawDice-prgm using ``Line'' and ``Circle'' is slow
%  - Hard-coded sprites --> Compression with list (http://tibasicdev.wikidot.com/compression)
%  - Friendly Screen
%  - Plot Sprites
%  - Text Sprites
%
% Multiplayer games:
%  Single device: --> Use about 3 slides for this
%   Easy to make, but disadvantages: cannot press two buttons simultaneously etc.
%  Two device: --> Much more interesting, spend a lecture on it!
%   Calculator unique identity
%   One Screen: one boss, one follower
%    .
%   Two Screen: democracy
%    .
% 
% 
% IDEAS:
%
% Memory leaks
%
% Risk dice wars (two calculators)
% Game of Life
% Interactive Poker
%






\end{document}

%% END %%