\section{Graphics}

\subsection{Friendly window}

\begin{frame}
\frametitle{Friendly window}

\visible<5>{\ticalcfig{\ticalcfigCircle{\ticalcfigCircleColThree}{0.944}}} % zoom
\visible<6->{\ticalcfig{\ticalcfigCircle{\ticalcfigCircleColFour}{0.62}}} % vars

\begin{itemize}
  \item<1-> Om iets te tekenen, hebben we coordinaten nodig.
  \item<2-> Een mooi coordinatenstelsel is handig!
  \item<3-> Kies zodat elke pixel ``1'' waard is.
  \begin{itemize}
    \item<4-> Coordinaten $x\in[0,94]$ en $y\in[0,62]$
  \end{itemize}
  \item<7-> See also: \\
  	http://tibasicdev.wikidot.com/friendly-window
\end{itemize}

\vspace{3cm}

\begin{tikzpicture}[overlay,remember picture]
	\node[yshift=0.6cm] (BR) at (current page.south east){ };
	\node [anchor=south east,xshift=0.195cm] at (BR)
	{%
	\only<4->{%
		\begin{ticalc}
			\:ZStandard\\%
			\:84\>Xmin\\%
			\:72\>Ymax\\%
			\:ZInteger
		\end{ticalc}
	}%
	};
	\node [anchor=south east,xshift=-3.5cm] at (BR)
	{%
	\only<5>{%
		\begin{ticalc}
			\select{ZOOM}\,MEMORY\\%
			3\arrowup Zoom\,Out\\%
			4\:ZDecimal\\%
			5\:ZSquare\\%
			\selectitem{6\:}ZStandard\\%
			7\:ZTrig\\%
			\selectitem{8\:}ZInteger\\%
			9\arrowdown ZoomStat
		\end{ticalc}
	}%
	\only<6->{%
		\begin{ticalc}
			\select{VARS}\,Y-VARS\\%
			\selectitem{\one\:}Window\ldots\\%
			2\:Zoom\ldots\\%
			3\:GDB\ldots\\%
			4\:Picture\ldots\\%
			5\:Statistics\ldots\\%
			6\:Table\ldots\\%
			7\:String\ldots
		\end{ticalc}
	}%
	};
	\node [anchor=south east,xshift=-7.195cm] at (BR)
	{%
	\only<6->{%
		\begin{ticalc}
			\select{X/Y}\,T/\theta\,U/V/W\\%
			\selectitem{\one\:}Xmin\\%
			2\:Xmax\\%
			3\:Xscl\\%
			4\:Ymin\\%
			\selectitem{5\:}Ymax\\%
			6\:Yscl\\%
			7\arrowdown Xres
		\end{ticalc}
	}%
	};
\end{tikzpicture}


\end{frame}




\begin{frame}
\frametitle{\tifonttxt{prgm\theta FWINDOW}}

\visible<1->{\ticalcfig{\ticalcfigCircleSecond\ticalcfigCircle{\ticalcfigCircleColThree}{0.944}}} % format

\begin{itemize}
  \item \lenitem{Het is handig om een subprogramma te maken om een friendly window te activeren.}
  \item \lenitem{Maak dit \tifonttxt{prgm} na!}
\end{itemize}

\vspace{3cm}


\begin{tikzpicture}[overlay,remember picture]
	\node[yshift=0.6cm] (BR) at (current page.south east){ };
	\node [anchor=south east,xshift=0.195cm] at (BR)
	{%
	\only<1->{%
		\begin{ticalc}[3.5cm]
			PROGRAM\:\theta FWINDOW\\%
			\:FnOff\\%
			\:GridOff\\%
			\:AxesOff\\%
			\:ZStandard\\%
			\:84\>Xmin\\%
			\:72\>Ymax\\%
			\:ZInteger\\%
			\:ClrDraw
		\end{ticalc}
	}%
	};
\end{tikzpicture}

\end{frame}



\subsection{Basic drawing functions}

% 1) shapes (line,circle,..)
% 2) pixel on/off

\begin{frame}
\frametitle{Schermgrootte \& Draw line}

\visible<3->{\ticalcfig{\ticalcfigCircleSecond\ticalcfigCircle{\ticalcfigCircleColThree}{0.615}}} % draw

\begin{itemize}
  \item<1-> TI-84: 96x64 pixels
	  \begin{itemize}
	  	\item<2-> \lenitem{Om te tekenen: $x\in[0,94]$ en $y\in[0,62]$}
	  \end{itemize}
  \item<3-> In het \tiDRAW-menu staan functies om te tekenen.
  \item<4-> \tifonttxt{prgmOUTLINE} kleurt alle buitenste pixels.
\end{itemize}

\vspace{2cm}


\begin{tikzpicture}[overlay,remember picture]
	\node[yshift=0.6cm] (BR) at (current page.south east){ };
	\node [anchor=south east,xshift=0.195cm] at (BR)
	{%
	\only<3->{%
		\begin{ticalc}
			\select{DRAW}\,POINTS\,STO\\%
			\one\:ClrDraw\\%
			\selectitem{2\:}Line(\\%
			3\:Horizontal\\%
			4\:Vertical\\%
			5\:Tangent(\\%
			6\:DrawF\\%
			7\arrowdown Shade(
		\end{ticalc}
	}%
	};
	\node [anchor=south east,xshift=-3.5cm] at (BR)
	{%
	\only<4->{%
		\begin{ticalc}[3.5cm]
			PROGRAM\:OUTLINE
			\:ZStandard\\%
			\:84\>Xmin\\%
			\:72\>Ymax\\%
			\:ZInteger\\%
			\:Line(0\comma0\comma0\comma62)\\%
			\:Line(0\comma0\comma94\comma0)\\%
			\:Line(0\comma62\comma94\comma62)\\%
			\:Line(94\comma0\comma94\comma62)
		\end{ticalc}
	}%
	};
\end{tikzpicture}

\end{frame}



\begin{frame}
\frametitle{Let's draw!}

\visible<1->{\ticalcfig{\ticalcfigCircleSecond\ticalcfigCircle{\ticalcfigCircleColThree}{0.615}}} % draw

\vspace{-1.8cm}

\begin{itemize}
  \item<1-> Alles om te tekenen staat in het \tiDRAW-menu.
  \item<2-> Teken eens:
  \begin{enumerate}
    \item Een lijn \tifonttxt{\>} een driehoek
    \item Een cirkel
    \item Een smily
    \item Wat je wilt!
  \end{enumerate}
  \item<3-> Let op hoe snel/handig alle functies zijn.
  \item<4-> \lenitem{Dit hoeft niet in een \tifonttxt{prgm}: kan gewoon op het hoofdscherm.}
\end{itemize}

\vspace{1cm}


\begin{tikzpicture}[overlay,remember picture]
	\node[yshift=0.6cm] (BR) at (current page.south east){ };
	\node [anchor=south east,xshift=0.195cm] at (BR)
	{%
	\only<1->{%
		\begin{ticalc}
			\select{DRAW}\,POINTS\,STO\\%
			\one\:ClrDraw\\%
			2\:Line(\\%
			3\:Horizontal\\%
			4\:Vertical\\%
			5\:Tangent(\\%
			6\:DrawF\\%
			7\arrowdown Shade(
		\end{ticalc}
	}%
	};
	\node [anchor=south east,xshift=-3.5cm] at (BR)
	{%
	\only<2->{%
		\begin{ticalc}[4.7cm]
			\:Line(5\comma5\comma35\comma35)\\%
			\:Circle(20\comma20\comma10)\\%
			\:DrawF\,X\^1.1\\%
			\:Shade(X\comma X\sq\comma0\comma10\comma1\comma1)\\%
			\:Shade(X\comma X\sq\comma10\comma20\comma1\comma2)\\%
			\:Shade(X\comma X\sq\comma20\comma30\comma1\comma4)\\%
			\:Shade(X\comma X\sq\comma30\comma40\comma2\comma4)\\%
			\:Shade(X\comma X\sq\comma40\comma99\comma3\comma7)
		\end{ticalc}
	}%
	};
	\node [anchor=south east,xshift=-8.645cm] at (BR)
	{%
	\only<1->{%
		\begin{ticalc}
			\:ZStandard\\%
			\:84\>Xmin\\%
			\:72\>Ymax\\%
			\:ZInteger
		\end{ticalc}
	}%
	};
\end{tikzpicture}
  
\end{frame}


\begin{frame}
\frametitle{\tiDRAW}

\visible<1->{\ticalcfig{\ticalcfigCircleSecond\ticalcfigCircle{\ticalcfigCircleColThree}{0.615}}} % draw

\begin{itemize}
  \item<1-> Alles om te tekenen staat in het \tiDRAW-menu.
  \item<2-> Een lijn tekenen is vrij snel\ldots
  \item<3-> Maar de \tifonttxt{Circle(} functie is zeer langzaam!
  \begin{itemize}
    \item<4-> Ongeschikt voor games!
  \end{itemize}
\end{itemize}

\vspace{2cm}


\begin{tikzpicture}[overlay,remember picture]
	\node[yshift=0.6cm] (BR) at (current page.south east){ };
	\node [anchor=south east,xshift=0.195cm] at (BR)
	{%
	\only<1->{%
		\begin{ticalc}
			\select{DRAW}\,POINTS\,STO\\%
			\one\:ClrDraw\\%
			2\:Line(\\%
			3\:Horizontal\\%
			4\:Vertical\\%
			5\:Tangent(\\%
			6\:DrawF\\%
			7\arrowdown Shade(
		\end{ticalc}
	}%
	};
\end{tikzpicture}

\end{frame}





\begin{frame}
\frametitle{Per pixel drawing}

\visible<1->{\ticalcfig{\ticalcfigCircleSecond\ticalcfigCircle{\ticalcfigCircleColThree}{0.615}}} % draw

\begin{itemize}
  \item<1-> \lenitem{Je kunt ook pixels/punten individueel aan en uit zetten.}
  \item<2-> Dit bevind zich onder \inlineticalc{POINTS} in het \tiDRAW-menu.
  \item<3-> Waarom je dit ooit zou doen?
  \visible<4->{\begin{itemize}
    \item<4-> Veel sneller!! Je wilt een real-time game!
  \end{itemize}}
  \item<5-> Maar kost veel geheugen om iets te tekenen\ldots
\end{itemize}


\begin{tikzpicture}[overlay,remember picture]
	\node[yshift=0.6cm] (BR) at (current page.south east){ };
	\node [anchor=south east,xshift=0.195cm] at (BR)
	{%
	\only<1>{%
		\begin{ticalc}
			\select{DRAW}\,POINTS\,STO\\%
			\selectitem{\one\:}ClrDraw\\%
			2\:Line(\\%
			3\:Horizontal\\%
			4\:Vertical\\%
			5\:Tangent(\\%
			6\:DrawF\\%
			7\arrowdown Shade(
		\end{ticalc}
	}%
	\only<2->{%
		\begin{ticalc}
			DRAW\,\select{POINTS}\,STO\\%
			\selectitem{\one\:}Pt-On(\\%
			2\:Pt-Off(\\%
			3\:Pt-Change(\\%
			4\:Pxl-On(\\%
			5\:Pxl-Off(\\%
			6\:Pxl-Change(\\%
			7\:pxl-Test(
		\end{ticalc}
	}%
	};
\end{tikzpicture}


\end{frame}





\begin{frame}
\frametitle{Per pixel drawing}

\visible<1->{\ticalcfig{\ticalcfigCircleSecond\ticalcfigCircle{\ticalcfigCircleColThree}{0.615}}} % draw

\begin{itemize}
  \item<1-> Punten (x,y) beginnen linksonderin.
  \item<2-> Pixels (R,C) beginnen linksbovenin.
  \begin{itemize}
    \item R (rij) telt verticaal \underline{van} boven \underline{naar} beneden!
    \item C (kolom) telt horizontaal
  \end{itemize}
  \item<3-> Bij welk punt hoort pixel (R,C)? (Tip: $y_{max}=62$)
  \visible<5->{\begin{itemize}
    \item Punt (C,62-R)
  \end{itemize}}
  \item<4-> En bij welke pixel hoort punt (x,y)? 
  \visible<5->{\begin{itemize}
    \item Pixel (62-y,x)
  \end{itemize}}
\end{itemize}


\begin{tikzpicture}[overlay,remember picture]
	\node[yshift=0.6cm] (BR) at (current page.south east){ };
	\node [anchor=south east,xshift=0.195cm] at (BR)
	{%
	\only<1->{%
		\begin{ticalc}
			DRAW\,\select{POINTS}\,STO\\%
			\only<2-3,5>{\one\:}\only<1,4>{\selectitem{\one\:}}Pt-On(\\%
			2\:Pt-Off(\\%
			3\:Pt-Change(\\%
			\only<1,4,5>{4\:}\only<2-3>{\selectitem{4\:}}Pxl-On(\\%
			5\:Pxl-Off(\\%
			6\:Pxl-Change(\\%
			7\:pxl-Test(
		\end{ticalc}
	}%
	};
\end{tikzpicture}


\end{frame}




\subsection{Sprites}

% Background image: StorePic & RecallPic
% Plot sprites
% 2*4 digits = line
% Animating a sprite



\begin{frame}
\frametitle{Sprite: Meerdere lijnen}

\begin{itemize}
  \item<1-> In een game wil je vaak een object tekenen, zoals pacman.
  \item<2-> Zo'n object heet een ``sprite''.
  \item<3-> Op de TI84 is het een combinatie van meerdere pixels of lijnen.
\end{itemize}

\vspace{3cm}


\begin{tikzpicture}[overlay,remember picture]
	\node[yshift=0.6cm] (BR) at (current page.south east){ };
	\node [anchor=south east,xshift=0.195cm] at (BR)
	{%
	\only<4->{%
		\begin{ticalc}[3.5cm]
			PROGRAM\:SPRITE\\%
			\:prgm\theta FWINDOW\\%
			\:Line(10\comma10\comma20\comma20\\%
			\:Line(20\comma20\comma30\comma10\\%
			\:Line(30\comma10\comma20\comma15\\%
			\:Line(20\comma15\comma10\comma10
		\end{ticalc}
	}%
	};
	\node [anchor=south east,xshift=-3.75cm] at (BR)
	{%
	\only<4->{%
		\begin{ticalc}[3.5cm]
			PROGRAM\:\theta FWINDOW\\%
			\:FnOff\\%
			\:GridOff\\%
			\:AxesOff\\%
			\:ZStandard\\%
			\:84\>Xmin\\%
			\:72\>Ymax\\%
			\:ZInteger\\%
			\:ClrDraw
		\end{ticalc}
	}%
	};
\end{tikzpicture}

\end{frame}


\begin{frame}
\frametitle{Plot Sprites}

\begin{itemize}
  \item<1-> Hoe kan dit effici\"enter?
  \visible<2->{\item<2-> Store alle getallen in een list!
  \item<3-> Met ``Plot'' kun je \textbf{maximaal 3} sprites maken.
  \item<4-> Voordeel: Sprite makkelijk te bewerken en bewegen.
  \item<5-> Nadeel: \tifonttxt{DispGraph} overschrijft het hele scherm.}
\end{itemize}

\vspace{2.5cm}

\begin{tikzpicture}[overlay,remember picture]
	\node[yshift=0.6cm] (BR) at (current page.south east){ };
	\node [anchor=south east,xshift=0.195cm] at (BR)
	{%
	\only<1->{%
		\begin{ticalc}[3.5cm]
			PROGRAM\:SPRITE\\%
			\:prgm\theta FWINDOW\\%
			\:Line(10\comma10\comma20\comma20\\%
			\:Line(20\comma20\comma30\comma10\\%
			\:Line(30\comma10\comma20\comma15\\%
			\:Line(20\comma15\comma10\comma10
		\end{ticalc}
	}%
	};
	\node [anchor=south east,xshift=-3.7cm] at (BR)
	{%
	\only<2->{%
		\begin{ticalc}[4cm]
			PROGRAM\:SPRITEPL\\%
			\:prgm\theta FWINDOW\\%
			\:\{10\comma20\comma30\comma20\comma10\>L\sub{1}\\%
			\:\{10\comma20\comma10\comma15\comma10\>L\sub{2}\\%
			\visible<3->{\:Plot1(xyLine\comma L\sub{1}\comma L\sub{2}\\%
			\:DispGraph}
		\end{ticalc}
	}%
	};
	\node [anchor=south east,xshift=-8.1cm] at (BR)
	{%
	\only<4->{%
		\begin{ticalc}[4cm]
			\qt TRANSLEER\,X\:\\%
			L\sub{1}+5\>L\sub{1}\\%
			\qt ROTEER\:\\%
			L\sub{1}cos(\theta)-L\sub{2}sin(\theta)\>L\sub{3}\\%
			L\sub{1}sin(\theta)+L\sub{2}cos(\theta)\>L\sub{2}\\%
			L\sub{3}\>L\sub{1}
		\end{ticalc}
	}%
	};
\end{tikzpicture}

\end{frame}



\begin{frame}
\frametitle{Looping Line Sprites}

\begin{itemize}
  \item<2-> We kunnen ook zelf met \tifonttxt{Line} over een list loopen!
  \item<3-> Voordeel: Cleared het scherm niet, unlike \tifonttxt{DispGraph}.
  \item<3-> Nadeel: Tekent langzamer\ldots
  \item<4-> Voor kleine sprites kun je ook met pixels tekenen.
\end{itemize}

\vspace{2.5cm}

\begin{tikzpicture}[overlay,remember picture]
	\node[yshift=0.6cm] (BR) at (current page.south east){ };
	\node [anchor=south east,xshift=0.195cm] at (BR)
	{%
	\only<1->{%
		\begin{ticalc}[3.5cm]
			PROGRAM\:SPRITE\\%
			\:prgm\theta FWINDOW\\%
			\:Line(10\comma10\comma20\comma20\\%
			\:Line(20\comma20\comma30\comma10\\%
			\:Line(30\comma10\comma20\comma15\\%
			\:Line(20\comma15\comma10\comma10
		\end{ticalc}
	}%
	};
	\node [anchor=south east,xshift=-3.75cm] at (BR)
	{%
	\only<2->{%
		\begin{ticalc}[3.8cm]
			PROGRAM\:SPRITELL\\% loop line
			\:prgm\theta FWINDOW\\%
			\:\{\one0\comma20\comma30\comma20\comma\one0\>L\sub{\one}\\%
			\:\{\one0\comma20\comma\one0\comma\one5\comma\one0\>L\sub{2}\\%
			\:For(X\comma1\comma dim(L\sub{1})-1\\%
			\:Line(L\sub{1}(X)\comma L\sub{2}(X)\comma\\%
				L\sub{1}(X+1)\comma L\sub{2}(X+1\\%
			\:End
		\end{ticalc}
	}%
	};
	\node [anchor=south east,xshift=-8cm] at (BR)
	{%
	\only<4->{%
		\begin{ticalc}[3.8cm]
			PROGRAM\:SPRITELP\\% loop point
			\:prgm\theta FWINDOW\\%
			\:\{\one0\comma20\comma30\comma20\>L\sub{\one}\\%
			\:\{\one0\comma20\comma\one0\comma\one5\>L\sub{2}\\%
			\:For(X\comma1\comma dim(L\sub{1})-1\\%
			\:Pt-On(L\sub{1}(X)\comma L\sub{2}(X
			\:End
		\end{ticalc}
	}%
	};
\end{tikzpicture}


\end{frame}



\begin{frame}
\frametitle{Compressed Lines}

\begin{itemize}
  \item<2-> In al het voorgaande werd een sprite getekend met aaneengesloten lijnen.
  \item<3-> Alternatief: compressie van een lijn in een enkel getal.
  \item<4-> Voordeel: flexibel en slechts 1 list.
  \item<4-> Nadeel: lastiger om mee te werken.
\end{itemize}

\vspace{2.5cm}

\begin{tikzpicture}[overlay,remember picture]
	\node[yshift=0.6cm] (BR) at (current page.south east){ };
	\node [anchor=south east,xshift=0.195cm] at (BR)
	{%
	\only<3->{%
		\begin{ticalc}[4cm]
			PROGRAM\:SPRITES1\\%
			\:prgm\theta FWINDOW\\%
			\:\{10102020\comma20203010
				\comma30102015\comma20151010\\%
			\:For(X\comma1\comma dim(Ans\\%
			\:Line(iPart(Ans(X)\\%
				/\E6),iPart(\E2fPart(\\%
				Ans(X)/\E6)),iPart(\E2\\%
				fPart(Ans(X)/\E4)),\E2\\%
				fPart(Ans(X)/\E2\\%
			\:End
		\end{ticalc}
	}%
	};
	\node [anchor=south east,xshift=-4.25cm] at (BR)
	{%
	\only<1->{%
		\begin{ticalc}[3.8cm]
			PROGRAM\:SPRITELL\\%
			\:prgm\theta FWINDOW\\%
			\:\{\one0\comma20\comma30\comma20\comma\one0\>L\sub{\one}\\%
			\:\{\one0\comma20\comma\one0\comma\one5\comma\one0\>L\sub{2}\\%
			\:For(X\comma1\comma dim(L\sub{1})-1\\%
			\:Line(L\sub{1}(X)\comma L\sub{2}(X)\comma\\%
				L\sub{1}(X+1)\comma L\sub{2}(X+1\\%
			\:End
		\end{ticalc}
	}%
	};
	\node [anchor=south east,xshift=-8.5cm] at (BR)
	{%
	\only<1->{%
		\begin{ticalc}[3.5cm]
			PROGRAM\:SPRITE\\%
			\:prgm\theta FWINDOW\\%
			\:Line(10\comma10\comma20\comma20\\%
			\:Line(20\comma20\comma30\comma10\\%
			\:Line(30\comma10\comma20\comma15\\%
			\:Line(20\comma15\comma10\comma10
		\end{ticalc}
	}%
	};
\end{tikzpicture}

\end{frame}


\begin{frame}
\frametitle{Compressed Lines}

\begin{itemize}
  \item<1-> Dezelfde \tifonttxt{For}-loop kan gebruikt worden voor \textit{elke} sprite!
\end{itemize}

\vspace{4cm}

\begin{tikzpicture}[overlay,remember picture]
	\node[yshift=0.6cm] (BR) at (current page.south east){ };
	\node [anchor=south east,xshift=0.195cm] at (BR)
	{%
	\only<1->{%
		\begin{ticalc}[4cm]
			PROGRAM\:SPRITES1\only<2->{a}\\%
			\:prgm\theta FWINDOW\\%
			\:\{10102020\comma20203010
				\comma30102015\comma20151010\\%
			\:For(X\comma1\comma dim(Ans\\%
			\:Line(iPart(Ans(X)\\%
				/\E6),iPart(\E2fPart(\\%
				Ans(X)/\E6)),iPart(\E2\\%
				fPart(Ans(X)/\E4)),\E2\\%
				fPart(Ans(X)/\E2\\%
			\:End
		\end{ticalc}
	}%
	};
	\node [anchor=south east,xshift=-4.25cm] at (BR)
	{%
	\only<2->{%
		\begin{ticalc}[4cm]
			PROGRAM\:SPRITES1\only<2->{b}\\%
			\:prgm\theta FWINDOW\\%
			\:\{10102020\comma20203010
				\comma30102015\comma20151010\\%
			\:prgm\theta DRAWSPT
		\end{ticalc}
	}%
	};
	\node [anchor=south east,xshift=-4.25cm,yshift=1.85cm] at (BR)
	{%
	\only<2->{%
		\begin{ticalc}[4cm]
			PROGRAM\:\theta DRAWSPT\\%
			\:For(X\comma1\comma dim(Ans\\%
			\:Line(iPart(Ans(X)\\%
				/\E6),iPart(\E2fPart(\\%
				Ans(X)/\E6)),iPart(\E2\\%
				fPart(Ans(X)/\E4)),\E2\\%
				fPart(Ans(X)/\E2\\%
			\:End
		\end{ticalc}
	}%
	};
\end{tikzpicture}

\end{frame}




\begin{frame}
\frametitle{Animation 1}

\begin{itemize}
  \item<1-> Een animatie is niets anders dan steeds een andere sprite tekenen (of andere positie/rotatie/\ldots).
  \item<2-> Wat doet deze animatie?
  \item<3-> \visible<3->{PROBLEEM: Je ziet alle vorige sprites ook\ldots}
\end{itemize}

\vspace{2.5cm}

\begin{tikzpicture}[overlay,remember picture]
	\node[yshift=0.6cm] (BR) at (current page.south east){ };
	\node [anchor=south east,xshift=0.195cm] at (BR)
	{%
	\only<2->{%
		\begin{ticalc}[4cm]
			PROGRAM\:SPRITES2\\%
			\:prgm\theta FWINDOW\\%
			\:\{10102020\comma20203010
				\comma30102015\comma20151010\>L\sub{1}\\%
			\:For(I\comma\one\comma10\\%
			\:\qt UPDATE\\%
			\:L\sub{1}+2(\E4+1)\>L\sub{1}\\%
			\:\qt DRAW\\%
			\:L\sub{1}\:prgm\theta DRAWSPT\\%
			\:End
		\end{ticalc}
	}%
	};
	\node [anchor=south east,xshift=-4.25cm] at (BR)
	{%
	\only<1->{%
		\begin{ticalc}[4cm]
			PROGRAM\:\theta DRAWSPT\\%
			\:For(X\comma1\comma dim(Ans\\%
			\:Line(iPart(Ans(X)\\%
				/\E6),iPart(\E2fPart(\\%
				Ans(X)/\E6)),iPart(\E2\\%
				fPart(Ans(X)/\E4)),\E2\\%
				fPart(Ans(X)/\E2\\%
			\:End
		\end{ticalc}
	}%
	};
\end{tikzpicture}

\end{frame}


\begin{frame}
\frametitle{Animation 2}

\hspace{-1cm}
\begin{minipage}{\textwidth}
\begin{itemize}
  \item<2-> OPLOSSING: Gum in een lijn
  \item<3-> PROBLEEM: Overschrijft de achtergrond
\end{itemize}
\end{minipage}

\vspace{2.5cm}

\begin{tikzpicture}[overlay,remember picture]
	\node[yshift=0.6cm] (BR) at (current page.south east){ };
	\node [anchor=south east,xshift=0.195cm] at (BR)
	{%
	\only<1->{%
		\begin{ticalc}[4cm]
			PROGRAM\:SPRITES2\\%
			\:prgm\theta FWINDOW\\%
			\:\{10102020\comma20203010
				\comma30102015\comma20151010\>L\sub{1}\\%
			\:For(I\comma\one\comma10\\%
			\:\qt UPDATE\\%
			\:L\sub{1}+2(\E4+1)\>L\sub{1}\\%
			\:\qt DRAW\\%
			\:L\sub{1}\:prgm\theta DRAWSPT\\%
			\:End
		\end{ticalc}
	}%
	};
	\node [anchor=south east,xshift=-4.25cm] at (BR)
	{%
	\only<1->{%
		\begin{ticalc}[4cm]
			PROGRAM\:\theta DRAWSPT\\%
			\:For(X\comma1\comma dim(Ans\\%
			\:Line(iPart(Ans(X)\\%
				/\E6),iPart(\E2fPart(\\%
				Ans(X)/\E6)),iPart(\E2\\%
				fPart(Ans(X)/\E4)),\E2\\%
				fPart(Ans(X)/\E2\select{)\comma B}\\%
			\:End
		\end{ticalc}
	}%
	};
	\node [anchor=south east,xshift=0.195cm,yshift=3.55cm] at (BR)
	{%
	\only<2->{%
		\begin{ticalc}[4cm]
			PROGRAM\:SPRITES3\\%
			\:prgm\theta FWINDOW\\%
			\:\{10102020\comma20203010
				\comma30102015\comma20151010\>L\sub{1}\\%
			\:For(I\comma\one\comma10\\%
			\:\qt UPDATE\\%
			\:\select{0\>B}\:L\sub{1}\:prgm\theta DRAWSPT\\%
			\:L\sub{1}+2(\E4+1)\>L\sub{1}\\%
			\:\qt DRAW\\%
			\:\select{1\>B}\:L\sub{1}\:prgm\theta DRAWSPT\\%
			\:End
		\end{ticalc}
	}%
	};
\end{tikzpicture}

\end{frame}






\begin{frame}
\frametitle{Background image: \tifonttxt{StorePic}}

\visible<2->{\ticalcfig{\ticalcfigCircleSecond\ticalcfigCircle{\ticalcfigCircleColThree}{0.615}}} % draw

\begin{itemize}
  \item<1-> \lenitem{Indien je iets stilstaands hebt, zoals een title screen of een background, dan wil je vaak dat het niet overschreven kan worden.}
  \item<2-> \lenitem{Met \tifonttxt{StorePic} kun je het opslaan in een plaatje (alleen mogelijk met het \textit{hele} scherm).}
  \item<3-> \lenitem{Met \tifonttxt{RecallPic} kun je dat plaatje weer tekenen (bovenop wat er al op je scherm staat).}
  \item<4-> \lenitem{Je kunt maximaal 10 \tifonttxt{Pic}s opslaan.}
\end{itemize}

\vspace{2.5cm}

\begin{tikzpicture}[overlay,remember picture]
	\node[yshift=0.6cm] (BR) at (current page.south east){ };
	\node [anchor=south east,xshift=0.195cm] at (BR)
	{%
	\only<2->{%
		\begin{ticalc}
			DRAW\,POINTS\,\select{STO}\\%
			\only<3->{\one\:}\only<2>{\selectitem{\one\:}}StorePic\\%
			\only<2>{2\:}\only<3->{\selectitem{2\:}}RecallPic\\%
			3\:StoreGDB(\\%
			4\:RecallGDB
		\end{ticalc}
	}%
	};
	\node [anchor=south east,xshift=-3.5cm] at (BR)
	{%
	\only<2->{%
		\begin{ticalc}[3.5cm]
			PROGRAM\:SPRITEBG\\%
			\:prgm\theta FWINDOW\\%
			\:Circle(10\comma10\comma30\\%
			\:StorePic\,\only<4>{\select{4}}\only<-3>{4}\\%
			\visible<3->{\:\ldots do\,stuff\ldots\\%
			\:RecallPic\,\only<4>{\select{4}}\only<-3>{4}}
		\end{ticalc}
	}%
	};
\end{tikzpicture}

\end{frame}




\begin{frame}
\frametitle{Caching}

\hspace{-1cm}
\begin{minipage}{\textwidth}
\begin{itemize}
  \item<1-> `Caching' is het opslaan van een plaatje voor later gebruik.
  \item<2-> Stel dat je sprite over het scherm beweegt:
  \begin{itemize}
    \item<3-> Indien je blijft tekenen zie je een heel spoor waar je sprite geweest is.
    \item<4-> Zijn vorige positie moeten we clearen!
  \end{itemize}
  \item<5-> Dit kan met `caching':\\
  			\tifonttxt{StorePic} en \tifonttxt{RecallPic}.
\end{itemize}
\end{minipage}


\vspace{2.5cm}

\begin{tikzpicture}[overlay,remember picture]
	\node[yshift=0.6cm] (BR) at (current page.south east){ };
	\node [anchor=south east,xshift=0.195cm] at (BR)
	{%
	\only<1->{%
		\begin{ticalc}[4cm]
			PROGRAM\:SPRITES4\\%
			\:prgm\theta FWINDOW\\%
			\:Circle(10\comma10\comma30\\%
			\:\only<-4>{StorePic\,4}\only<5->{\select{StorePic\,4}}\\%
			\:\{10102020\comma20203010
				\comma30102015\comma20151010\>L\sub{1}\\%
			\:For(I\comma\one\comma10\\%
			\:\qt UPDATE\\%
			\:L\sub{1}+2(\E4+1)\>L\sub{1}\\%
			\:\qt DRAW\\%
			\visible<5->{\:\select{ClrDraw\:RecallPic\,4}\\}%
			\:L\sub{1}\:prgm\theta DRAWSPT\\%
			\:End
		\end{ticalc}
	}%
	};
\end{tikzpicture}

\end{frame}







\begin{frame}
\frametitle{Animation 3: Clear small sprites}

\begin{itemize}
  \item<1-> Small sprites kunnen gecleared worden door spaties bovenop de sprite te tekenen.
  \item<2-> PROBLEEM: Werkt alleen makkelijk\\
  			voor kleine sprites (max. 5 pixels hoog):\\
  			Waardeloos voor dit voorbeeld!
  \item<3-> PROBLEEM: Achtergrond verdwijnt.
\end{itemize}

\vspace{2.5cm}



\begin{tikzpicture}[overlay,remember picture]
	\node[yshift=0.6cm] (BR) at (current page.south east){ };
	\node [anchor=south east,xshift=0.195cm] at (BR)
	{%
	\only<1->{%
		\begin{ticalc}[4cm]
			PROGRAM\:SPRITES5\\%
			\:prgm\theta FWINDOW\\%
			\:Circle(10\comma10\comma30\\%
			\:\{10102020\comma20203010
				\comma30102015\comma20151010\>L\sub{1}\\%
			\:For(I\comma\one\comma10\\%
			\:\qt UPDATE\\%
			\:L\sub{1}+2(\E4+1)\>L\sub{1}\\%
			\:\qt DRAW\\%
			\:Text(62-10\comma10\comma\\%
				\qt\tiSpace\tiSpace\tiSpace\tiSpace\tiSpace\tiSpace\tiSpace\tiSpace\tiSpace\\%
			\:L\sub{1}\:prgm\theta DRAWSPT\\%
			\:End
		\end{ticalc}
	}%
	};
\end{tikzpicture}



\end{frame}




\begin{frame}
\frametitle{Exercise}


Maak zelf een simpele sprite die langzaam van links naar rechts beweegt.


\end{frame}





