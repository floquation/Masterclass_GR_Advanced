\section{Goto \& Labels}

\begin{frame}
\frametitle{R-r-r-r-epeat: Herhaling}

\begin{itemize}
  \item<1-> In een programma komt het vaak voor dat je een opdracht wilt herhalen.
  \item<2-> Kijk bijvoorbeeld naar \inlineticalc{PROGRAM\:WIEWINT} van de exercises van vorige les.
  \item<3-> Soms wil je een opdracht een gegeven aantal keer herhalen.
  	\visible<4->{Dit is heel vaak hetzelfde programmeren\ldots Stel je voor dat je iets 132x wilt doen!}
  \item<5-> Soms wil je een opdracht een van-te-voren onbekend aantal keer herhalen\ldots
  	\visible<6->{Dat is niet te programmeren\ldots}
  \item<7-> Voor dit soort gevallen, bestaan er opdrachten die iets herhalen:
  	\tifonttxt{For}, \tifonttxt{While} en \tifonttxt{Goto} met \tifonttxt{Lbl}.
\end{itemize}

\end{frame}



\begin{frame}
\frametitle{Het \tifonttxt{Lbl} (Label) statement}

\begin{itemize}
  \item<1-> Het \tifonttxt{Lbl} statement doet \emph{niets} op zichzelf.
  \item<2-> Het markeert een locatie in het programma.
  \item<3-> Deze locatie kan elders in het programma gebruikt worden.
  \item<4-> \tifonttxt{Lbl} staat onder \tiPRGM\,in het \tifonttxt{CTL}-menu.
  \item<5-> Blader daarvoor naar beneden \tiDown.
\end{itemize}


\begin{tikzpicture}[overlay,remember picture]
	\node[] (BL) at (current page.south east){ };
	\node [anchor=south east,xshift=0.195cm,yshift=0.6cm] at (BL)
	{%
	\only<4>{%
		\begin{ticalc}
			\select{CTL}\,I/O\,EXEC \\
			\selectitem{1\+\:}If \\
			2\:Then \\
			3\:Else \\
			4\:For( \\
			5\:While \\
			6\:Repeat( \\
			7\arrowdown End
		\end{ticalc}
	}%
	\only<5>{%
		\begin{ticalc}
			\select{CTL}\,I/O\,EXEC \\
			4\arrowup For( \\
			5\:While \\
			6\:Repeat( \\
			7\:End \\
			8\:Pause \\
			\selectitem{9\:}Lbl \\
			0\arrowdown Goto
		\end{ticalc}
	}%
	};
\end{tikzpicture}

\end{frame}




\begin{frame}
\frametitle{Het \tifonttxt{Goto} (Ga naar) statement}

\begin{itemize}
  \item<1-> Het \tifonttxt{Goto} statement heeft \tifonttxt{Lbl} nodig.
  \item<2-> Wanneer een programma bij de \tifonttxt{Goto}-regel aan komt,
  	dan springt hij naar de bijbehorende \tifonttxt{Lbl} en gaat daar verder met het uitvoeren van het programma.
  \item<3-> Je kunt zo dus door het hele programma heen springen!
  \item<4-> \tifonttxt{Goto} staat onder \tiPRGM\,in het \tifonttxt{CTL}-menu.
  \item<5-> Blader daarvoor naar beneden \tiDown.
\end{itemize}


\begin{tikzpicture}[overlay,remember picture]
	\node[] (BL) at (current page.south east){ };
	\node [anchor=south east,xshift=0.195cm,yshift=0.6cm] at (BL)
	{%
	\only<4>{%
		\begin{ticalc}
			\select{CTL}\,I/O\,EXEC \\
			\selectitem{1\+\:}If \\
			2\:Then \\
			3\:Else \\
			4\:For( \\
			5\:While \\
			6\:Repeat( \\
			7\arrowdown End
		\end{ticalc}
	}%
	\only<5>{%
		\begin{ticalc}
			\select{CTL}\,I/O\,EXEC \\
			4\arrowup For( \\
			5\:While \\
			6\:Repeat( \\
			7\:End \\
			8\:Pause \\
			9\:Lbl \\
			\selectitem{0\arrowdown} Goto
		\end{ticalc}
	}%
	};
\end{tikzpicture}

\end{frame}





\begin{frame}
\frametitle{Het \tifonttxt{Goto} (Ga naar) statement}
\framesubtitle{Voorbeeld}

\begin{itemize}
  \item<1-> Voorbeeld: Het programma slaat hier \inlineticalc{\:Disp\,\qt DIT WORDT NOOIT WEERGEVEN\qt} over
  	en springt in \'e\'en keer naar \inlineticalc{\:Disp\,\qt DIT WORDT ALTIJD UITGEVOERD\qt}.
  	Het negeert dus \inlineticalc{Stop}: Dat wordt nooit uitgevoerd.
  \item<2-> Labels kunnen een naam hebben bestaande uit cijfers en letters, met maximaal 2 karakters.
	  \begin{itemize}
	    \item Bijvoorbeeld \tifonttxt{AB} of \tifonttxt{0A}, maar niet \tifonttxt{TUD2}.
	  \end{itemize}
\end{itemize}

\vspace{1.5cm}

\begin{tikzpicture}[overlay,remember picture]
	\node[] (BL) at (current page.south east){ };
	\node [anchor=south east,xshift=0.195cm,yshift=0.6cm] at (BL)
	{%
	\only<1>{%
		\begin{ticalc}
			PROGRAM\:GOTOLBL0\\%
			\:Goto\,0\\%
			\:Disp\,\qt DIT WORDT NOOIT WEERGEVEN\qt\\%
			\:Stop\\%
			\:Lbl\,0\\%
			\:Disp\,\qt DIT WORDT ALTIJD UITGEVOERD\qt
		\end{ticalc}
	}%
	\only<2>{%
		\begin{ticalc}
			PROGRAM\:GOTOLBL1\\%
			\:Goto\,TU\\%
			\:Disp\,\qt DIT WORDT NOOIT WEERGEVEN\qt\\%
			\:Stop\\%
			\:Lbl\,TU\\%
			\:Disp\,\qt DIT WORDT ALTIJD UITGEVOERD\qt
		\end{ticalc}
	}%
	};
\end{tikzpicture}

\end{frame}



\begin{frame}
\frametitle{Probeer \tifonttxt{Goto} zelf!}

Schrijf zelf een klein \tifonttxt{prgm} met alleen \tifonttxt{Disp}, \tifonttxt{Lbl} en \tifonttxt{Goto} statements,
om er een gevoel voor te krijgen.


\begin{tikzpicture}[overlay,remember picture]
	\node[] (BL) at (current page.south east){ };
	\node [anchor=south east,xshift=0.195cm,yshift=0.6cm] (BL2) at (BL)
	{%
		\begin{ticalc}
			\select{CTL}\,I/O\,EXEC \\
			4\arrowup For( \\
			5\:While \\
			6\:Repeat( \\
			7\:End \\
			8\:Pause \\
			9\:Lbl \\
			\selectitem{0\arrowdown} Goto
		\end{ticalc}
	};
	\node [anchor=east,xshift=-1.5cm] (BL3) at (BL2)
	{%
		\begin{ticalc}
			\select{CTL}\,I/O\,EXEC \\
			4\arrowup For( \\
			5\:While \\
			6\:Repeat( \\
			7\:End \\
			8\:Pause \\
			\selectitem{9\:}Lbl \\
			0\arrowdown Goto
		\end{ticalc}
	};
	\node [anchor=east,xshift=-1.5cm] (BL4) at (BL3)
	{%
		\begin{ticalc}
			CTL\,\select{I/O}\,EXEC \\
			1\+\:Input \\
			2\:Prompt \\
			\selectitem{3\:}Disp \\
			4\:DispGraph \\
			5\:DispTable \\
			6\:Output( \\
			7\arrowdown getKey
		\end{ticalc}
	};
\end{tikzpicture}

\end{frame}





