\section{Herhaling statements}
\subsection{For-loops}



\begin{frame}
\frametitle{Wat is een \tifonttxt{For}-loop?}

\begin{itemize}
  \item<1-> Alhoewel \tifonttxt{Goto} en \tifonttxt{Lbl} gebruikt kunnen worden om statements te herhalen,
  	zijn er krachtigere methodes.
  \item<2-> Het \tifonttxt{For} statement herhaalt een stukje code `for (NL: voor)' bepaalde integers.
  \begin{itemize}
    \item<3-> Herinner je dat `integers' gehele getallen zijn
  \end{itemize}
  \item<4-> Hiermee kun je een gegeven aantal keren een stukje code herhalen.
  \item<5-> Bijvoorbeeld: ``Herhaal dit voor de integers 1 tot 5'' = ``Herhaal dit 5 keer''
\end{itemize}

\end{frame}





\begin{frame}
\frametitle{Wat is een \tifonttxt{For}-loop?}

\begin{itemize}
  \item<1-> Op de rekenmachine ziet dit er zo uit:
  \begin{itemize}
    \item<1-> ``Weergeef de integers 1 tot 5'' = ``Herhaal \tifonttxt{Disp} 5 keer''
  \end{itemize}
  \item<2-> Het is ook mogelijk getallen over te slaan:
  \begin{itemize}
    \item<2-> ``Weergeef de meervouden van 3 tot en met 9'' = ``Herhaal \tifonttxt{Disp} \only<-2>{<??>}\only<3->{3} keer''
  \end{itemize}
  \item<4-> De structuur is dus: \tifonttxt{For(}variable,start,end,increment\tifonttxt{)}
  \item<5-> \tifonttxt{For(} staat onder \tiPRGM\,in het \tifonttxt{CTL}-menu.
  \item<6-> De loop eindigt met de oude vertrouwde \tifonttxt{End}
\end{itemize}

\vspace{2cm}

\begin{tikzpicture}[overlay,remember picture]
	\node[] (BL) at (current page.south east){ };
	\node [anchor=south east,xshift=0.195cm,yshift=0.6cm] at (BL)
	{%
	\only<1>{%
		\begin{ticalc}
			PROGRAM\:FIRSTFOR\\%
			\:For(I\comma 1\comma 5)\\%
			\:Disp\,I\\%
			\:End
		\end{ticalc}
	}%
	\only<2-4,6>{%
		\begin{ticalc}
			PROGRAM\:FIRSTFOR\\%
			\:\only<4-6>{\select{For(I\comma 3\comma 9\comma 3)}}\only<-3>{For(I\comma 3\comma 9\comma 3)}\\%
			\:Disp\,I\\%
			\:\only<6>{\select{End}}\only<-5>{End}
		\end{ticalc}
	}%
	\only<5>{%
		\begin{ticalc}
			\select{CTL}\,I/O\,EXEC \\
			1\+\:If \\
			2\:Then \\
			3\:Else \\
			\selectitem{4\:}For( \\
			5\:While \\
			6\:Repeat( \\
			7\arrowdown End
		\end{ticalc}
	}%
	};
\end{tikzpicture}

\end{frame}


\begin{frame}
\frametitle{\tifonttxt{For(} probeer het zelf!}

Probeer de volgende \tifonttxt{Prgms} te schrijven:
\begin{enumerate}
  \item \tifonttxt{Disp} de getallen $1,2,3,4,5$
  \item \tifonttxt{Disp} de getallen $-1,0,1$
  \item \tifonttxt{Disp} de getallen $1,0,-1$ (in die volgorde!)
  \item \tifonttxt{Disp} de getallen $1,2,3,4,5,11,12,13,14,15$
  \item \tifonttxt{Disp} alle positieve even getallen tot en met 12.
  \item \lenitem[0.7\linewidth]{\tifonttxt{Disp} alle positieve oneven getallen tot en met 11.}
\end{enumerate}

\vspace{2cm}

\begin{tikzpicture}[overlay,remember picture]
	\node[yshift=0.6cm] (BL) at (current page.south east){ };
	\node [anchor=south east,xshift=0.195cm] (BL2) at (BL)
	{%
		\begin{ticalc}
			\select{CTL}\,I/O\,EXEC \\
			1\+\:If \\
			2\:Then \\
			3\:Else \\
			\selectitem{4\:}For( \\
			5\:While \\
			6\:Repeat( \\
			7\arrowdown End
		\end{ticalc}
	};
	\node [anchor=south east,xshift=-3.5cm] (BL3) at (BL)
	{%
		\begin{ticalc}
			PROGRAM\:FIRSTFOR\\%
			\:For(I\comma 3\comma 9\comma 3)\\%
			\:Disp\,I\\%
			\:End
		\end{ticalc}
	};
\end{tikzpicture}


\end{frame}



\begin{frame}
\frametitle{\tifonttxt{For(} probeer het zelf!}
\framesubtitle{Een paar antwoorden}

\begin{enumerate}
  \item \tifonttxt{Disp} de getallen $1,0,-1$ (in die volgorde!)
  \item \tifonttxt{Disp} de getallen $1,2,3,4,5,11,12,13,14,15$
  \item \tifonttxt{Disp} alle positieve oneven getallen tot en met 11.
\end{enumerate}

\hspace{-0.7cm}
\begin{minipage}{0.25\textwidth}
\begin{ticalc}
	PROGRAM\:FORMIN1\\%
	\:For(I\comma 1\comma -1\comma -1)\\%
	\:Disp\,I\\%
	\:End
\end{ticalc}\\
\begin{ticalc}
	PROGRAM\:FORODD\\%
	\:For(I\comma 1\comma 11\comma 2)\\%
	\:Disp\,I\\%
	\:End
\end{ticalc}
\end{minipage}
\hspace{0.7cm}
\begin{minipage}{0.25\textwidth}
\begin{ticalc}
	PROGRAM\:FORDUBLE\\%
	\:For(I\comma 1\comma 5)\\%
	\:Disp\,I\\%
	\:End\\%
	\:For(I\comma 11\comma 15)\\%
	\:Disp\,I\\%
	\:End
\end{ticalc}
\end{minipage}
\hspace{0.7cm}
\begin{minipage}{0.33\textwidth}
\flushleft{Of als de volgorde\\ niet uitmaakt:}
\begin{ticalc}
	PROGRAM\:FORDBLE2\\%
	\:For(I\comma 1\comma 5)\\%
	\:Disp\,I\\%
	\:Disp\,I+10\\%
	\:End
\end{ticalc}
\end{minipage}


\end{frame}







