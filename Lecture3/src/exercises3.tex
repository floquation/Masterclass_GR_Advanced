\section{Exercises}
\subsection{Exercises}


\begin{frame}
\frametitle{\tifonttxt{WIEWINT3}: Wie worden er tweede en derde?}

Breid \tifonttxt{WIEWINT2} uit door ook te weergeven wie tweede en derde zijn geworden.

\underline{Tip}: Er bestaat geen ``max2'' functie of iets dergelijks voor het vinden van het op-\'e\'en-na maximum,
dus je zult iets anders moeten bedenken met behulp van \tifonttxt{max}, of zelfs iets totaal anders.

\begin{ticalc}[3.5cm]
	PROGRAM\:WIEWINT2\\%
	\:Prompt\,L\sub{1}\\%
	\:max(L\sub{1})\>M\\%
	\:Disp\,\qt WINNERS\:\qt\\%
	\:For(I\comma1\comma dim(L\sub{1}))\\%
	\:If\,M=L\sub{1}(I)\:Then\\%
	\:Disp\,I\\%
	\:End\\%
	\:End
\end{ticalc}

\end{frame}




\begin{frame}
\frametitle{Exercises}

\begin{enumerate}
  \item Maak \inlineticalc{FORDISP} na, maar gebruik `\tifonttxt{While}' i.p.v. `\tifonttxt{For}'!
	\begin{itemize}
	  \item Zie je de gelijkenis? Beide statements zorgen voor herhaling.
	\end{itemize}
  \item Zelfde vraag, maar gebruik nu \tifonttxt{Goto} en \tifonttxt{Lbl}.
	\begin{itemize}
	  \item Dat kost veel moeite\ldots Gelukkig bestaan \tifonttxt{For} en \tifonttxt{While}!
	\end{itemize}
  \item Vind hoe vaak 2 in \inlineticalc{Prompt\,X} past. 
	\begin{itemize}
	  \item Bijv.: in het getal $40$ past `$2$' $3$x.
	\end{itemize}
  \item Vind alle priemgetallen kleiner dan \inlineticalc{Prompt\,X}.
	\begin{itemize}
	  \item Een priemgetal is alleen (integer) deelbaar door $1$ en zichzelf. ($2,3,5,7,11,13,17,19,23,\ldots$)
	\end{itemize}
  \item Ontbind het getal \inlineticalc{Prompt\,X}\,in priemgetallen. {\tiny{(Prime Factorisation)}}
	\begin{itemize}
	  \item Bijv.: $84=2^2\cdot3\cdot7$.
	  \tiny{\item Zie ook: http://www.2dtx.com/prime/prime84.html}
	\end{itemize}
  \item \lenitem[0.7\linewidth]{Schrijf het `risk dobbelstenen' programma van de lecture over pseudocode m.b.v. Lists.}
\end{enumerate}
MAAK ALTIJD EERST PSEUDOCODE!

\begin{tikzpicture}[overlay,remember picture]
	\node[yshift=0.6cm] (BL) at (current page.south east){ };
	\node [anchor=south east,xshift=0.195cm] (BL2) at (BL)
	{%
	\begin{ticalc}
		PROGRAM\:FORDISP\\%
		\:For(I\comma1\comma3\comma1)\\%
		\:Disp \qt FOR\qt\\%
		\:End
	\end{ticalc}
	};
\end{tikzpicture}



\end{frame}



\subsection{Answers}


\begin{frame}
\frametitle{\tifonttxt{WIEWINT3}: Wie worden er tweede en derde?}
\framesubtitle{Een mogelijk antwoord (er zijn zoals altijd meer manieren!)}

\vspace{-0.5cm}
\hspace{-1cm}
\begin{minipage}{0.83\textwidth}
	\begin{algorithm}[H]
	\caption{``WieWint3? met Lists''}
	\begin{algorithmic}[1]
	\Function{WieWint}{$\mathrm{L}_{1}$: list met scores}
	  \State Init. $\mathrm{L}_{2}$ als $\mathrm{L}_{1}$: iedereen doet mee voor rank 1
	  \For{rank 1, 2, 3}
		  \State Vindt punten voor huidige rank (uit $\mathrm{L}_{2}$): M
		  \State Clear $\mathrm{L}_{2}$ voor hergebruik
		  \For{alle spelers uit $\mathrm{L}_{1}$}
		  \If{punten = M} Deze speler wint!
		  \ElsIf{punten < M}
		  	\State Deze speler is wellicht een rank lager!
		  	\State Voeg hem toe aan $\mathrm{L}_{2}$
		  \EndIf
		  \EndFor \, Pause
	  \EndFor
	\EndFunction
	\end{algorithmic}
	\end{algorithm}
\end{minipage}
\begin{minipage}{0.15\textwidth}%
\begin{ticalc}[3.25cm]	
	PROGRAM\:WIEWINT3\\%
	\:Prompt\,L\sub{1}\\%
	\:L\sub{1}\>L\sub{2}\\%
	\:For(J\comma1\comma3)\\%
	\:max(L\sub{2})\>M\\%
	\:\{min(L\sub{1}\>L\sub{2}\\%
	\:Disp\,\qt RANK\:\qt\comma J\\%
	\:Disp\,\qt PEOPLE\:\qt\\%
	\:For(I\comma1\comma dim(L\sub{1}))\\%
	\:If\,M=L\sub{1}(I)\:Then\\%
	\:Disp\,I\\%
	\:Else\\%
	\:If\,M>L\sub{1}(I)\:Then\\%
	\:augment(L\sub{2}\comma\{L\sub{1}(I )\})\>L\sub{2}\\%
	\:End
	\:End
	\:End\\%
	\:Pause
	\:End
\end{ticalc}

\begin{minipage}{2.2\linewidth}
\scriptsize{(augment = samenvoegen)}
\end{minipage}
\end{minipage}

\end{frame}




\begin{frame}
\frametitle{Antwoord: \tifonttxt{FORDISP}}

\begin{minipage}{0.68\textwidth}
\begin{itemize}
  \item Deze opdracht is een mooi voorbeeld voor het principe ``the right tool for the job''.
  \item Merk op dat ``vroeger'' mensen geen \tifonttxt{For} en \tifonttxt{While} hadden. De computer begreep alleen \tifonttxt{Goto}. Lucky you!
  \item Het gebruik van \tifonttxt{Goto} op onderstaande manier is ``bad practice'' (don't do it).
  \begin{itemize}
    \item ``Using a Goto to exit any block of code requiring an End command causes a memory leak,
    	which will not be usable until the program finishes running or executes a Return command, and which will slow your program down.''
    \item \tiny{Zie ook: http://tibasicdev.wikidot.com/goto}
  \end{itemize}
\end{itemize}
\end{minipage}
\begin{minipage}{0.3\textwidth}
\begin{ticalc}
	PROGRAM\:FORDISP\\%
	\:For(I\comma1\comma3\comma1)\\%
	\:Disp \qt FOR\qt\\%
	\:End\:Pause\\%
	\:\\%
	\:1\>I\\%
	\:While\,I\le3\\%
	\:Disp\,\qt WHILE\qt\\%
	\:I+1\>I\\%
	\:End\:Pause\\%
	\:\\%
	\:1\>I\\%
	\:Lbl\,A\\%
	\:Disp\,\qt GOTO\qt\\%
	\:I+1\>I\\%
	\:If\,I\le3\:Then\\%
	\:Goto\,A\\%
	\:End	
\end{ticalc}
\end{minipage}

\end{frame}



\begin{frame}
\frametitle{Antwoord: Hoe vaak past $2$ in \tifonttxt{X}?}

\vspace{-0.5cm}
\hspace{-1cm}
\begin{minipage}{0.8\textwidth}
	\begin{algorithm}[H]
	\caption{``Hoe vaak past 2 in $X$?''}
	\begin{algorithmic}[1]
	\Function{HoeVaak2}{$X$}
	  \State Doe wat sanity-checks \& initialiseer
	  \While{X is een geheel getal ($\equiv$integer)}
	    \State Vervang $X$ door $X/2$.
	    \State Houd een tellertje bij, \tifonttxt{I}.
	  \EndWhile
	\EndFunction
	\end{algorithmic}
	\end{algorithm}
\end{minipage}
\begin{minipage}{0.15\textwidth}%
\begin{ticalc}
	PROGRAM\:HOEVAAK2\\%
	\:Prompt\,X\\%
	\:If\,fPart(X)\!0\,or\,X=0
	\:Then\\%
	\:Disp\,0\\%
	\:Stop\\%
	\:End\\%
	\:\\%
	\:\min 1\>I\\%
	\:While\,fPart(X)= 0\\%
	\:X/2\>X\\%
	\:I+1\>I\\%
	\:End\\%
	\:Disp\,I
\end{ticalc}
\end{minipage}

\end{frame}




\begin{frame}
\frametitle{Antwoord: Vind alle priemgetallen \tifonttxt{\le X}}

\vspace{-0.5cm}
\hspace{-1cm}
\begin{minipage}{0.83\textwidth}
	\begin{algorithm}[H]
	\caption{``Vind alle priemgetallen $\le X$.''}
	\begin{algorithmic}[1]
	\Function{Primes}{$X$: max. waarde om te checken}
	  \State Init. een list, $\mathrm{L}_1$, om de primes bij te houden
	  \For{\tifonttxt{I}: Alle integers $\le X$}
	    \State Neem \`a priori aan dat \tifonttxt{I} prime is.
	    \For{\tifonttxt{J}: Alle primes kleiner dan \tifonttxt{I}}
	      \If{\tifonttxt{I} deelbaar is door $\mathrm{L}_1(J)$}
	        \State \tifonttxt{I} is \underline{geen} prime. Break loop.
	      \EndIf
	    \EndFor
	    \State Indien \tifonttxt{I} niet `geen prime' is, voeg toe aan $\mathrm{L}_1$
	  \EndFor
	\EndFunction
	\end{algorithmic}
	\end{algorithm}
\end{minipage}
\begin{minipage}{0.15\textwidth}%
\begin{ticalc}
	PROGRAM\:PRIMES\\%
	\:Prompt\,X\\%
	\:\{2\}\>L\sub{1}\\%
	\:For(I\comma3\comma X)\\%
	\:1\>P\\%
	\:For(J\comma1\comma dim(L\sub{1}))\\%
	\:If\,fPart(I/L\sub{1}(J ))=0\:Then\\%
	\:0\>P\\%
	\:dim(L\sub{1})+1\>J\\%
	\:End\\%
	\:End\\%
	\:If\,P\:Then\\%
	\:augment(L\sub{1},\{I\}) \>L\sub{1}\\%
	\:End\:End\\%
	\:Disp\,L\sub{1}
\end{ticalc}
\end{minipage}

\end{frame}



\begin{frame}
\frametitle{Antwoord: Prime Factorisation van \tifonttxt{X}}


\vspace{-0.5cm}
\hspace{-1cm}
\begin{minipage}{0.82\textwidth}
	\begin{algorithm}[H]
	\caption{Pseudocode Prime Factorisation}
	\begin{algorithmic}[1]
	\Function{PrimeFac}{$X$: het te factoriseren getal}
	  \For{\tifonttxt{P}: Alle primes $<\sqrt{X}$ \textbf{OF} Alle natuurlijke getallen $<\sqrt{X}$} \Comment{\small{Beide opties geven hetzelfde resultaat.
	  												Het eerste is sneller voor grote $X$, het tweede voor kleine $X$. De tweede optie is makkelijker.}}
	    \If{$X$ deelbaar door \tifonttxt{P}}
	      \State Store (=Onthoud/Bewaar) \tifonttxt{P}
	      \State Ga verder met \tifonttxt{X/P}
	      \State Herhaal de loop voor \tifonttxt{P}
	    \EndIf
	  \EndFor
	\EndFunction
	\end{algorithmic}
	\end{algorithm}
\end{minipage}
\begin{minipage}{0.15\textwidth}%
\begin{ticalc}
	PROGRAM\:PRIMEFAC\\%
	\:Prompt\,X\\%
	\:If\,X=0\:Then\\%
	\:\{0\}\>L\sub{1}
	\:Disp\,L\sub{1}\\%
	\:Stop
	\:End\\%
	\:X/abs(X)\>S\\%
	\:XS\>Y\\%
	\:\\%
	\:\{S\}\>L\sub{1}\\%
	\:For(P\comma2\comma int(\sqrt(Y) ))\\%
	\:Y/P\>Z\\%
	\:If\,fPart(Z)=0\\%
	\:Then\\%
	\:augment(L\sub{1}\comma\{P\}) \>L\sub{1}\\%
	\:Z\>Y
	\:P-1\>P\\%
	\:End\:End\\%
	\:Disp\,L\sub{1}
\end{ticalc}
\end{minipage}

\end{frame}



\begin{frame}
\frametitle{Antwoord: Risk Dobbelstenen I}

\begin{algorithm}[H]
\caption{Pseudocode Risk Dobbelstenen}
\begin{algorithmic}[1]
% Risk dobbelstenen
\Function{RiskDice}{$N_{att}$}
	\State Maak $N_{att}$ random getallen tussen $1$ en $6$
	\State Vraag of $N_{def}$ $1$ of $2$ is
	\State Maak $N_{def}$ random getallen tussen $1$ en $6$
	\State Sorteer beide setten van hoog naar laag
	\If{Getal 1 van set 1 > Getal 1 van set 2}  $p_1=p_1+1$
	\Else \hspace{0.4cm} $p_2=p_2+1$
	\EndIf
	\If{Getal 2 van set 1 > Getal 2 van set 2}  $p_1=p_1+1$
	\Else \hspace{0.4cm} $p_2=p_2+1$
	\EndIf
	\State Display de scores: $p_1$ en $p_2$
\EndFunction
\end{algorithmic}
\end{algorithm}

\end{frame}



\begin{frame}
\frametitle{Antwoord: Risk Dobbelstenen II}

\vspace{-0.5cm}
\hspace{-1cm}
\begin{minipage}{0.5\textwidth}
\begin{ticalc}[4cm]
	PROGRAM\:RISKDICE\\%
	\:0\>A\:0\>D\\%
	\:While\,A<1\,or\,A>3\\%
	\:Input\,\qt NATT?\,\qt\comma A\\%
	\:End\\%
	\:seq(randInt(1\comma6)\comma X \comma1\comma A\comma1)\>L\sub{1}\\%
	\:SortD(L\sub{1})\\%
	\:Disp\,L\sub{1}\\%
	\:\\%
	\:While\,D<1\,or\,D>A\,or\,D>2\\%
	\:Input\,\qt NDEF?\,\qt\comma D\\%
	\:End\\%
	\:seq(randInt(1\comma6)\comma X \comma1\comma D\comma1)\>L\sub{2}\\%
	\:SortD(L\sub{2})\\%
	\:Disp\,L\sub{2}\\%
	\:
\end{ticalc}
\end{minipage}
\begin{minipage}{0.5\textwidth}%
\begin{ticalc}[4.2cm]
	\:0\>P\:0\>Q\\%
	\:If\,L\sub{1}(1)>L\sub{2}(1)\\%
	\:Then\:P+1\>P\\%
	\:Else\:Q+1\>Q\\%
	\:End\\%
	\:If\,D>1\:Then\\%
	\:If\,L\sub{1}(2)>L\sub{2}(2)\\%
	\:Then\:P+1\>P\\%
	\:Else\:Q+1\>Q\\%
	\:End\:End\\%
	\:\\%
	\:ClrHome\\%
	\:Disp\,L\sub{1}\comma L\sub{2}\\%
	\:Disp\,\qt ATT\,LOSES\:\qt\comma Q\\%
	\:Disp\,\qt DEF\,LOSES\:\qt\comma P	
\end{ticalc}
\end{minipage}

\end{frame}
