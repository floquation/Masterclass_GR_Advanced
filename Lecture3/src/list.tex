\section{Datatype: Lists}

\subsection{List datatype}

\begin{frame}
\frametitle{Een nieuw datatype: List}

\begin{itemize}
  \item<1-> Wat is beter dan \'e\'en getal?
  \begin{itemize}
    \item<2-> \only<2->{MEER GETALLEN!}
  \end{itemize}
  \item<3-> Een list (NL: lijst) kan meerdere getallen in dezelfde variabele storen.
  \item<4-> Dit maakt het makkelijk om over een lijst getallen te itereren (=loopen).
\end{itemize}

\end{frame}




\begin{frame}
\frametitle{Een nieuw datatype: List}

\visible<1>{\ticalcfig{
	\ticalcfigCircleSecond
	\ticalcfigCircle[\ticalcfigCircleNumButtonsize]{\ticalcfigCircleColTwo}{0.161}% 1
	\ticalcfigCircle[\ticalcfigCircleNumButtonsize]{\ticalcfigCircleColThree}{0.161}% 2
	\ticalcfigCircle[\ticalcfigCircleNumButtonsize]{\ticalcfigCircleColFour}{0.161}% 3
	\ticalcfigCircle[\ticalcfigCircleNumButtonsize]{\ticalcfigCircleColTwo}{0.27}% 4
	\ticalcfigCircle[\ticalcfigCircleNumButtonsize]{\ticalcfigCircleColThree}{0.27}% 5
	\ticalcfigCircle[\ticalcfigCircleNumButtonsize]{\ticalcfigCircleColFour}{0.27}% 6
}} % List variables + 2nd
\visible<2>{\ticalcfig{
	\ticalcfigCircleSecond
	\ticalcfigCircle[\ticalcfigCircleNumButtonsize]{\ticalcfigCircleColTwo}{0.46}% , comma
	\ticalcfigCircle[\ticalcfigCircleNumButtonsize]{\ticalcfigCircleColThree}{0.46}% { accolade open
	\ticalcfigCircle[\ticalcfigCircleNumButtonsize]{\ticalcfigCircleColFour}{0.46}% } accolade close
}} % Accolades { } and comma ,
\visible<3>{\ticalcfig{
	\ticalcfigCircleSecond
	\ticalcfigCircle[\ticalcfigCircleNumButtonsize]{\ticalcfigCircleColTwo}{0.161}% 1
	\ticalcfigCircle[\ticalcfigCircleNumButtonsize]{\ticalcfigCircleColFour}{0.161}% 3
	\ticalcfigCircle[\ticalcfigCircleNumButtonsize]{\ticalcfigCircleColThree}{0.46}% ( bracket open
	\ticalcfigCircle[\ticalcfigCircleNumButtonsize]{\ticalcfigCircleColFour}{0.46}% ) bracket close
}} % L1(3)


\begin{itemize}
  \item<1-> \lenitem{Een lijst store je in de \tifonttxt{L\sub{1}} to \tifonttxt{L\sub{6}} variables.}
  \item<2-> \lenitem{Op het scherm van de rekenmachine definieer je een list tussen accolades (\tifonttxt{\{\}}). Probeer het zelf!}
  \item<3-> \lenitem{Indien je het derde getal uit list \tifonttxt{L\sub{1}} wilt hebben, dan type je \tifonttxt{L\sub{1}(3)}.}
\end{itemize}


\begin{tikzpicture}[overlay,remember picture]
	\node[yshift=0.6cm] (BL) at (current page.south east){ };
	\node [anchor=south east,xshift=0.195cm] (BL2) at (BL)
	{%
	\only<2->{%
		\begin{ticalc}
			\{5\comma3\comma2\}\>L\sub{1}\\%
			\hfill\{5\comma3\comma2\}\\%
			L\sub{1}\\%
			\hfill\{5\comma3\comma2\}\\%
			\visible<3->{L\sub{1}(3)}\\%
			\visible<3->{\hfill 2}%
		\end{ticalc}
	}%
	};
\end{tikzpicture}

\end{frame}



\subsection{List hulpjes}


\begin{frame}
\frametitle{Nuttige hulpjes voor Lists}

\visible<1->{\ticalcfig{
	\ticalcfigCircleSecond
	\ticalcfigCircle[\ticalcfigCircleNumButtonsize]{\ticalcfigCircleColThree}{0.725}% List
}} % 2nd Stat = List

\begin{itemize}
  \item<1-> \lenitem{Onder \tiSecond\tiSTAT=\tiLIST\,staan nuttige hulpjes voor Lists}
  \item<2-> \lenitem{Onder \tifonttxt{NAMES} staan variabelen die je als List kunt gebruiken.
  	Dit zijn de vertrouwde \tifonttxt{L\sub{1}} to \tifonttxt{L\sub{6}}, maar dan nog veel meer mogelijkheden.}
  \item<3-> \lenitem{Onder \tifonttxt{OPS} staan list-operaties, zoals sorteren (\tifonttxt{SortA(}),
    of het aantal elementen in een lijst opvragen (\tifonttxt{dim(}).}
  \item<4-> \lenitem[0.7\linewidth]{Onder \tifonttxt{MATH} staan wiskundige operaties,
    zoals het berekenen van het gemiddelde (\tifonttxt{mean(}).}
\end{itemize}

\vspace{2cm}

\begin{tikzpicture}[overlay,remember picture]
	\node[yshift=0.6cm] (BL) at (current page.south east){ };
	\node [anchor=south east,xshift=0.195cm] (BL2) at (BL)
	{%
	\only<2>{%
		\begin{ticalc}
			\select{NAMES}\,OPS\,MATH\\%
			\selectitem{1\+\:}L\sub{1}\\%
			2\:L\sub{2}\\%
			3\:L\sub{3}\\%
			4\:L\sub{4}\\%
			5\:L\sub{5}\\%
			6\:L\sub{6}\\%
			7\arrowdown A
		\end{ticalc}
	}%
	\only<3>{%
		\begin{ticalc}
			NAMES\,\select{OPS}\,MATH\\%
			\selectitem{1\+\:}SortA(\\%
			2\:SortD(\\%
			\selectitem{3\:}dim(\\%
			4\:Fill(\\%
			5\:seq(\\%
			6\:cumSum(\\%
			7\arrowdown \Delta List(
		\end{ticalc}
	}%
	\only<4>{%
		\begin{ticalc}
			NAMES\,OPS\,\select{MATH}\\%
			1\+\:min(\\%
			2\:max(\\%
			\selectitem{3\:}mean(\\%
			4\:median(\\%
			5\:sum(\\%
			6\:prod(\\%
			7\arrowdown stdDev(
		\end{ticalc}
	}%
	};
\end{tikzpicture}

\end{frame}



\begin{frame}
\frametitle{Nuttige hulpjes voor Lists}

\visible<1->{\ticalcfig{
	\ticalcfigCircleSecond
	\ticalcfigCircle[\ticalcfigCircleNumButtonsize]{\ticalcfigCircleColThree}{0.725}% List
}} % 2nd Stat = List

Laten we wat operaties uit proberen:
\begin{itemize}
  \item<2-> Wat doet \tifonttxt{SortA(}? \visible<3->{Sorteer oplopend (ascending)!}
  \item<4-> Wat doet \tifonttxt{SortD(}? \visible<5->{Sorteer aflopend (descending)!}
  \item<6-> Wat doet \tifonttxt{max(}? \visible<7->{Maximale waarde!}
  \item<8-> \lenitem{Wat doet \tifonttxt{prod(}? \visible<9->{Vermenigvuldigt
    alle getallen tussen index \tifonttxt{1} en \tifonttxt{3}!}}
  \item<10-> Wat doet \tifonttxt{seq(}? \visible<11->{Het vult een list m.b.v. een formule!}
  \item<12-> \lenitem[0.6\linewidth]{Wat doet \tifonttxt{\Delta List}? \visible<13->{Het geeft een list die \'e\'en korter
    is en het verschil tussen opvolgende elementen heeft.}}
  \item<14-> \lenitem[0.6\linewidth]{Dit is allemaal een stuk krachtiger dan rekenen met een enkel getal!}
\end{itemize}


\begin{tikzpicture}[overlay,remember picture]
	\node[yshift=0.6cm] (BL) at (current page.south east){ };
	\node [anchor=south east,xshift=0.195cm] (BL2) at (BL)
	{%
	\only<1-5>{%
		\begin{ticalc}
			\visible<1->{\{5\comma2\comma3\comma0\}\>L\sub{1}}\\%
			\visible<1->{\hfill\{5\,2\,3\,0\}}\\%
			\visible<2->{SortA(L\sub{1})}\\%
			\visible<2->{\hfill Done}\\%
			\visible<3->{L\sub{1}}\\%
			\visible<3->{\hfill\{0\,2\,3\,5\}}\\%
			\visible<4->{SortD(L\sub{1})}\\%
			\visible<4->{\hfill Done}\\%
			\visible<5->{L\sub{1}}\\%
			\visible<5->{\hfill\{5\,3\,2\,0\}}
		\end{ticalc}
	}%
	\only<6-9>{%
		\begin{ticalc}
			\visible<6->{\{5\comma2\comma3\comma0\}\>L\sub{1}}\\%
			\visible<6->{\hfill\{5\,2\,3\,0\}}\\%
			\visible<6->{max(L\sub{1})}\\%
			\visible<7->{\hfill 5}\\%
			\visible<8->{prod(L\sub{1}\comma1\comma3)}\\%
			\visible<9->{\hfill 30}
		\end{ticalc}
	}%
	\only<10->{%
		\begin{ticalc}[3.5cm]
			\visible<10->{seq(X\sq\comma X\comma0\comma5\comma1)\>L\sub{1}}\\%
			\visible<11->{\hfill\{0\,1\,4\,9\,16\,25\}}\\%
			\visible<10->{\Delta List(L\sub{1})}\\%
			\visible<13->{\hfill\{1\,3\,5\,7\,9\}}
		\end{ticalc}
	}%
	};
\end{tikzpicture}

\end{frame}



\begin{frame}
\frametitle{\tifonttxt{WIEWINT} met Lists}

\begin{minipage}{0.48\textwidth}
Kun je je dit programma nog herinneren? Nasty\ldots

\vspace{0.2cm}
\begin{ticalc}
	PROGRAM\:WIEWINT\\%
	\:Prompt\,A\\%
	\:Prompt\,B\\%
	\:Prompt\,C\\%
	\:Prompt\,D\\%
	\:\\%
	\:If\,A>B\:Then\\%
		\:If\,A>C\:Then\\% A or D win
			\:If\,A>D\:Then\\%
				\:Disp\,\qt A\,WINS\qt\\%
			\:Else\\%
				\:Disp\,\qt D\,WINS\qt\\%
			\:End\\%
		\:Else\\% C or D win
			\:If\,C>D\:Then\\%
				\:Disp\,\qt C\,WINS\qt\\%
\end{ticalc}
\end{minipage}
~
\begin{minipage}{0.48\textwidth}
\begin{ticalc}	
			\:Else\\%
				\:Disp\,\qt D\,WINS\qt\\%
			\:End\\%		
		\:End\\%
	\:Else\\% A does not win
		\:If\,B>C\:Then\\% B or D win
			\:If\,B>D\:Then\\%
				\:Disp\,\qt B\,WINS\qt\\%
			\:Else\\%
				\:Disp\,\qt D\,WINS\qt\\%
			\:End\\%
		\:Else\\% C or D win
			\:If\,C>D\:Then\\%
				\:Disp\,\qt C\,WINS\qt\\%
			\:Else\\%
				\:Disp\,\qt D\,WINS\qt\\%
			\:End\\%
		\:End\\%
	\:End
\end{ticalc}
\end{minipage}

\end{frame}





\begin{frame}
\frametitle{\tifonttxt{WIEWINT} met Lists}

\begin{minipage}{0.7\textwidth}%
	\begin{algorithm}[H]
	\caption{``WieWint? met Lists''}
	\begin{algorithmic}[1]
	\Function{WieWint}{list met scores}
	  \visible<2->{\State Vindt punten van de winnaar: M}
	  \visible<3->{\For{alle spelers}
	  \visible<4->{\If{punten = M}
	    \visible<5->{\State Deze speler wint!}
	  \EndIf}
	  \EndFor}
	\EndFunction
	\end{algorithmic}
	\end{algorithm}
\end{minipage}%
\begin{minipage}{0.27\textwidth}%
\visible<6->{%
\begin{ticalc}[3.5cm]
	PROGRAM\:WIEWINT2\\%
	\:Prompt\,L\sub{1}\\%
	\:max(L\sub{1})\>M\\%
	\:Disp\,\qt WINNERS\:\qt\\%
	\:For(I\comma1\comma dim(L\sub{1}))\\%
	\:If\,M=L\sub{1}(I)\:Then\\%
	\:Disp\,I\\%
	\:End\\%
	\:End
\end{ticalc}
}

\visible<8->{%
\begin{ticalc}
	prgmWIEWINT2\\%
	L\sub{1}=?\{5\comma3\comma5\comma2\\%
	WINNERS\:\\%
	\hfill 1\\%
	\hfill 3\\%
	\hfill Done
\end{ticalc}
}
\end{minipage}%

\visible<7->{Van 33 regels voor 4 spelers, naar slechts 8 regels voor een \textbf{onbeperkt} aantal spelers!
Kun je de kracht van loops en lists al waarderen?}


\addtocounter{algorithm}{-1} % Prevents the algorithm number to increase every frame
\end{frame}
\addtocounter{algorithm}{1} % Make sure the number increases for a new algorithm on a different slide




\subsection{Matrix datatype}

\begin{frame}
\frametitle{Sidenote: Matrix datatype}

\visible<1->{\ticalcfig{
	\ticalcfigCircleSecond
	\ticalcfigCircle[\ticalcfigCircleNumButtonsize]{\ticalcfigCircleColOne}{0.56}% 1
}} % Matrix

\begin{itemize}
  \item<1-> Een List is een \'e\'en-dimensionale lijst van getallen
  \item<2-> Zonder in details te treden, een Matrix is een twee-dimensionale lijst van getallen.
  \item<3-> Matrices vind je onder \tiSecond\tiXInv=\tiMATRIX
  \item<4-> Dit kan nuttig zijn voor je data opslag.
\end{itemize}

\end{frame}
