%% LaTeX Beamer presentation template (requires beamer package)
%% see http://bitbucket.org/rivanvx/beamer/wiki/Home
%% idea contributed by H. Turgut Uyar
%% template based on a template by Till Tantau
%% this template is still evolving - it might differ in future releases!

\documentclass{beamer}

\mode<presentation>
{
\usetheme{Warsaw}

\setbeamercovered{transparent}
}

\usepackage{../../Common/sty/TI84} %TODO: How to get the link working...?



\title{Masterclass programmeren op de GR TI-84 (les 3)}

%\subtitle{}

% - Use the \inst{?} command only if the authors have different
%   affiliation.
%\author{F.~Author\inst{1} \and S.~Another\inst{2}}
\author{Kevin van As}

% - Use the \inst command only if there are several affiliations.
% - Keep it simple, no one is interested in your street address.
% \institute[Universities of]
% {
% \inst{1}%
% Department of Computer Science\\
% Univ of S
% \and
% \inst{2}%
% Department of Theoretical Philosophy\\
% Univ of E}

\date{\today}


% This is only inserted into the PDF information catalog. Can be left
% out.
\subject{Masterclass GR TI-84 programmeren (les 3)}



% If you have a file called "university-logo-filename.xxx", where xxx
% is a graphic format that can be processed by latex or pdflatex,
% resp., then you can add a logo as follows:

% \pgfdeclareimage[height=0.5cm]{university-logo}{university-logo-filename}
% \logo{\pgfuseimage{university-logo}}



% Delete this, if you do not want the table of contents to pop up at
% the beginning of each subsection:
\AtBeginSubsection[]
{
\begin{frame}<beamer>
\frametitle{Outline}
\tableofcontents[currentsection,currentsubsection]
\end{frame}
}

% If you wish to uncover everything in a step-wise fashion, uncomment
% the following command:

%\beamerdefaultoverlayspecification{<+->}

\begin{document}

\begin{frame}
\titlepage
\end{frame}


\begin{frame}
\frametitle{Recap!}

% \visible<2>{\ticalcfig{\ticalcfigCircle{\ticalcfigCircleColThree}{0.615}}} % prgm
% \visible<3-4>{\ticalcfig{\ticalcfigCircle{\ticalcfigCircleColFour}{0.62}}} % Vars
% \visible<5>{\ticalcfig{\ticalcfigCircle{\ticalcfigCircleColThree}{0.615}}} % prgm

We hebben gekeken naar:
\begin{itemize}
	\item<2-> Datatype: Boolean
	\item<3-> Controle middels condities: \tifonttxt{If}
	\item<4-> Pseudocode \& Algoritmes
\end{itemize}
\end{frame}

\begin{frame}
\frametitle{Vooruitzicht!}

Vandaag zullen we kijken naar:
\begin{itemize}
	\item<2-> Herhaling
		\begin{itemize}
		  \item<3-> Goto \& Labels
		  \item<4-> For-loops
		  \item<5-> While-loops
		\end{itemize}
	\item<6-> Pause
	\item<7-> Datatype: Lists
\end{itemize}

\end{frame}


%% END %%

\begin{frame}
\frametitle{Outline}
\tableofcontents
% You might wish to add the option [pausesections]
\end{frame}

% The core pages
\section{Goto \& Labels}

\begin{frame}
\frametitle{R-r-r-r-epeat: Herhaling}

\begin{itemize}
  \item<1-> In een programma komt het vaak voor dat je een opdracht wilt herhalen.
  \item<2-> Kijk bijvoorbeeld naar \inlineticalc{PROGRAM\:WIEWINT} van de exercises van vorige les.
  \item<3-> Soms wil je een opdracht een gegeven aantal keer herhalen.
  	\visible<4->{Dit is heel vaak hetzelfde programmeren\ldots Stel je voor dat je iets 132x wilt doen!}
  \item<5-> Soms wil je een opdracht een van-te-voren onbekend aantal keer herhalen\ldots
  	\visible<6->{Dat is niet te programmeren\ldots}
  \item<7-> Voor dit soort gevallen, bestaan er opdrachten die iets herhalen:
  	\tifonttxt{For}, \tifonttxt{While} en \tifonttxt{Goto} met \tifonttxt{Lbl}.
\end{itemize}

\end{frame}



\begin{frame}
\frametitle{Het \tifonttxt{Lbl} (Label) statement}

\begin{itemize}
  \item<1-> Het \tifonttxt{Lbl} statement doet \emph{niets} op zichzelf.
  \item<2-> Het markeert een locatie in het programma.
  \item<3-> Deze locatie kan elders in het programma gebruikt worden.
  \item<4-> \tifonttxt{Lbl} staat onder \tiPRGM\,in het \tifonttxt{CTL}-menu.
  \item<5-> Blader daarvoor naar beneden \tiDown.
\end{itemize}


\begin{tikzpicture}[overlay,remember picture]
	\node[] (BL) at (current page.south east){ };
	\node [anchor=south east,xshift=0.195cm,yshift=0.6cm] at (BL)
	{%
	\only<4>{%
		\begin{ticalc}
			\select{CTL}\,I/O\,EXEC \\
			\selectitem{1\+\:}If \\
			2\:Then \\
			3\:Else \\
			4\:For( \\
			5\:While \\
			6\:Repeat( \\
			7\arrowdown End
		\end{ticalc}
	}%
	\only<5>{%
		\begin{ticalc}
			\select{CTL}\,I/O\,EXEC \\
			4\arrowup For( \\
			5\:While \\
			6\:Repeat( \\
			7\:End \\
			8\:Pause \\
			\selectitem{9\:}Lbl \\
			0\arrowdown Goto
		\end{ticalc}
	}%
	};
\end{tikzpicture}

\end{frame}




\begin{frame}
\frametitle{Het \tifonttxt{Goto} (Ga naar) statement}

\begin{itemize}
  \item<1-> Het \tifonttxt{Goto} statement heeft \tifonttxt{Lbl} nodig.
  \item<2-> Wanneer een programma bij de \tifonttxt{Goto}-regel aan komt,
  	dan springt hij naar de bijbehorende \tifonttxt{Lbl} en gaat daar verder met het uitvoeren van het programma.
  \item<3-> Je kunt zo dus door het hele programma heen springen!
  \item<4-> \tifonttxt{Goto} staat onder \tiPRGM\,in het \tifonttxt{CTL}-menu.
  \item<5-> Blader daarvoor naar beneden \tiDown.
\end{itemize}


\begin{tikzpicture}[overlay,remember picture]
	\node[] (BL) at (current page.south east){ };
	\node [anchor=south east,xshift=0.195cm,yshift=0.6cm] at (BL)
	{%
	\only<4>{%
		\begin{ticalc}
			\select{CTL}\,I/O\,EXEC \\
			\selectitem{1\+\:}If \\
			2\:Then \\
			3\:Else \\
			4\:For( \\
			5\:While \\
			6\:Repeat( \\
			7\arrowdown End
		\end{ticalc}
	}%
	\only<5>{%
		\begin{ticalc}
			\select{CTL}\,I/O\,EXEC \\
			4\arrowup For( \\
			5\:While \\
			6\:Repeat( \\
			7\:End \\
			8\:Pause \\
			9\:Lbl \\
			\selectitem{0\arrowdown} Goto
		\end{ticalc}
	}%
	};
\end{tikzpicture}

\end{frame}





\begin{frame}
\frametitle{Het \tifonttxt{Goto} (Ga naar) statement}
\framesubtitle{Voorbeeld}

\begin{itemize}
  \item<1-> Voorbeeld: Het programma slaat hier \inlineticalc{\:Disp\,\qt DIT WORDT NOOIT WEERGEVEN\qt} over
  	en springt in \'e\'en keer naar \inlineticalc{\:Disp\,\qt DIT WORDT ALTIJD UITGEVOERD\qt}.
  	Het negeert dus \inlineticalc{Stop}: Dat wordt nooit uitgevoerd.
  \item<2-> Labels kunnen een naam hebben bestaande uit cijfers en letters, met maximaal 2 karakters.
	  \begin{itemize}
	    \item Bijvoorbeeld \tifonttxt{AB} of \tifonttxt{0A}, maar niet \tifonttxt{TUD2}.
	  \end{itemize}
\end{itemize}

\vspace{1.5cm}

\begin{tikzpicture}[overlay,remember picture]
	\node[] (BL) at (current page.south east){ };
	\node [anchor=south east,xshift=0.195cm,yshift=0.6cm] at (BL)
	{%
	\only<1>{%
		\begin{ticalc}
			PROGRAM\:GOTOLBL0\\%
			\:Goto\,0\\%
			\:Disp\,\qt DIT WORDT NOOIT WEERGEVEN\qt\\%
			\:Stop\\%
			\:Lbl\,0\\%
			\:Disp\,\qt DIT WORDT ALTIJD UITGEVOERD\qt
		\end{ticalc}
	}%
	\only<2>{%
		\begin{ticalc}
			PROGRAM\:GOTOLBL1\\%
			\:Goto\,TU\\%
			\:Disp\,\qt DIT WORDT NOOIT WEERGEVEN\qt\\%
			\:Stop\\%
			\:Lbl\,TU\\%
			\:Disp\,\qt DIT WORDT ALTIJD UITGEVOERD\qt
		\end{ticalc}
	}%
	};
\end{tikzpicture}

\end{frame}



\begin{frame}
\frametitle{Probeer \tifonttxt{Goto} zelf!}

Schrijf zelf een klein \tifonttxt{prgm} met alleen \tifonttxt{Disp}, \tifonttxt{Lbl} en \tifonttxt{Goto} statements,
om er een gevoel voor te krijgen.


\begin{tikzpicture}[overlay,remember picture]
	\node[] (BL) at (current page.south east){ };
	\node [anchor=south east,xshift=0.195cm,yshift=0.6cm] (BL2) at (BL)
	{%
		\begin{ticalc}
			\select{CTL}\,I/O\,EXEC \\
			4\arrowup For( \\
			5\:While \\
			6\:Repeat( \\
			7\:End \\
			8\:Pause \\
			9\:Lbl \\
			\selectitem{0\arrowdown} Goto
		\end{ticalc}
	};
	\node [anchor=east,xshift=-1.5cm] (BL3) at (BL2)
	{%
		\begin{ticalc}
			\select{CTL}\,I/O\,EXEC \\
			4\arrowup For( \\
			5\:While \\
			6\:Repeat( \\
			7\:End \\
			8\:Pause \\
			\selectitem{9\:}Lbl \\
			0\arrowdown Goto
		\end{ticalc}
	};
	\node [anchor=east,xshift=-1.5cm] (BL4) at (BL3)
	{%
		\begin{ticalc}
			CTL\,\select{I/O}\,EXEC \\
			1\+\:Input \\
			2\:Prompt \\
			\selectitem{3\:}Disp \\
			4\:DispGraph \\
			5\:DispTable \\
			6\:Output( \\
			7\arrowdown getKey
		\end{ticalc}
	};
\end{tikzpicture}

\end{frame}






\section{Herhaling statements}
\subsection{For-loops}



\begin{frame}
\frametitle{Wat is een \tifonttxt{For}-loop?}

\begin{itemize}
  \item<1-> Alhoewel \tifonttxt{Goto} en \tifonttxt{Lbl} gebruikt kunnen worden om statements te herhalen,
  	zijn er krachtigere methodes.
  \item<2-> Het \tifonttxt{For} statement herhaalt een stukje code `for (NL: voor)' bepaalde integers.
  \begin{itemize}
    \item<3-> Herinner je dat `integers' gehele getallen zijn
  \end{itemize}
  \item<4-> Hiermee kun je een gegeven aantal keren een stukje code herhalen.
  \item<5-> Bijvoorbeeld: ``Herhaal dit voor de integers 1 tot 5'' = ``Herhaal dit 5 keer''
\end{itemize}

\end{frame}





\begin{frame}
\frametitle{Wat is een \tifonttxt{For}-loop?}

\begin{itemize}
  \item<1-> Op de rekenmachine ziet dit er zo uit:
  \begin{itemize}
    \item<1-> ``Weergeef de integers 1 tot 5'' = ``Herhaal \tifonttxt{Disp} 5 keer''
  \end{itemize}
  \item<2-> Het is ook mogelijk getallen over te slaan:
  \begin{itemize}
    \item<2-> ``Weergeef de meervouden van 3 tot en met 9'' = ``Herhaal \tifonttxt{Disp} \only<-2>{<??>}\only<3->{3} keer''
  \end{itemize}
  \item<4-> De structuur is dus: \tifonttxt{For(}variable,start,end,increment\tifonttxt{)}
  \item<5-> \tifonttxt{For(} staat onder \tiPRGM\,in het \tifonttxt{CTL}-menu.
  \item<6-> De loop eindigt met de oude vertrouwde \tifonttxt{End}
\end{itemize}

\vspace{2cm}

\begin{tikzpicture}[overlay,remember picture]
	\node[] (BL) at (current page.south east){ };
	\node [anchor=south east,xshift=0.195cm,yshift=0.6cm] at (BL)
	{%
	\only<1,6>{%
		\begin{ticalc}
			PROGRAM\:FIRSTFOR\\%
			\:For(I\comma 1\comma 5)\\%
			\:Disp\,I\\%
			\:End
		\end{ticalc}
	}%
	\only<2-4>{%
		\begin{ticalc}
			PROGRAM\:FIRSTFOR\\%
			\:For(I\comma 3\comma 9\comma 3)\\%
			\:Disp\,I\\%
			\:End
		\end{ticalc}
	}%
	\only<5>{%
		\begin{ticalc}
			\select{CTL}\,I/O\,EXEC \\
			1\+\:If \\
			2\:Then \\
			3\:Else \\
			\selectitem{4\:}For( \\
			5\:While \\
			6\:Repeat( \\
			7\arrowdown End
		\end{ticalc}
	}%
	};
\end{tikzpicture}

\end{frame}


\begin{frame}
\frametitle{\tifonttxt{For(} probeer het zelf!}

Probeer de volgende \tifonttxt{Prgms} te schrijven:
\begin{enumerate}
  \item \tifonttxt{Disp} de getallen $1,2,3,4,5$
  \item \tifonttxt{Disp} de getallen $-1,0,1$
  \item \tifonttxt{Disp} de getallen $1,0,-1$ (in die volgorde!)
  \item \tifonttxt{Disp} de getallen $1,2,3,4,5,11,12,13,14,15$
  \item \tifonttxt{Disp} alle positieve even getallen tot en met 12.
  \item \tifonttxt{Disp} alle positieve oneven getallen tot en met 11.
\end{enumerate}

\vspace{2cm}

\begin{tikzpicture}[overlay,remember picture]
	\node[yshift=0.6cm] (BL) at (current page.south east){ };
	\node [anchor=south east,xshift=0.195cm] (BL2) at (BL)
	{%
		\begin{ticalc}
			\select{CTL}\,I/O\,EXEC \\
			1\+\:If \\
			2\:Then \\
			3\:Else \\
			\selectitem{4\:}For( \\
			5\:While \\
			6\:Repeat( \\
			7\arrowdown End
		\end{ticalc}
	};
	\node [anchor=south east,xshift=-3.5cm] (BL3) at (BL)
	{%
		\begin{ticalc}
			PROGRAM\:FIRSTFOR\\%
			\:For(I\comma 3\comma 9\comma 3)\\%
			\:Disp\,I\\%
			\:End
		\end{ticalc}
	};
\end{tikzpicture}


\end{frame}



\begin{frame}
\frametitle{\tifonttxt{For(} probeer het zelf!}
\framesubtitle{Een paar antwoorden}

\begin{enumerate}
  \item \tifonttxt{Disp} de getallen $1,0,-1$ (in die volgorde!)
  \item \tifonttxt{Disp} de getallen $1,2,3,4,5,11,12,13,14,15$
  \item \tifonttxt{Disp} alle positieve oneven getallen tot en met 11.
\end{enumerate}

\hspace{-0.7cm}
\begin{minipage}{0.25\textwidth}
\begin{ticalc}
	PROGRAM\:FORMIN1\\%
	\:For(I\comma 1\comma -1\comma -1)\\%
	\:Disp\,I\\%
	\:End
\end{ticalc}\\
\begin{ticalc}
	PROGRAM\:FORODD\\%
	\:For(I\comma 1\comma 11\comma 2)\\%
	\:Disp\,I\\%
	\:End
\end{ticalc}
\end{minipage}
\hspace{0.7cm}
\begin{minipage}{0.25\textwidth}
\begin{ticalc}
	PROGRAM\:FORDUBLE\\%
	\:For(I\comma 1\comma 5)\\%
	\:Disp\,I\\%
	\:End\\%
	\:For(I\comma 11\comma 15)\\%
	\:Disp\,I\\%
	\:End
\end{ticalc}
\end{minipage}
\hspace{0.7cm}
\begin{minipage}{0.33\textwidth}
\flushleft{Of als de volgorde\\ niet uitmaakt:}
\begin{ticalc}
	PROGRAM\:FORDBLE2\\%
	\:For(I\comma 1\comma 5)\\%
	\:Disp\,I\\%
	\:Disp\,I+10\\%
	\:End
\end{ticalc}
\end{minipage}


\end{frame}








\subsection{While-loops}


\begin{frame}
\frametitle{Wat is een \tifonttxt{While}-loop?}

\begin{itemize}
  \item<1-> \tifonttxt{For} is leuk indien je weet hoe vaak je iets wilt herhalen.
  \item<2-> \tifonttxt{While} is leuk indien je dat \`a priori juist \emph{niet} weet.
  \item<3-> Een \tifonttxt{While}-loop gaat door totdat een conditie (\tifonttxt{If}) onwaar wordt.
  \item<4-> Bijvoorbeeld:
  \begin{itemize}
    \item<5-> ``Herhaal dit totdat het regent''
    \item<6-> ``Herhaal dit totdat de berekening nauwkeurig genoeg is''
  \end{itemize}
\end{itemize}

\end{frame}





\begin{frame}
\frametitle{Wat is een \tifonttxt{While}-loop?}

\begin{itemize}
  \item<1-> Op de rekenmachine ziet dit er zo uit:
  \begin{itemize}
    \item<1-> ``Herhaal dit totdat het regent''
    \item<3-> \tifonttxt{int(rand*10)} genereert een willekeurige integer tussen \tifonttxt{0} en \tifonttxt{9}.
    	Dit geeft een $10\%$ kans dat de loop breaks (=stopt) elke keer dat hij execute (=herhaalt).
  \end{itemize}
  \item<4-> De structuur is dus: \tifonttxt{While\,}<<conditie>>
  \item<5-> \tifonttxt{While} staat onder \tiPRGM\,in het \tifonttxt{CTL}-menu.
  \item<6-> De loop eindigt met de oude vertrouwde \tifonttxt{End}
\end{itemize}

\vspace{2cm}

\begin{tikzpicture}[overlay,remember picture]
	\node[yshift=0.6cm] (BL) at (current page.south east){ };
	\node [anchor=south east,xshift=0.195cm] at (BL)
	{%
	\only<1-4,6->{%
		\begin{ticalc}
			PROGRAM\:FRSTWHLE\\%
			\:0\>R
			\:\only<4-6>{\select{While\,R\!1}}\only<1-3>{While\,R\!1}\\%
			\:int(rand*10)\>R\\%
			\:Disp\,\qt DROOG\qt\\%
			\:\only<6>{\select{End}}\only<1-5>{End}\\%
			\:Disp\,\qt REGEN\qt
		\end{ticalc}
	}%
	\only<5>{%
		\begin{ticalc}
			\select{CTL}\,I/O\,EXEC \\
			1\+\:If \\
			2\:Then \\
			3\:Else \\
			4\:For( \\
			\selectitem{5\:}While \\
			6\:Repeat( \\
			7\arrowdown End
		\end{ticalc}
	}%
	};
	\node [anchor=south east,xshift=-3.6cm] (BL3) at (BL)
	{%
	\only<2->{
		\begin{ticalc}
			DROOG\\%
			DROOG\\%
			DROOG\\%
			DROOG\\%
			REGEN\\%
			\hfill Done
		\end{ticalc}
	}
	};
\end{tikzpicture}

\end{frame}





\begin{frame}
\frametitle{\tifonttxt{While} zelf aan de slag \tifonttxt{End}}

\begin{enumerate}
  \item Maak \tifonttxt{FRSTWHLE} na. \tifonttxt{int(} en \tifonttxt{rand} staan onder \tiMATH.
  \item Voer dit \tifonttxt{prgm} een aantal keer uit. Zie je altijd hetzelfde?
  \item Hoe kun je het verwachtte aantal repetities veranderen?
  \item Breid het programma nu uit met de zon: Pas als de zon stopt met schijnen, dan kan het gaan regenen.
  \begin{itemize}
    \item \underline{Tip}: Dit betekent dat je een (bijna) identieke \tifonttxt{While}-loop nodig hebt voor de zon,
    binnenin de loop voor de regen.
  \end{itemize}
\end{enumerate}

\vspace{2.5cm}


\begin{tikzpicture}[overlay,remember picture]
	\node[yshift=0.6cm] (BL) at (current page.south east){ };
	\node [anchor=south east,xshift=0.195cm] (BL2) at (BL)
	{%
		\begin{ticalc}[3.4cm]
			MATH\,\select{NUM}\,CPX\,PRB \\
			1\+\:abs( \\
			2\:round( \\
			3\:iPart( \\
			4\:fPart( \\
			\selectitem{5\:}int( \\
			6\:min( \\
			7\arrowdown max(
		\end{ticalc}
	};
	\node [anchor=south east,xshift=-3.6cm] (BL3) at (BL)
	{%
		\begin{ticalc}[3.4cm]
			MATH\,NUM\,CPX\,\select{PRB} \\
			\selectitem{1\+\:}rand( \\
			2\:nPr \\
			3\:nCr \\
			4\:! \\
			5\:randInt( \\
			6\:randNorm( \\
			7\arrowdown randBin(
		\end{ticalc}
	};
	\node [anchor=south east,xshift=-7.4cm] (BL4) at (BL)
	{%
		\begin{ticalc}
			PROGRAM\:FRSTWHLE\\%
			\:0\>R
			\:While\,R\!1\\%
			\:int(rand*10)\>R\\%
			\:Disp\,\qt DROOG\qt\\%
			\:End\\%
			\:Disp\,\qt REGEN\qt
		\end{ticalc}
	};
\end{tikzpicture}

\end{frame}



\begin{frame}
\frametitle{\tifonttxt{While} zelf aan de slag \tifonttxt{End}}
\framesubtitle{Antwoord}

\begin{enumerate}
  \item<1-> Voer dit \tifonttxt{prgm} een aantal keer uit. Zie je altijd hetzelfde?
  \begin{itemize}
    \item<1-> Hoe vaak de loop wordt uitgevoerd varieert!
  \end{itemize}
  \item<2-> Hoe kun je het verwachtte aantal repetities veranderen?
  \begin{itemize}
    \item<2-> Het getal `\tifonttxt{10}' moet verandert worden. Groter = meer repetities.
  \end{itemize}
  \item<3-> \lenitem[0.7\linewidth]{Breid het programma nu uit met de zon: Pas als de zon stopt met schijnen, dan kan het gaan regenen.}
\end{enumerate}

\vspace{2cm}


\begin{tikzpicture}[overlay,remember picture]
	\node[yshift=0.6cm] (BL) at (current page.south east){ };
	\node [anchor=south east,xshift=0.195cm] (BL2) at (BL)
	{%
	\only<3->{
		\begin{ticalc}
			PROGRAM\:SCNDWHLE\\%
			\:0\>R
			\:While\,R\!1\\%
			\:0\>S
			\:While\,S\!1\\%
			\:int(rand*4)\>S\\%
			\:Disp\,\qt ZON\qt\\%
			\:End\\%
			\:int(rand*10)\>R\\%
			\:Disp\,\qt DROOG\qt\\%
			\:End\\%
			\:Disp\,\qt REGEN\qt
		\end{ticalc}
	}%
	\only<1-2>{
		\begin{ticalc}
			DROOG\\%
			DROOG\\%
			DROOG\\%
			DROOG\\%
			DROOG\\%
			DROOG\\%
			REGEN\\%
			\hfill Done
		\end{ticalc}
	}%
	};
	\node [anchor=south east,xshift=-3.6cm] (BL3) at (BL)
	{%
		\begin{ticalc}
			PROGRAM\:FRSTWHLE\\%
			\:0\>R
			\:While\,R\!1\\%
			\:int(rand*\only<2>{\select{10}}\only<1,3->{10})\>R\\%
			\:Disp\,\qt DROOG\qt\\%
			\:End\\%
			\:Disp\,\qt REGEN\qt
		\end{ticalc}
	};
\end{tikzpicture}

\end{frame}



\subsection{Pause}

\begin{frame}
\frametitle{Intermezzo: \tifonttxt{Pause}}

\begin{itemize}
  \item<1-> In de \tifonttxt{While}-\tifonttxt{Prgms} van de vorige slides, werd je beeldscherm behoorlijk vol geschreven.
  \item<2-> Zo kan de gebruiker natuurlijk niet zien wat we \tifonttxt{Disp}en!
  \item<3-> Met \tifonttxt{Pause} kun je de executie van het programma pauseren.
  \item<4-> Executie gaat verder wanneer de gebruiker op \tiENTER\,drukt.
  \item<5-> \tifonttxt{Pause} staat onder \tiPRGM\,in het \tifonttxt{CTL}-menu.
  \item<6-> Blader daarvoor naar beneden \tiDown.
  \item<7-> \lenitem[0.7\linewidth]{Speel ermee door \tifonttxt{Pause} toe te voegen in \tifonttxt{SCNDWHLE}!}
\end{itemize}

\vspace{2cm}

\begin{tikzpicture}[overlay,remember picture]
	\node[yshift=0.6cm] (BL) at (current page.south east){ };
	\node [anchor=south east,xshift=0.195cm] (BL2) at (BL)
	{%
	\only<1-4>{%
		\begin{ticalc}
			PROGRAM\:SCNDWHLE\\%
			\:0\>R
			\:While\,R\!1\\%
			\:0\>S
			\:While\,S\!1\\%
			\:int(rand*4)\>S\\%
			\:Disp\,\qt ZON\qt\\%
			\:End\\%
			\:int(rand*10)\>R\\%
			\:Disp\,\qt DROOG\qt\\%
			\:End\\%
			\:Disp\,\qt REGEN\qt
		\end{ticalc}
	}%
	\only<7->{%
		\begin{ticalc}
			PROGRAM\:SCNDWHLE\\%
			\:0\>R
			\:While\,R\!1\\%
			\:0\>S
			\:While\,S\!1\\%
			\:int(rand*4)\>S\\%
			\:Disp\,\qt ZON\qt\\%
			\:End\\%
			\:int(rand*10)\>R\\%
			\:Disp\,\qt DROOG\qt\\%
			\:\select{Pause}\\%
			\:End\\%
			\:Disp\,\qt REGEN\qt
		\end{ticalc}
	}%
	\only<5>{%
		\begin{ticalc}
			\select{CTL}\,I/O\,EXEC \\
			\selectitem{1\+\:}If \\
			2\:Then \\
			3\:Else \\
			4\:For( \\
			5\:While \\
			6\:Repeat( \\
			7\arrowdown End
		\end{ticalc}
	}%
	\only<6>{%
		\begin{ticalc}
			\select{CTL}\,I/O\,EXEC \\
			4\arrowup For( \\
			5\:While \\
			6\:Repeat( \\
			7\:End \\
			\selectitem{8\:}Pause \\
			9\:Lbl \\
			0\arrowdown Goto
		\end{ticalc}
	}%
	};
	\node [anchor=south east,xshift=-3.6cm] (BL3) at (BL)
	{%
		\begin{ticalc}
			PROGRAM\:FRSTWHLE\\%
			\:0\>R
			\:While\,R\!1\\%
			\:int(rand*10)\>R\\%
			\:Disp\,\qt DROOG\qt\\%
			\:End\\%
			\:Disp\,\qt REGEN\qt
		\end{ticalc}
	};
\end{tikzpicture}

\end{frame}




\section{Datatype: Lists}




\section{Exercises}
\subsection{Exercises}


\begin{frame}
\frametitle{\tifonttxt{WIEWINT3}: Wie worden er tweede en derde?}

Breid \tifonttxt{WIEWINT2} uit door ook te weergeven wie tweede en derde zijn geworden.

\underline{Tip}: Er bestaat geen ``max2'' functie of iets dergelijks voor het vinden van het op-\'e\'en-na maximum,
dus je zult iets anders moeten bedenken met behulp van \tifonttxt{max}, of zelfs iets totaal anders.

\begin{ticalc}[3.5cm]
	PROGRAM\:WIEWINT2\\%
	\:Prompt\,L\sub{1}\\%
	\:max(L\sub{1})\>M\\%
	\:Disp\,\qt WINNERS\:\qt\\%
	\:For(I\comma1\comma dim(L\sub{1}))\\%
	\:If\,M=L\sub{1}(I)\:Then\\%
	\:Disp\,I\\%
	\:End\\%
	\:End
\end{ticalc}

\end{frame}




\begin{frame}
\frametitle{Exercises}

\begin{enumerate}
  \item Maak \inlineticalc{FORDISP} na, maar gebruik `\tifonttxt{While}' i.p.v. `\tifonttxt{For}'!
	\begin{itemize}
	  \item Zie je de gelijkenis? Beide statements zorgen voor herhaling.
	\end{itemize}
  \item Zelfde vraag, maar gebruik nu \tifonttxt{Goto} en \tifonttxt{Lbl}.
	\begin{itemize}
	  \item Dat kost veel moeite\ldots Gelukkig bestaan \tifonttxt{For} en \tifonttxt{While}!
	\end{itemize}
  \item Vind hoe vaak 2 in \inlineticalc{Prompt\,X} past. 
	\begin{itemize}
	  \item Bijv.: in het getal $40$ past `$2$' $3$x.
	\end{itemize}
  \item Vind alle priemgetallen kleiner dan \inlineticalc{Prompt\,X}.
	\begin{itemize}
	  \item Een priemgetal is alleen (integer) deelbaar door $1$ en zichzelf. ($2,3,5,7,11,13,17,19,23,\ldots$)
	\end{itemize}
  \item Ontbind het getal \inlineticalc{Prompt\,X}\,in priemgetallen. {\tiny{(Prime Factorisation)}}
	\begin{itemize}
	  \item Bijv.: $84=2^2\cdot3\cdot7$.
	  \tiny{\item Zie ook: http://www.2dtx.com/prime/prime84.html}
	\end{itemize}
  \item \lenitem[0.7\linewidth]{Schrijf het `risk dobbelstenen' programma van de lecture over pseudocode m.b.v. Lists.}
\end{enumerate}
MAAK ALTIJD EERST PSEUDOCODE!

\begin{tikzpicture}[overlay,remember picture]
	\node[yshift=0.6cm] (BL) at (current page.south east){ };
	\node [anchor=south east,xshift=0.195cm] (BL2) at (BL)
	{%
	\begin{ticalc}
		PROGRAM\:FORDISP\\%
		\:For(I\comma1\comma3\comma1)\\%
		\:Disp \qt FOR\qt\\%
		\:End
	\end{ticalc}
	};
\end{tikzpicture}



\end{frame}



\subsection{Answers}


\begin{frame}
\frametitle{\tifonttxt{WIEWINT3}: Wie worden er tweede en derde?}
\framesubtitle{Een mogelijk antwoord (er zijn zoals altijd meer manieren!)}

\vspace{-0.5cm}
\hspace{-1cm}
\begin{minipage}{0.83\textwidth}
	\begin{algorithm}[H]
	\caption{``WieWint3? met Lists''}
	\begin{algorithmic}[1]
	\Function{WieWint}{$\mathrm{L}_{1}$: list met scores}
	  \State Init. $\mathrm{L}_{2}$ als $\mathrm{L}_{1}$: iedereen doet mee voor rank 1
	  \For{rank 1, 2, 3}
		  \State Vindt punten voor huidige rank (uit $\mathrm{L}_{2}$): M
		  \State Clear $\mathrm{L}_{2}$ voor hergebruik
		  \For{alle spelers uit $\mathrm{L}_{1}$}
		  \If{punten = M} Deze speler wint!
		  \ElsIf{punten < M}
		  	\State Deze speler is wellicht een rank lager!
		  	\State Voeg hem toe aan $\mathrm{L}_{2}$
		  \EndIf
		  \EndFor \, Pause
	  \EndFor
	\EndFunction
	\end{algorithmic}
	\end{algorithm}
\end{minipage}
\begin{minipage}{0.15\textwidth}%
\begin{ticalc}[3.25cm]	
	PROGRAM\:WIEWINT3\\%
	\:Prompt\,L\sub{1}\\%
	\:L\sub{1}\>L\sub{2}\\%
	\:For(J\comma1\comma3)\\%
	\:max(L\sub{2})\>M\\%
	\:\{min(L\sub{1}\>L\sub{2}\\%
	\:Disp\,\qt RANK\:\qt\comma J\\%
	\:Disp\,\qt PEOPLE\:\qt\\%
	\:For(I\comma1\comma dim(L\sub{1}))\\%
	\:If\,M=L\sub{1}(I)\:Then\\%
	\:Disp\,I\\%
	\:Else\\%
	\:If\,M>L\sub{1}(I)\:Then\\%
	\:augment(L\sub{2}\comma\{L\sub{1}(I )\})\>L\sub{2}\\%
	\:End
	\:End
	\:End\\%
	\:Pause
	\:End
\end{ticalc}

\begin{minipage}{2.2\linewidth}
\scriptsize{(augment = samenvoegen)}
\end{minipage}
\end{minipage}

\end{frame}




\begin{frame}
\frametitle{Antwoord: \tifonttxt{FORDISP}}

\begin{minipage}{0.68\textwidth}
\begin{itemize}
  \item Deze opdracht is een mooi voorbeeld voor het principe ``the right tool for the job''.
  \item Merk op dat ``vroeger'' mensen geen \tifonttxt{For} en \tifonttxt{While} hadden. De computer begreep alleen \tifonttxt{Goto}. Lucky you!
  \item Het gebruik van \tifonttxt{Goto} op onderstaande manier is ``bad practice'' (don't do it).
  \begin{itemize}
    \item ``Using a Goto to exit any block of code requiring an End command causes a memory leak,
    	which will not be usable until the program finishes running or executes a Return command, and which will slow your program down.''
    \item \tiny{Zie ook: http://tibasicdev.wikidot.com/goto}
  \end{itemize}
\end{itemize}
\end{minipage}
\begin{minipage}{0.3\textwidth}
\begin{ticalc}
	PROGRAM\:FORDISP\\%
	\:For(I\comma1\comma3\comma1)\\%
	\:Disp \qt FOR\qt\\%
	\:End\:Pause\\%
	\:\\%
	\:1\>I\\%
	\:While\,I\le3\\%
	\:Disp\,\qt WHILE\qt\\%
	\:I+1\>I\\%
	\:End\:Pause\\%
	\:\\%
	\:1\>I\\%
	\:Lbl\,A\\%
	\:Disp\,\qt GOTO\qt\\%
	\:I+1\>I\\%
	\:If\,I\le3\:Then\\%
	\:Goto\,A\\%
	\:End	
\end{ticalc}
\end{minipage}

\end{frame}



\begin{frame}
\frametitle{Antwoord: Hoe vaak past $2$ in \tifonttxt{X}?}

\vspace{-0.5cm}
\hspace{-1cm}
\begin{minipage}{0.8\textwidth}
	\begin{algorithm}[H]
	\caption{``Hoe vaak past 2 in $X$?''}
	\begin{algorithmic}[1]
	\Function{HoeVaak2}{$X$}
	  \State Doe wat sanity-checks \& initialiseer
	  \While{X is een geheel getal ($\equiv$integer)}
	    \State Vervang $X$ door $X/2$.
	    \State Houd een tellertje bij, \tifonttxt{I}.
	  \EndWhile
	\EndFunction
	\end{algorithmic}
	\end{algorithm}
\end{minipage}
\begin{minipage}{0.15\textwidth}%
\begin{ticalc}
	PROGRAM\:HOEVAAK2\\%
	\:Prompt\,X\\%
	\:If\,fPart(X)\!0\,or\,X=0
	\:Then\\%
	\:Disp\,0\\%
	\:Stop\\%
	\:End\\%
	\:\\%
	\:\min 1\>I\\%
	\:While\,fPart(X)= 0\\%
	\:X/2\>X\\%
	\:I+1\>I\\%
	\:End\\%
	\:Disp\,I
\end{ticalc}
\end{minipage}

\end{frame}




\begin{frame}
\frametitle{Antwoord: Vind alle priemgetallen \tifonttxt{\le X}}

\vspace{-0.5cm}
\hspace{-1cm}
\begin{minipage}{0.83\textwidth}
	\begin{algorithm}[H]
	\caption{``Vind alle priemgetallen $\le X$.''}
	\begin{algorithmic}[1]
	\Function{Primes}{$X$: max. waarde om te checken}
	  \State Init. een list, $\mathrm{L}_1$, om de primes bij te houden
	  \For{\tifonttxt{I}: Alle integers $\le X$}
	    \State Neem \`a priori aan dat \tifonttxt{I} prime is.
	    \For{\tifonttxt{J}: Alle primes kleiner dan \tifonttxt{I}}
	      \If{\tifonttxt{I} deelbaar is door $\mathrm{L}_1(J)$}
	        \State \tifonttxt{I} is \underline{geen} prime. Break loop.
	      \EndIf
	    \EndFor
	    \State Indien \tifonttxt{I} niet `geen prime' is, voeg toe aan $\mathrm{L}_1$
	  \EndFor
	\EndFunction
	\end{algorithmic}
	\end{algorithm}
\end{minipage}
\begin{minipage}{0.15\textwidth}%
\begin{ticalc}
	PROGRAM\:PRIMES\\%
	\:Prompt\,X\\%
	\:\{2\}\>L\sub{1}\\%
	\:For(I\comma3\comma X)\\%
	\:1\>P\\%
	\:For(J\comma1\comma dim(L\sub{1}))\\%
	\:If\,fPart(I/L\sub{1}(J ))=0\:Then\\%
	\:0\>P\\%
	\:dim(L\sub{1})+1\>J\\%
	\:End\\%
	\:End\\%
	\:If\,P\:Then\\%
	\:augment(L\sub{1},\{I\}) \>L\sub{1}\\%
	\:End\:End\\%
	\:Disp\,L\sub{1}
\end{ticalc}
\end{minipage}

\end{frame}



\begin{frame}
\frametitle{Antwoord: Prime Factorisation van \tifonttxt{X}}


\vspace{-0.5cm}
\hspace{-1cm}
\begin{minipage}{0.82\textwidth}
	\begin{algorithm}[H]
	\caption{Pseudocode Prime Factorisation}
	\begin{algorithmic}[1]
	\Function{PrimeFac}{$X$: het te factoriseren getal}
	  \For{\tifonttxt{P}: Alle primes $<\sqrt{X}$ \textbf{OF} Alle natuurlijke getallen $<\sqrt{X}$} \Comment{\small{Beide opties geven hetzelfde resultaat.
	  												Het eerste is sneller voor grote $X$, het tweede voor kleine $X$. De tweede optie is makkelijker.}}
	    \If{$X$ deelbaar door \tifonttxt{P}}
	      \State Store (=Onthoud/Bewaar) \tifonttxt{P}
	      \State Ga verder met \tifonttxt{X/P}
	      \State Herhaal de loop voor \tifonttxt{P}
	    \EndIf
	  \EndFor
	\EndFunction
	\end{algorithmic}
	\end{algorithm}
\end{minipage}
\begin{minipage}{0.15\textwidth}%
\begin{ticalc}
	PROGRAM\:PRIMEFAC\\%
	\:Prompt\,X\\%
	\:If\,X=0\:Then\\%
	\:\{0\}\>L\sub{1}
	\:Disp\,L\sub{1}\\%
	\:Stop
	\:End\\%
	\:X/abs(X)\>S\\%
	\:XS\>Y\\%
	\:\\%
	\:\{S\}\>L\sub{1}\\%
	\:For(P\comma2\comma int(\sqrt(Y) ))\\%
	\:Y/P\>Z\\%
	\:If\,fPart(Z)=0\\%
	\:Then\\%
	\:augment(L\sub{1}\comma\{P\}) \>L\sub{1}\\%
	\:Z\>Y
	\:P-1\>P\\%
	\:End\:End\\%
	\:Disp\,L\sub{1}
\end{ticalc}
\end{minipage}

\end{frame}



\begin{frame}
\frametitle{Antwoord: Risk Dobbelstenen I}

\begin{algorithm}[H]
\caption{Pseudocode Risk Dobbelstenen}
\begin{algorithmic}[1]
% Risk dobbelstenen
\Function{RiskDice}{$N_{att}$}
	\State Maak $N_{att}$ random getallen tussen $1$ en $6$
	\State Vraag of $N_{def}$ $1$ of $2$ is
	\State Maak $N_{def}$ random getallen tussen $1$ en $6$
	\State Sorteer beide setten van hoog naar laag
	\If{Getal 1 van set 1 > Getal 1 van set 2}  $p_1=p_1+1$
	\Else \hspace{0.4cm} $p_2=p_2+1$
	\EndIf
	\If{Getal 2 van set 1 > Getal 2 van set 2}  $p_1=p_1+1$
	\Else \hspace{0.4cm} $p_2=p_2+1$
	\EndIf
	\State Display de scores: $p_1$ en $p_2$
\EndFunction
\end{algorithmic}
\end{algorithm}

\end{frame}



\begin{frame}
\frametitle{Antwoord: Risk Dobbelstenen II}

\vspace{-0.5cm}
\hspace{-1cm}
\begin{minipage}{0.5\textwidth}
\begin{ticalc}[4cm]
	PROGRAM\:RISKDICE\\%
	\:0\>A\:0\>D\\%
	\:While\,A<1\,or\,A>3\\%
	\:Input\,\qt NATT?\,\qt\comma A\\%
	\:End\\%
	\:seq(randInt(1\comma6)\comma X \comma1\comma A\comma1)\>L\sub{1}\\%
	\:SortD(L\sub{1})\\%
	\:Disp\,L\sub{1}\\%
	\:\\%
	\:While\,D<1\,or\,D>A\,or\,D>2\\%
	\:Input\,\qt NDEF?\,\qt\comma D\\%
	\:End\\%
	\:seq(randInt(1\comma6)\comma X \comma1\comma D\comma1)\>L\sub{2}\\%
	\:SortD(L\sub{2})\\%
	\:Disp\,L\sub{2}\\%
	\:
\end{ticalc}
\end{minipage}
\begin{minipage}{0.5\textwidth}%
\begin{ticalc}[4.2cm]
	\:0\>P\:0\>Q\\%
	\:If\,L\sub{1}(1)>L\sub{2}(1)\\%
	\:Then\:P+1\>P\\%
	\:Else\:Q+1\>Q\\%
	\:End\\%
	\:If\,D>1\:Then\\%
	\:If\,L\sub{1}(2)>L\sub{2}(2)\\%
	\:Then\:P+1\>P\\%
	\:Else\:Q+1\>Q\\%
	\:End\:End\\%
	\:\\%
	\:ClrHome\\%
	\:Disp\,L\sub{1}\comma L\sub{2}\\%
	\:Disp\,\qt ATT\,LOSES\:\qt\comma Q\\%
	\:Disp\,\qt DEF\,LOSES\:\qt\comma P	
\end{ticalc}
\end{minipage}

\end{frame}


% TOPIC:
% Goto & Label (memory leak!)
% Repeat -> For & While
% Pause
% Lists
% 
% 
% IDEAS:
% Vind hoe vaak 2 in $x$ past
% Prime number
% Prime Factorization
% Herhaal voor multiplay game -> List
% Exactor
% ``Er zijn vier spelers die elk punten hebben. Bepaal de winnaar, i.e. degene met de meeste punten.'' -> Maar nu met List!
% Convert While <-> For loop
% Convert Goto/Label to While-loop
% Fibonacci in list
% Risk dobbelstenen met list
% 
%






\end{document}

%% END %%