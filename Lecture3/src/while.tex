\subsection{While-loops}


\begin{frame}
\frametitle{Wat is een \tifonttxt{While}-loop?}

\begin{itemize}
  \item<1-> \tifonttxt{For} is leuk indien je weet hoe vaak je iets wilt herhalen.
  \item<2-> \tifonttxt{While} is leuk indien je dat \`a priori juist \emph{niet} weet.
  \item<3-> Een \tifonttxt{While}-loop gaat door totdat een conditie (\tifonttxt{If}) onwaar wordt.
  \item<4-> Bijvoorbeeld:
  \begin{itemize}
    \item<5-> ``Herhaal dit totdat het regent''
    \item<6-> ``Herhaal dit totdat de berekening nauwkeurig genoeg is''
  \end{itemize}
\end{itemize}

\end{frame}





\begin{frame}
\frametitle{Wat is een \tifonttxt{While}-loop?}

\begin{itemize}
  \item<1-> Op de rekenmachine ziet dit er zo uit:
  \begin{itemize}
    \item<1-> ``Herhaal dit totdat het regent''
    \item<3-> \tifonttxt{int(rand*10)} genereert een willekeurige integer tussen \tifonttxt{0} en \tifonttxt{9}.
    	Dit geeft een $10\%$ kans dat de loop breaks (=stopt) elke keer dat hij execute (=herhaalt).
  \end{itemize}
  \item<4-> De structuur is dus: \tifonttxt{While\,}<<conditie>>
  \item<5-> \tifonttxt{While} staat onder \tiPRGM\,in het \tifonttxt{CTL}-menu.
  \item<6-> De loop eindigt met de oude vertrouwde \tifonttxt{End}
\end{itemize}

\vspace{2cm}

\begin{tikzpicture}[overlay,remember picture]
	\node[yshift=0.6cm] (BL) at (current page.south east){ };
	\node [anchor=south east,xshift=0.195cm] at (BL)
	{%
	\only<1-4,6->{%
		\begin{ticalc}
			PROGRAM\:FRSTWHLE\\%
			\:0\>R
			\:\only<4-6>{\select{While\,R\!1}}\only<1-3>{While\,R\!1}\\%
			\:int(rand*10)\>R\\%
			\:Disp\,\qt DROOG\qt\\%
			\:\only<6>{\select{End}}\only<1-5>{End}\\%
			\:Disp\,\qt REGEN\qt
		\end{ticalc}
	}%
	\only<5>{%
		\begin{ticalc}
			\select{CTL}\,I/O\,EXEC \\
			\one\:If \\
			2\:Then \\
			3\:Else \\
			4\:For( \\
			\selectitem{5\:}While \\
			6\:Repeat( \\
			7\arrowdown End
		\end{ticalc}
	}%
	};
	\node [anchor=south east,xshift=-3.6cm] (BL3) at (BL)
	{%
	\only<2->{
		\begin{ticalc}
			DROOG\\%
			DROOG\\%
			DROOG\\%
			DROOG\\%
			REGEN\\%
			\hfill Done
		\end{ticalc}
	}
	};
\end{tikzpicture}

\end{frame}





\begin{frame}
\frametitle{\tifonttxt{While} zelf aan de slag \tifonttxt{End}}

\begin{enumerate}
  \item Maak \tifonttxt{FRSTWHLE} na. \tifonttxt{int(} en \tifonttxt{rand} staan onder \tiMATH.
  \item Voer dit \tifonttxt{prgm} een aantal keer uit. Zie je altijd hetzelfde?
  \item Hoe kun je het verwachtte aantal repetities veranderen?
  \item Breid het programma nu uit met de zon: Pas als de zon stopt met schijnen, dan kan het gaan regenen.
  \begin{itemize}
    \item \underline{Tip}: Dit betekent dat je een (bijna) identieke \tifonttxt{While}-loop nodig hebt voor de zon,
    binnenin de loop voor de regen.
  \end{itemize}
\end{enumerate}

\vspace{2.5cm}


\begin{tikzpicture}[overlay,remember picture]
	\node[yshift=0.6cm] (BL) at (current page.south east){ };
	\node [anchor=south east,xshift=0.195cm] (BL2) at (BL)
	{%
		\begin{ticalc}[3.4cm]
			MATH\,\select{NUM}\,CPX\,PRB \\
			\one\:abs( \\
			2\:round( \\
			3\:iPart( \\
			4\:fPart( \\
			\selectitem{5\:}int( \\
			6\:min( \\
			7\arrowdown max(
		\end{ticalc}
	};
	\node [anchor=south east,xshift=-3.6cm] (BL3) at (BL)
	{%
		\begin{ticalc}[3.4cm]
			MATH\,NUM\,CPX\,\select{PRB} \\
			\selectitem{\one\:}rand \\
			2\:nPr \\
			3\:nCr \\
			4\:! \\
			5\:randInt( \\
			6\:randNorm( \\
			7\arrowdown randBin(
		\end{ticalc}
	};
	\node [anchor=south east,xshift=-7.4cm] (BL4) at (BL)
	{%
		\begin{ticalc}
			PROGRAM\:FRSTWHLE\\%
			\:0\>R
			\:While\,R\!1\\%
			\:int(rand*10)\>R\\%
			\:Disp\,\qt DROOG\qt\\%
			\:End\\%
			\:Disp\,\qt REGEN\qt
		\end{ticalc}
	};
\end{tikzpicture}

\end{frame}



\begin{frame}
\frametitle{\tifonttxt{While} zelf aan de slag \tifonttxt{End}}
\framesubtitle{Antwoord}

\begin{enumerate}
  \item<1-> Voer dit \tifonttxt{prgm} een aantal keer uit. Zie je altijd hetzelfde?
  \begin{itemize}
    \item<1-> Hoe vaak de loop wordt uitgevoerd varieert!
  \end{itemize}
  \item<2-> Hoe kun je het verwachtte aantal repetities veranderen?
  \begin{itemize}
    \item<2-> Het getal `\tifonttxt{10}' moet verandert worden. Groter = meer repetities.
  \end{itemize}
  \item<3-> \lenitem[0.7\linewidth]{Breid het programma nu uit met de zon: Pas als de zon stopt met schijnen, dan kan het gaan regenen.}
\end{enumerate}

\vspace{2cm}


\begin{tikzpicture}[overlay,remember picture]
	\node[yshift=0.6cm] (BL) at (current page.south east){ };
	\node [anchor=south east,xshift=0.195cm] (BL2) at (BL)
	{%
	\only<3->{
		\begin{ticalc}
			PROGRAM\:SCNDWHLE\\%
			\:0\>R
			\:While\,R\!1\\%
			\:0\>S
			\:While\,S\!1\\%
			\:int(rand*4)\>S\\%
			\:Disp\,\qt ZON\qt\\%
			\:End\\%
			\:int(rand*10)\>R\\%
			\:Disp\,\qt DROOG\qt\\%
			\:End\\%
			\:Disp\,\qt REGEN\qt
		\end{ticalc}
	}%
	\only<1-2>{
		\begin{ticalc}
			DROOG\\%
			DROOG\\%
			DROOG\\%
			DROOG\\%
			DROOG\\%
			DROOG\\%
			REGEN\\%
			\hfill Done
		\end{ticalc}
	}%
	};
	\node [anchor=south east,xshift=-3.6cm] (BL3) at (BL)
	{%
		\begin{ticalc}
			PROGRAM\:FRSTWHLE\\%
			\:0\>R
			\:While\,R\!1\\%
			\:int(rand*\only<2>{\select{10}}\only<1,3->{10})\>R\\%
			\:Disp\,\qt DROOG\qt\\%
			\:End\\%
			\:Disp\,\qt REGEN\qt
		\end{ticalc}
	};
\end{tikzpicture}

\end{frame}



\subsection{Pause}

\begin{frame}
\frametitle{Intermezzo: \tifonttxt{Pause}}

\begin{itemize}
  \item<1-> In de \tifonttxt{While}-\tifonttxt{Prgms} van de vorige slides, werd je beeldscherm behoorlijk vol geschreven.
  \item<2-> Zo kan de gebruiker natuurlijk niet zien wat we \tifonttxt{Disp}en!
  \item<3-> Met \tifonttxt{Pause} kun je de executie van het programma pauseren.
  \item<4-> Executie gaat verder wanneer de gebruiker op \tiENTER\,drukt.
  \item<5-> \tifonttxt{Pause} staat onder \tiPRGM\,in het \tifonttxt{CTL}-menu.
  \item<6-> Blader daarvoor naar beneden \tiDown.
  \item<7-> \lenitem[0.7\linewidth]{Speel ermee door \tifonttxt{Pause} toe te voegen in \tifonttxt{SCNDWHLE}!}
\end{itemize}

\vspace{2cm}

\begin{tikzpicture}[overlay,remember picture]
	\node[yshift=0.6cm] (BL) at (current page.south east){ };
	\node [anchor=south east,xshift=0.195cm] (BL2) at (BL)
	{%
	\only<1-4>{%
		\begin{ticalc}
			PROGRAM\:SCNDWHLE\\%
			\:0\>R
			\:While\,R\!1\\%
			\:0\>S
			\:While\,S\!1\\%
			\:int(rand*4)\>S\\%
			\:Disp\,\qt ZON\qt\\%
			\:End\\%
			\:int(rand*10)\>R\\%
			\:Disp\,\qt DROOG\qt\\%
			\:End\\%
			\:Disp\,\qt REGEN\qt
		\end{ticalc}
	}%
	\only<7->{%
		\begin{ticalc}
			PROGRAM\:SCNDWHLE\\%
			\:0\>R
			\:While\,R\!1\\%
			\:0\>S
			\:While\,S\!1\\%
			\:int(rand*4)\>S\\%
			\:Disp\,\qt ZON\qt\\%
			\:End\\%
			\:int(rand*10)\>R\\%
			\:Disp\,\qt DROOG\qt\\%
			\:\select{Pause}\\%
			\:End\\%
			\:Disp\,\qt REGEN\qt
		\end{ticalc}
	}%
	\only<5>{%
		\begin{ticalc}
			\select{CTL}\,I/O\,EXEC \\
			\selectitem{1\+\:}If \\
			2\:Then \\
			3\:Else \\
			4\:For( \\
			5\:While \\
			6\:Repeat( \\
			7\arrowdown End
		\end{ticalc}
	}%
	\only<6>{%
		\begin{ticalc}
			\select{CTL}\,I/O\,EXEC \\
			4\arrowup For( \\
			5\:While \\
			6\:Repeat( \\
			7\:End \\
			\selectitem{8\:}Pause \\
			9\:Lbl \\
			0\arrowdown Goto
		\end{ticalc}
	}%
	};
	\node [anchor=south east,xshift=-3.6cm] (BL3) at (BL)
	{%
		\begin{ticalc}
			PROGRAM\:FRSTWHLE\\%
			\:0\>R
			\:While\,R\!1\\%
			\:int(rand*10)\>R\\%
			\:Disp\,\qt DROOG\qt\\%
			\:End\\%
			\:Disp\,\qt REGEN\qt
		\end{ticalc}
	};
\end{tikzpicture}

\end{frame}



