\section{Hoe open je een programma?}

\begin{frame}
\frametitle{\tifontbigtxt{NEW} \tiPRGM}

\visible<1-2>{\ticalcfig{\ticalcfigCircle{\ticalcfigCircleColThree}{0.615}}} % prgm
\visible<3>{\ticalcfig{\ticalcfigCircleEnter\ticalcfigCircleRight}}
\visible<4>{\ticalcfig{\ticalcfigCircleEnter \ticalcfigCircleAlpha\ticalcfigCircleSecond}}
\visible<5>{\ticalcfig{\ticalcfigCircleSecond\ticalcfigCircle{\ticalcfigCircleColTwo}{0.81}}}

We gaan ons eerste programma aanmaken.
\begin{itemize}
  \item Druk op \tiPRGM.
  \pause %1
  \item \lenitem{Tenzij je eerder een programma hebt gemaakt, zie je alleen
  		\inlineticalc{\only<1-2>{\select{EXEC}}\only<3->{EXEC}\,EDIT\,\only<3->{\select{NEW}}\only<1-2>{NEW}}.}
  \pause %2
  \item Blader met \tiRight\,naar \tifonttxt{NEW} en druk op \tiENTER.
  \pause %3
  \item \lenitem{Typ een naam in. Een prgm naam is maximaal 8 karakters. Druk vervolgens op \tiENTER.}
	  \begin{ticalc}[3.25cm]
	  	PROGRAM\\
	  	NAME=MYPRGM01
	  \end{ticalc}
  
  Merk op dat je een \tiCursorAlpha-cursor hebt: \tiALOCK\,is geactiveerd.
  \pause %4
  \item Sluit het programma nu met \tiQUIT.
\end{itemize}
\end{frame}

\begin{frame}
\frametitle{\tifontbigtxt{EDIT} \tiPRGM}

\visible<1-2>{\ticalcfig{\ticalcfigCircle{\ticalcfigCircleColThree}{0.615}}}
\visible<3>{\ticalcfig{\ticalcfigCircleRight}}
\visible<4>{\ticalcfig{\ticalcfigCircleDown\ticalcfigCircleEnter}}
\visible<5>{\ticalcfig{}}
\visible<6>{\ticalcfig{\ticalcfigCircleAlpha}}

Om het programma nu weer te openen, doe:
\begin{itemize}
  \item Druk op \tiPRGM.
  \pause %1
  \item \lenitem{Je ziet nu een lijst met alle programma's die je kunt uitvoeren.}
	  \begin{ticalc}[3.25cm]
	  	\only<1-2>{\select{EXEC}}\only<3->{EXEC}\,\only<3->{\select{EDIT}}\only<1-2>{EDIT}\,NEW
	  	\selectitem{1\:}MYPRGM01
	  \end{ticalc}
  \pause %2
  \item Blader met \tiRight\,naar \tifonttxt{EDIT}.
  \pause %3
  \item Hier zie je dezelfde lijst. Gebruik \tiDown\, om naar je programma te bladeren en druk op \tiENTER.
  \pause %4
  \item Als alternatief, kun je ook het nummer intoetsen wat voor je programma staat. Dit is een hotkey om je programma te openen.
  \pause %5
  		Ook kun je met \tiALPHA\, de eerste letter van je programmanaam intoetsen om snel je programma te vinden
  		(indien je later een grotere lijst met programma's hebt dan 1).
\end{itemize}
\end{frame}

\begin{frame}
\frametitle{Deleten/Archiveren van een \tiPRGM}
	TODO?
\end{frame}



%% END %%