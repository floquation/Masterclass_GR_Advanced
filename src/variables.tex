\section{Variabelen en Datatypes}
\subsection{Getallen}

\begin{frame}
\frametitle{Datatype: Wat is een getal?}

Onder het datatype `getal', verstaan we alle numerieke waarden:
\pause%1
\begin{enumerate}
  \item Gehele getallen / Integers - E.g., \tifonttxt{-4}
\pause%2
  \item Reele getallen / Real Numbers - E.g., \tifonttxt{3.141592654}%TODO: Re�le met accent of e
\pause%3
  \item Complexe getallen / Complex Numbers - E.g., \tifonttxt{2+3\ci}
\end{enumerate}

\pause%4
Voor je rekenmachine zijn al deze getallen hetzelfde datatype (in tegenstelling tot op de computer!),
en daarom kan dezelfde variabele gebruikt worden voor elk van bovenstaande voorbeelden.

\end{frame}

\begin{frame}
\frametitle{\tifontbigtxt{Ans}}

\visible<2>{\ticalcfig{\ticalcfigCircleSecond\ticalcfigCircleEnter}}
\visible<3>{\ticalcfig{\ticalcfigCircleSecond\ticalcfigCircle[\ticalcfigCircleNumButtonsize]{\ticalcfigCircleColFour}{0.055}}} % Ans

\begin{itemize}
  \item \lenitem{Wanneer je een som intyped op je rekenmachine, zie je bijvoorbeeld:
\begin{ticalc}[3.25cm]
	\sqrt(3\sq+4\sq)\\
	\hfill 5
\end{ticalc}}
\pause%1
  \item \lenitem{Met \tiENTER\,herhalen we de laatste som.\\
  		Met \tiENTRY=\tiSecond\tiENTER\,halen we de som terug op ons scherm. (Probeer het!)}
\pause%2
  \item Met \tiANS\tiENTER\,printen we ook hetzelfde antwoord.
		\tifonttxt{Ans} is een ``variabele'' waarin het antwoord van de vorige bewerking is opgeslagen.
		Hier geldt dat \tifonttxt{Ans} een andere naam is voor \tifonttxt{5}.
\pause%3
  \item Bijvoorbeeld, wat is hiervan de uitkomst?
 
\begin{ticalc}[3.25cm]
	\sqrt(3\sq+4\sq)\\
	\hfill 5\\
	(Ans-4)\^Ans+1
	\visible<5->{\\\hfill 2}
\end{ticalc}
\pause%4
\visible<5->{Vervang alle \tifonttxt{Ans} door 5!}
\end{itemize}
\end{frame}

\begin{frame}
\frametitle{\tifontbigtxt{Ans}}
\framesubtitle{Meer voorbeelden}

Om te testen of je het begrijpt, wat zijn van de volgende sommen de uitkomsten?

\pause%1
\begin{ticalc}[3.25cm]
	3+4\\
	\hfill 7\\
	Ans-Ans*2
	\visible<3->{\\\hfill -7}
\end{ticalc}
\pause%2
\pause%3
\begin{ticalc}[3.25cm]
	0\\
	\hfill 0\\
	e\^Ans
	\visible<5->{\\\hfill 1}
\end{ticalc}
\pause%4
\pause%5

\vspace{5pt}
En wat moeilijker:

\begin{ticalc}[3.25cm]
	3\\
	\hfill 3\\
	Ans+2
	\visible<7->{\\\hfill 5}\\
	Ans-2
	\visible<7->{\\\hfill 3}
\end{ticalc}
\pause%6

\visible<7>{Merk op dat hierbij \tifonttxt{Ans} eerst gelijk is aan \tifonttxt{3}, maar vervolgens wordt overschreven door de nieuwe uitkomst: \tifonttxt{5}!
			
			Hierdoor evalueert de laatste som als \tifonttxt{5-2=3}.}

\end{frame}

\begin{frame}
\frametitle{\tifontbigtxt{Ans}}
\framesubtitle{Meer voorbeelden}

En een laatste moeilijke:

\begin{ticalc}[3.25cm]
	\sqrt(9)\\
	\hfill 3\\
	Ans/3
	\visible<2->{\\\hfill 1}\\
	cos(\pi Ans)
	\visible<2->{\\\hfill -1}\\
	Ans+3
	\visible<2->{\\\hfill 2}
\end{ticalc}
\pause%1

\visible<2->{Net als de vorige opgave, \tifonttxt{Ans} is eerst \tifonttxt{3}, en wordt overschreven door \tifonttxt{1}.

\pause%2
			De volgende som is dan \tifonttxt{cos(\pi*1)}.
			Merk hierbij op dat, net zoals je met $2x$ eigenlijk bedoelt $2*x$, hetzelfde op je rekenmachine geldt:
			\tifonttxt{\pi Ans}$=$\tifonttxt{\pi*Ans}.
			
\pause%3
			Als laatste stap is \tifonttxt{Ans} gelijk aan -1, waardoor de laatste som op \tifonttxt{2} eindigt.}
\end{frame}

\begin{frame}
\frametitle{Andere variabelen voor getallen}
\framesubtitle{Variabelen zijn doosjes.}


\begin{itemize}
  \item In het algemeen geldt dat een variabele een doosje is.
  		\pause%1
  		In dat doosje kan je iets stoppen (een getal), en je kunt het er later weer uithalen.
  		\pause%2
  		Een variabele is dus niets anders dan een `container' voor een getal.
  		\pause%3
  \item Net zagen we dat de inhoud van het doosje \tifonttxt{Ans} automatisch
  		vervangen werd, elke keer dat we een berekening uit voerde.
  		\pause%4
  \item Alle andere doosjes moeten we met de hand openen en vullen.
\end{itemize}
\end{frame}

\begin{frame}
\frametitle{Andere variabelen voor getallen}
\framesubtitle{Het lezen van variabelen}

\visible<1>{\ticalcfig{\ticalcfigCircleAlpha}} % Alpha
\visible<2>{\ticalcfig{\ticalcfigCircleAlpha\ticalcfigCircle{\ticalcfigCircleColOne}{0.635}}} % Alpha Math = A
\visible<3-4>{\ticalcfig{\ticalcfigCircleAlpha\ticalcfigCircle[\ticalcfigCircleNumButtonsize]{\ticalcfigCircleColTwo}{0.161}}} % Alpha 1 = Y
\visible<5>{\ticalcfig{\ticalcfigCircle{\ticalcfigCircleColOne}{0.952}}} % Y=
\visible<6->{\ticalcfig{\ticalcfigCircleSecond\ticalcfigCircle{\ticalcfigCircleColOne}{0.227}}} % 2nd Sto = RCL

\begin{itemize}
  \item \lenitem{Met \tiALPHA\,kunnen we alle letters, inclusief \tifonttxt{\theta}, bereiken.
  		Dit zijn alle variabelen die voor getallen bedoeld zijn.}
  		\pause%1
  \item	Bijvoorbeeld, \tiALPHA\tiMATH$=$\tifonttxt{A}.
  		\pause%2
  \item \lenitem{Je kunt de waarde van een variabele zien door de variabele te evalueren met \tiENTER:}
  
  		\begin{ticalc}[3.25cm]
  			Y\\
  			\hfill 2.585786438
  		\end{ticalc}
  		\pause%3
  		\underline{Probeer zelf wat variabelen}!
  		
  		\pause%4
  		Merk op dat \tifonttxt{X} en \tifonttxt{Y} variabelen zijn die door de grafische functies (\tiYEquals) worden gebruikt.
  		\pause%5
  \item	Als alternatief, kun je \tiRCL($=$\tiSecond\tiSTO)$+$\tifonttxt{Y}$+$\tiENTER\,gebruiken:
  
  		\begin{ticalc}[3.25cm]
  			Rcl \ Y
  		\end{ticalc}
  		\begin{ticalc}[3.25cm]
  			2.585786438\tiCursor
  		\end{ticalc}
  
\end{itemize}

\end{frame}


\begin{frame}
\frametitle{Andere variabelen voor getallen}
\framesubtitle{Het storen van getallen in variabelen}

\visible<1->{\ticalcfig{\ticalcfigCircle{\ticalcfigCircleColOne}{0.227}}} % Sto

Je kunt iets met \tiSTO\,als volgt in een doosje stoppen:

\pause%1
\inlineticalc{6\>A} stored het getal \tifonttxt{6} in de variabele \tifonttxt{A}.

\pause%2
\vspace{10pt}
\lenitem{``Store into a variable'' is de programmeerjargon voor een getal in een doosje stoppen.}

\pause%3
\vspace{30pt}
\visible<4->{En dat is alles wat je nodig hebt\ldots!}

\end{frame}

\begin{frame}
\frametitle{Andere variabelen voor getallen}
\framesubtitle{Voorbeelden}

% \begin{minipage}{0.39\textwidth}%
	\begin{ticalc}[3.25cm]
		3\>A\\
		\hfill 3\\
		2\>B\\
		\hfill 2\\
		A\^B
		\visible<2->{\\\hfill 9}
	\end{ticalc}

	\pause%1
	\visible<2->{Vervang \tifonttxt{A} door \tifonttxt{3} en \tifonttxt{B} door \tifonttxt{2} en de som is simpelweg $3^2=9$.}
% \end{minipage} ~

\pause%2
% \begin{minipage}{0.57\textwidth}%
	\begin{ticalc}[3.25cm]
		3\>A\\
		\hfill 3\\
		2\>A\\
		\hfill 2\\
		A\^A
		\visible<4->{\\\hfill 4}
	\end{ticalc}

	\pause%3
	\visible<4->{Vervang \tifonttxt{A} door \tifonttxt{2} en de som is simpelweg $2^2=4$.
	Merk hierbij op dat alleen het \emph{laatste} wat je in een doosje stopt in het doosje zit!}
% \end{minipage}

\end{frame}


\begin{frame}
\frametitle{Andere variabelen voor getallen}
\framesubtitle{Ultiem voorbeeld}

	\begin{ticalc}[3.25cm]
		3\>A\\
		\hfill 3\\
		A\>B\\
		\hfill 3\\
		B+A\>C
		\visible<2->{\\\hfill 6}\\
		0\>A
		\visible<2->{\\\hfill 0}\\
		A+B+C
		\visible<2->{\\\hfill 9}
	\end{ticalc}

	\pause%1
	\visible<2->{Merk op dat op het moment dat \tifonttxt{C} wordt uitgerekend, \tifonttxt{A} nog gelijk is aan \tifonttxt{3} (en niet \tifonttxt{0}).}
	
	\pause%2
	\vspace{5pt}
	\visible<3->{Speel zelf ook met wat gecompliceerde voorbeelden en kijk of je begrijpt wat je rekenmachine uitrekent!}	
\end{frame}

\subsection{Strings}

\begin{frame}
\frametitle{Datatype: Wat is een string?}

Volgens ``www.thefreedictionary.com":

\pause%1
\begin{enumerate}
  \item Material made of drawn-out, twisted fiber, used for fastening, tying, or lacing.
  \pause%2
  \item (Music) A cord stretched on an instrument and struck, plucked, or bowed to produce tones.
  \pause%3
  \item (Physics) One of the extremely minute objects that form the basis of string theory.
  \pause%4
  \item A number of objects arranged in a line: a string of islands.
  \pause%5
  \item \visible<6->{(Computers) A linear sequence of characters, words, or other data.}
\end{enumerate}

\pause%6
\visible<6->{Een `string' is gewoon een regel tekst!}

\end{frame}

\begin{frame}
\frametitle{Waar heb je een string voor nodig?}

\visible<6>{\ticalcfig{\ticalcfigCircle{\ticalcfigCircleColOne}{0.952}}} % Y=

Er zijn twee relevante toepassingen:
\pause%1
\begin{enumerate}
  \item \lenitem{Het weergeven van tekst aan de gebruiker van je programma.}
\pause%2
  		\begin{enumerate}
  		  \item De gebruiker uitleggen wat hij moet inputten.
\pause%3
  		  \item De gebruiker een resultaat in woorden laten zien.
\pause%4
  		  \item Etc.: Communiceren!
 	 	\end{enumerate}
\pause%5
  \item Vergelijkingen bij \tiYEquals\, zijn ook tekstregels:

		\inlineticalc{\qt X\sq\qt\>Y\_{1}} verandert de \tifonttxt{Y\_{1}} grafiek naar de functie $y=x^2$.

  		Daar gaan we in een later college op in.
\end{enumerate}

\end{frame}

\begin{frame}
\frametitle{Het storen van een string in een variabele}

\visible<1>{\ticalcfig{\ticalcfigCircle{\ticalcfigCircleColOne}{0.227}}} % Sto
\visible<2>{\ticalcfig{\ticalcfigCircle{\ticalcfigCircleColFour}{0.62}}} % Vars
\visible<3->{\ticalcfig{\ticalcfigCircleAlpha
		\ticalcfigCircle[\ticalcfigCircleNumButtonsize]{\ticalcfigCircleColTwo}{0.055} % 0/Catalog/VisibleSpace
		\ticalcfigCircle{\ticalcfigCircleColFive}{0.227} % Quote/+/mem 
}}

\begin{itemize}
  \item Gebruik dezelfde knop als voor getallen: \tiSTO.
  \pause%1
  \item \lenitem{Voor het string datatype, moeten we de \tifonttxt{Str\#} variabelen gebruiken.
  
  		Deze staan in het \tiVARS-menu onder ``\tifonttxt{String...}''.}
  \pause%2
  \item Bijvoorbeeld:
  
  		\begin{ticalc}[5cm]
  			\qt HELLO\,WORLD\qt\>Str1
  		\end{ticalc}
  \pause%3
  \item Merk hierbij op dat een string tussen quotatie-tekens (\tifonttxt{\qt}) moet staan: \tiQuote$=$\tiALPHA\tiPlus.
  
  		De spatie maak je met \tiSpace$=$\tiALPHA\tiZero.
\end{itemize}

\end{frame}

\begin{frame}
\frametitle{Strings}
\framesubtitle{Voorbeelden}

	\begin{ticalc}[6.5cm]
		\qt HELLO\,WORLD\qt\>Str1\\
		HELLO\,WORLD\\
		Str1+\qt ?\qt
		\visible<2->{\\ HELLO\,WORLD?}
	\end{ticalc}

	\pause%1
	\visible<2->{De tekst \tifonttxt{HELLO\,WORLD} wordt opgeslagen in \tifonttxt{Str1}.
				 Vervolgens voegen we er een vraagteken, \tifonttxt{?}, aan toe en weergeven we de uitkomst.}
	
	\vspace{8pt}
	\pause%2
	\begin{ticalc}[6.5cm]
		\qt HELLO\,WORLD\qt\>Str1\\
		HELLO\,WORLD\\
		\qt Str1\,=\,\qt+Str1
		\visible<4->{\\ Str1\,=\,HELLO\,WORLD}
	\end{ticalc}

	\pause%3
	\visible<4->{Merk op dat we ook \emph{functies en variabelen} tussen quotatie-tekens kunnen zetten,
				om de naam van die functie of variabele om te zetten naar een string.}

\end{frame}

\begin{frame}
\frametitle{Strings}
\framesubtitle{Opdrachten}

	\begin{minipage}{0.35\textwidth}%
		\begin{ticalc}[3.25cm]
			2+2\\
			\hfill 4\\
			2+2\\
			\hfill 5
		\end{ticalc}		
	\end{minipage}%
	~
	\begin{minipage}{0.62\textwidth}%
		Hoe kun je deze tekst op je scherm weergeven?
		
		\visible<2->{De eerste regel is de standaard output van de rekenmachine voor de som \tifonttxt{2+2}.
		De tweede regel is een string met genoeg spaties om \tifonttxt{5} rechts te zetten.}
	\end{minipage}%
	
	\vspace{12pt}	
	\begin{minipage}{0.35\textwidth}%
		\begin{ticalc}[3.25cm]
			\qt Hi\qt\>Str1 \\
			\qt 3*2=\qt\>Str2 \\
			Str1+\qt ,\,Kevin\qt
			\visible<3->{\\ Hi,\,Kevin}\\
			Str2+6
			\visible<4->{\\ ERR:\,DATA\,TYPE}
		\end{ticalc}
	\end{minipage}%
	~
	\begin{minipage}{0.62\textwidth}%
		Wat is de output van je rekenmachine voor deze som? En waarom?
				
		\visible<4->{\tifonttxt{6} is een getal, niet een string.
		Je kunt een string en een getal niet bij elkaar optellen.
		\inlineticalc{Str2+\qt 6\qt} had wel gewerkt.}
	\end{minipage}%
	
	\pause%1
	\pause%2
	\pause%3
\end{frame}

\subsection{Overige}

\begin{frame}
\frametitle{Andere datatypen}

Andere datatypen verdienen ook een korte benoeming:
\begin{itemize}
  \pause%1
  \item \textbf{Lists.} Een collectie getallen.
  
  		Nuttig om een berekening in 1 keer op heel veel getallen tegelijk uit te voeren.
  		Ook kun je gebruik maken van functies van de rekenmachine, zoals \tifonttxt{mean} (gemiddelde), op lijsten.
  \pause%2
  \item \textbf{Matrix.} Een soort 2D list. Geen middelbare school wiskunde.
  \pause%3
  \item \textbf{Picture.} Je kunt met de functies in het \tiDRAW-menu tekeningen op je rekenmachine maken.
  		Deze kun je in \tifonttxt{Pic\#} variabelen opslaan.
  \pause%4
  \item \textbf{Graph.} Je kunt alle functies die je bij \tiYEquals\, hebt staan opslaan in een \tifonttxt{GDB\#} variabele
  		(GDB = Graph Database).
  		
  		Daarmee kun je op een later tijdstip de functies/grafieken weer tevoorschijn toveren.
\end{itemize}

Deze datatypen/variabelen zullen we deels later behandelen.

\end{frame}


