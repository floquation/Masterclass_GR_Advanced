\section{Variabelen en Datatypes}
\subsection{Getallen}

\begin{frame}
\frametitle{\tifontbigtxt{Ans}}

\visible<2>{\ticalcfig{\ticalcfigCircleSecond\ticalcfigCircleEnter}}
\visible<3>{\ticalcfig{\ticalcfigCircleSecond\ticalcfigCircle[\ticalcfigCircleNumButtonsize]{\ticalcfigCircleColFour}{0.055}}}

\begin{itemize}
  \item \lenitem{Wanneer je een som intyped op je rekenmachine, zie je bijvoorbeeld:
\begin{ticalc}[3.25cm]
	\sqrt(3\sq+4\sq)\\
	\hfill 5
\end{ticalc}}
\pause%1
  \item \lenitem{Met \tiENTER\,herhalen we de laatste som.\\
  		Met \tiENTRY=\tiSecond\tiENTER\,halen we de som terug op ons scherm.}
\pause%2
  \item Met \tiANS\tiENTER\,printen we ook hetzelfde antwoord.
		\tifonttxt{Ans} is een variabele waarin het antwoord van de vorige bewerking is opgeslagen.
\pause%3
  \item Bijvoorbeeld, wat is hiervan de uitkomst?
 
\begin{ticalc}[3.25cm]
	\sqrt(3\sq+4\sq)\\
	\hfill 5\\
	(Ans-4)\^Ans+1
	\visible<5->{\\\hfill 2}
\end{ticalc}
\pause%4
\end{itemize}
\end{frame}


\subsection{Strings}



\subsection{Overige}



