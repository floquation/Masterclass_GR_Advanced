%%%%%%%%%%%%%%
%% PACKAGES %%
%%%%%%%%%%%%%%

\usepackage{fontspec}
%     \usepackage{libertineotf}

\usepackage{xcolor,graphicx}

\usepackage{sty/onimage}
	\usetikzlibrary{positioning}


\usepackage{hyphenat} % Can allow wordbreaks anywhere using -> \hyphenation{none}

% \usepackage[english]{babel}
% \usepackage[latin1]{inputenc}
% 
% % font definitions, try \usepackage{ae} instead of the following
% % three lines if you don't like this look
% \usepackage{mathptmx}
% \usepackage[scaled=.90]{helvet}
% \usepackage{courier}

\usepackage{ifluatex,ifxetex}

% \usepackage[T1]{fontenc}

\usepackage{xargs} % Allows for more than 1 optional argument in a command
\usepackage{xifthen}% provides \isempty test

%%%%%%%%%%%%%%
%% COMMANDS %%
%%%%%%%%%%%%%%

%% FIGURES %%

% http://tex.stackexchange.com/questions/53998/beamer-how-text-wrapping-around-a-graphic-right-aligned
\newcommand{\lenitem}[2][.8\linewidth]{\parbox[t]{#1}{\strut #2\strut}} % Wrap text in a fixed-size parbox environment.

    
    
    
%% TI Calculator top-right corner %%

\def\ticalcfigCircleButtonsize{0.021\textheight}
\def\ticalcfigCircleNumButtonsize{0.0255\textheight}
\def\ticalcfigCircleEnterButtonsize{0.028\textheight}
\def\ticalcfigCircleBigButtonsize{0.05\textheight}
% #1 (optional) = radius of circle
% #2 (obligatory) = x of circle center
% #3 (obligatory) = y of circle center
\newcommand{\ticalcfigCircle}[3][\ticalcfigCircleButtonsize]{\draw [green, line width=1pt] (#2,#3) circle [radius=#1];}
\def\ticalcfigCircleColOne{0.130}
\def\ticalcfigCircleColTwo{0.316}
\def\ticalcfigCircleColThree{0.502}
\def\ticalcfigCircleColFour{0.690}
\def\ticalcfigCircleColFive{0.878}
\def\ticalcfigCircleAlpha{\ticalcfigCircle{\ticalcfigCircleColOne}{0.739}}
\def\ticalcfigCircleSecond{\ticalcfigCircle{\ticalcfigCircleColOne}{0.816}}
\def\ticalcfigCircleEnter{\ticalcfigCircle[\ticalcfigCircleEnterButtonsize]{0.868}{0.128}}
\def\ticalcfigCircleLeft{\ticalcfigCircle{0.695}{0.79}}
\def\ticalcfigCircleDown{\ticalcfigCircle{0.815}{0.733}}
\def\ticalcfigCircleRight{\ticalcfigCircle{0.916}{0.79}} % This circle is shifted to the left ... Otherwise it would fall out of the image, which would shift the entire image to the left on the slide. TODO: Better alternative?
\def\ticalcfigCircleUp{\ticalcfigCircle{0.815}{0.85}}
\def\ticalcfigCircleArrowkeys{\ticalcfigCircle[\ticalcfigCircleBigButtonsize]{0.815}{0.79}}


% #1 (optional) = figure height
% #2 (obligatory) = body of tikzonimage environment. Use \ticalcfigCircle zero or more times.
% #3 (optional) = activate grid for debugging? Empty = no, Something = yes.
\newcommandx{\ticalcfig}[3][1=0.5\textheight, 3=]{
\begin{tikzpicture}[overlay,remember picture]
	\node[] (TR) at (current page.north east){ };
	\node [anchor=north east,xshift=0.05cm,yshift=-1.4cm] at (TR)
	{\ifthenelse{\isempty{#3}}{% if #3 is empty
		\begin{tikzonimage}[height=#1]{TI84_buttons.png}
			#2
		\end{tikzonimage}
	}{% if #3 is not empty
		\begin{tikzonimage}[height=#1]{TI84_buttons.png}[tsx/show help lines]
			#2
		\end{tikzonimage}
	}};
\end{tikzpicture}
}



%% FONT %%

% This heavily uses the 'graphicx' package.
\def\tidefinechars{\catcode`\;12 \catcode`\:12 \catcode`\!12 \catcode`\?12 \catcode`\^^M12 \catcode`\ 12%
	\def\>{-\llap{\lower0.122em\hbox{`}\llap{\lower0.366em\hbox{\reflectbox{`}}\kern0.122em }\kern-0.122em}}% Sto ->
	\def\!{=\llap{/}}% Unequal
	\def\sqrt{\XeTeXglyph95\llap{\lower0.5em\hbox{`}\kern0em}\llap{\lower-0.377em\hbox{\scalebox{0.5}[1]{-}}\kern-0.170em}}% Sqrt
	\def\sq{\XeTeXglyph101}% Square (2)
	\def\cube{\XeTeXglyph102}% Cube (3)
	\def\deg{\XeTeXglyph100}% degree symbol
	\def\Delta{\XeTeXglyph153}% Delta
	\def\theta{\XeTeXglyph154}% theta
	\def\Sigma{\XeTeXglyph155}% Sigma
	\def\Omega{\XeTeXglyph156}% Omega
	\def\alpha{\XeTeXglyph157}% alpha
	\def\beta{\XeTeXglyph158}% beta
	\def\gamma{\XeTeXglyph159}% gamma
	\def\delta{\XeTeXglyph160}% delta
	\def\eps{\XeTeXglyph161}% epsilon
	\def\ci{\XeTeXglyph162}% complex i / iota
	\def\iota{\XeTeXglyph162}% complex i / iota
	\def\lambda{\XeTeXglyph163}% lambda
	\def\mu{\XeTeXglyph164}% mu
	\def\pi{\XeTeXglyph165}% pi
	\def\rho{\XeTeXglyph166}% rho
	\def\sigma{\XeTeXglyph167}% sigma
	\def\tau{\XeTeXglyph168}% tau
	\def\phi{\XeTeXglyph169}% phi
	\def\xi{\XeTeXglyph170}% xi
	\def\ldots{\XeTeXglyph171}% ldots
	\def\percent{\XeTeXglyph8}% %
	\def\dollar{\XeTeXglyph7}% $
	\def\and{\XeTeXglyph9}% &
	\def\function{\XeTeXglyph0}% f with black-white inverted
	\def\~{\XeTeXglyph97}% ~
	\def\^{\XeTeXglyph65}% ^
}

\newfontfamily\tifont[Path=font/]{TI83_font.ttf}
\newcommand{\tifonttxt}[1]{{\tifont\tidefinechars\scriptsize#1}}
\newcommand{\tifontbigtxt}[1]{{\tifont\tidefinechars\normalsize#1}}

\newfontfamily\TIbutton[Path=font/]{TI83_button.ttf}
\DeclareTextFontCommand{\TI}{\TIbutton}
\ifluatex
 \def\tibutton#1{\TI{\directlua{fonts.otf.char("#1")}}}
 \newcommand{\tibuttonnum}[1]{\TI{#1}}% Does not work! I use XeTeX, I don't know about Lua.
\fi
\ifxetex
 \def\tibutton#1{\TI{\XeTeXglyph\the\XeTeXglyphindex"#1"}}
%  \def\tibuttonnum#1{\TI{\XeTeXglyph#1}}
 \newcommand{\tibuttonnum}[1]{\TI{\XeTeXglyph#1}}
\fi

\newsavebox{\selvestebox}
\newenvironment{ticalc}[1][1=\dimexpr\columnwidth-2\fboxsep\relax]{%
	% Define environment
	\tifont\tidefinechars\scriptsize%\addfontfeature{LetterSpace=0.0}
	\begin{lrbox}{\selvestebox}%
	\begin{minipage}{#1}
	\begin{flushleft}
}{
	\end{flushleft}
	\end{minipage}\end{lrbox}%
	\begin{centering}
	\fboxrule1pt\fboxsep2pt
	\fcolorbox{black}{green!30!black!25}{
		\hyphenation{none}\usebox{\selvestebox}
	}
	\end{centering}
}

\newcommand{\inlineticalc}[1]{{%
	% Define environment
	\tifont\tidefinechars\scriptsize%\addfontfeature{LetterSpace=0.0}
	\fboxrule0.8pt\fboxsep1pt
	\fcolorbox{black}{green!30!black!25}{
		#1
	}
}}





    
%%%%%%%%%%%%%%%%%%
%% TI Shortcuts %%
%%%%%%%%%%%%%%%%%%

% Test/Logic
\newcommand{\tiGT}			{\tibuttonnum{20}}
\newcommand{\tiMGT}			{\tibuttonnum{21}}
\newcommand{\tiGTO}			{\tibuttonnum{22}}

% Menu buttons
\newcommand{\tiQUIT}		{\tibuttonnum{24}}
\newcommand{\tiINS}			{\tibuttonnum{25}}
\newcommand{\tiALOCK}		{\tibuttonnum{26}}
\newcommand{\tiLINK}		{\tibuttonnum{27}}
\newcommand{\tiLIST}		{\tibuttonnum{28}}
\newcommand{\tiTEST}		{\tibuttonnum{29}}
\newcommand{\tiANGLE}		{\tibuttonnum{30}}
\newcommand{\tiDRAW}		{\tibuttonnum{31}}
\newcommand{\tiDISTR}		{\tibuttonnum{32}}
\newcommand{\tiMATRIX}		{\tibuttonnum{33}}
\newcommand{\tiMATRX}		{\tibuttonnum{52}}
\newcommand{\tiMATRXBut}	{\tibuttonnum{112}}
\newcommand{\tiMEM}			{\tibuttonnum{47}} % Memory
\newcommand{\tiCATALOG}		{\tibuttonnum{49}} % Catalog
\newcommand{\tiFINANCE}		{\tibuttonnum{56}} % Finance
\newcommand{\tiMODE}		{\tibuttonnum{93}}
\newcommand{\tiSTAT}		{\tibuttonnum{103}}
\newcommand{\tiAPPS}		{\tibuttonnum{110}}
\newcommand{\tiMATH}		{\tibuttonnum{111}}
\newcommand{\tiPRGM}		{\tibuttonnum{113}}
\newcommand{\tiVARS}		{\tibuttonnum{114}}
\newcommand{\tiYVARS}		{\tibuttonnum{212}} % Y-VARS

% Formula buttons (sqrt, etc.)
\newcommand{\tiSIN}			{\tibuttonnum{122}}
\newcommand{\tiCOS}			{\tibuttonnum{123}}
\newcommand{\tiTAN}			{\tibuttonnum{124}}
\newcommand{\tiASIN}		{\tibuttonnum{34}}
\newcommand{\tiACOS}		{\tibuttonnum{35}}
\newcommand{\tiATAN}		{\tibuttonnum{36}}
\newcommand{\tipi}			{\tibuttonnum{37}} % Small pi with [] around it
\newcommand{\tiPI}			{\tibuttonnum{51}} % Capital PI (Product sign)
\newcommand{\tiSIGMA}		{\tibuttonnum{54}} % Capital sigma (Sum sign)

\newcommand{\tiLCB}			{\tibuttonnum{40}} % Left curly bracket  {
\newcommand{\tiLPAREN}		{\tibuttonnum{40}} % Left Parenthesis (=LCB)
\newcommand{\tiRCB}			{\tibuttonnum{41}} % Right curly bracket }
\newcommand{\tiRPAREN}		{\tibuttonnum{41}} % Right Parenthesis (=RCB)
\newcommand{\tiLSB}			{\tibuttonnum{43}} % Left square bracket:  [
\newcommand{\tiRSB}			{\tibuttonnum{44}} % Right square bracket: ]
\newcommand{\tiLRB}			{\tibuttonnum{133}} % Left round bracket: (
\newcommand{\tiRRB}			{\tibuttonnum{134}} % Right round bracket: )

\newcommand{\tiEE}			{\tibuttonnum{39}} % 1E2 = 100
\newcommand{\tiTENPOWER}	{\tibuttonnum{42}} % 10^x
\newcommand{\tiEPOWER}		{\tibuttonnum{45}} % e^x
\newcommand{\tiXInv}		{\tibuttonnum{121}} % x^-1 -> Inverse of x
\newcommand{\tiLOG}			{\tibuttonnum{141}} % log
\newcommand{\tiLN}			{\tibuttonnum{151}} % ln

\newcommand{\tiNeg}			{\tibuttonnum{174}} % (-) -> negative number
\newcommand{\tiDot}			{\tibuttonnum{173}} % .
\newcommand{\tii}			{\tibuttonnum{57}} % Imaginary number (button)
\newcommand{\tiiScreen}		{\tibuttonnum{185}} % Imaginary number (on-screen)

\newcommand{\tiTimes}		{\tibuttonnum{145}} % a x b -> product
\newcommand{\tiProduct}		{\tibuttonnum{145}} % a x b -> product
\newcommand{\tiMultiply}	{\tibuttonnum{145}} % a x b -> product
\newcommand{\tiMinus}		{\tibuttonnum{155}} % a - b -> subtraction
\newcommand{\tiPlus}		{\tibuttonnum{165}} % a + b -> addition
\newcommand{\tiDivide}		{\tibuttonnum{135}} % a / b -> division
\newcommand{\tiDivision}	{\tibuttonnum{135}} % a / b -> division
\newcommand{\tiOver}		{\tibuttonnum{135}} % a / b -> division
\newcommand{\tiPower}		{\tibuttonnum{125}} % a ^ b -> power

% Numbers
\newcommand{\tiZero}		{\tibuttonnum{172}}
\newcommand{\tiOne}			{\tibuttonnum{162}}
\newcommand{\tiTwo}			{\tibuttonnum{163}}
\newcommand{\tiThree}		{\tibuttonnum{164}}
\newcommand{\tiFour}		{\tibuttonnum{152}}
\newcommand{\tiFive}		{\tibuttonnum{153}}
\newcommand{\tiSix}			{\tibuttonnum{154}}
\newcommand{\tiSeven}		{\tibuttonnum{142}}
\newcommand{\tiEight}		{\tibuttonnum{143}}
\newcommand{\tiNine}		{\tibuttonnum{144}}

% Other buttons
\newcommand{\tiRCL}			{\tibuttonnum{46}} % Recall variable
\newcommand{\tiOFF}			{\tibuttonnum{48}}
\newcommand{\tiON}			{\tibuttonnum{171}}
\newcommand{\tiONHALT}		{\tibuttonnum{147}} % ON/HALT
\newcommand{\tiANS}			{\tibuttonnum{61}}
\newcommand{\tiDEL}			{\tibuttonnum{94}}
\newcommand{\tiCLEAR}		{\tibuttonnum{115}}

\newcommand{\tiLeft}		{\tibuttonnum{95}}
\newcommand{\tiUp}			{\tibuttonnum{96}}
\newcommand{\tiRight}		{\tibuttonnum{97}}
\newcommand{\tiDown}		{\tibuttonnum{104}}

\newcommand{\tiSpace}		{\tibuttonnum{10}} % Visible spacebar
\newcommand{\tiSpaceButton}	{\tibuttonnum{50}} % Visible spacebar with [] around it

\newcommand{\tiSecond}		{\tibuttonnum{92}} % 2nd
\newcommand{\tiALPHA}		{\tibuttonnum{101}}

\newcommand{\tiENTER}		{\tibuttonnum{175}}
\newcommand{\tiENTRY}		{\tibuttonnum{62}}
\newcommand{\tiSOLVE}		{\tibuttonnum{63}}

\newcommand{\tiXTn}			{\tibuttonnum{102}} % X, T, Theta, n

\newcommand{\tiTRIGGER}		{\tibuttonnum{146}}

\newcommand{\tiSTO}			{\tibuttonnum{161}} % STO> button (store number into a var)

\newcommand{\tiCONT}		{\tibuttonnum{209}}

\newcommand{\tin}			{\tibuttonnum{213}} % small n (parameter of recursion)
\newcommand{\tiunmOne}		{\tibuttonnum{214}} % small u_n-1 (parameter of recursion)
\newcommand{\tivnmOne}		{\tibuttonnum{215}} % small v_n-1 (parameter of recursion)

% List
\newcommand{\tiLOne}		{\tibuttonnum{71}}
\newcommand{\tiLTwo}		{\tibuttonnum{72}}
\newcommand{\tiLThree}		{\tibuttonnum{73}}
\newcommand{\tiLFour}		{\tibuttonnum{74}}
\newcommand{\tiLFive}		{\tibuttonnum{75}}
\newcommand{\tiLSix}		{\tibuttonnum{76}}

% F buttons
\newcommand{\tiFOne}		{\tibuttonnum{65}}
\newcommand{\tiFTwo}		{\tibuttonnum{66}}
\newcommand{\tiFThree}		{\tibuttonnum{67}}
\newcommand{\tiFFour}		{\tibuttonnum{68}}
\newcommand{\tiFFive}		{\tibuttonnum{69}}

% Graphing
\newcommand{\tiYEquals}		{\tibuttonnum{82}} % Y=
\newcommand{\tiWINDOW}		{\tibuttonnum{83}}
\newcommand{\tiZOOM}		{\tibuttonnum{84}}
\newcommand{\tiTRACE}		{\tibuttonnum{85}}
\newcommand{\tiGRAPH}		{\tibuttonnum{86}}
\newcommand{\tiSTATPLOT}	{\tibuttonnum{15}}
\newcommand{\tiTBLSET}		{\tibuttonnum{16}}
\newcommand{\tiFORMAT}		{\tibuttonnum{17}}
\newcommand{\tiCALC}		{\tibuttonnum{18}}
\newcommand{\tiTABLE}		{\tibuttonnum{19}}

\newcommand{\tiBarGraph}	{\tibuttonnum{180}} % Icon for a bar graph
\newcommand{\tiLineGraph}	{\tibuttonnum{181}} % Icon for a line graph
\newcommand{\tiLinearGraph}	{\tibuttonnum{182}} % Linear line graph
\newcommand{\tiSpreadGraph}	{\tibuttonnum{183}} % Spread graph
\newcommand{\tiDoubleSpreadGraph}{\tibuttonnum{184}} % Spread graph with quadrants

\newcommand{\tiDotGraph}	{\tibuttonnum{5}} % Graph icon showing separate dots

\newcommand{\tiGraphIsLine}		{\tibuttonnum{201}} % Normal graph line
\newcommand{\tiGraphIsThick}	{\tibuttonnum{202}} % Thick line graph
\newcommand{\tiGraphIsFillUp}	{\tibuttonnum{203}} % Fill above the line
\newcommand{\tiGraphIsFillDown}	{\tibuttonnum{204}} % Fill beneath the line
\newcommand{\tiGraphIsLineTracer}{\tibuttonnum{205}} % Normal graph line + tracer O
\newcommand{\tiGraphIsTracer}	{\tibuttonnum{206}} % Graph is just a tracer
\newcommand{\tiGraphIsDotLine}	{\tibuttonnum{207}} % Normal graph line, but less sampling points

% Other characters
\newcommand{\tiitphat}		{\tibuttonnum{169}} % ^p italic
\newcommand{\tiphat}		{\tibuttonnum{217}} % ^p
\newcommand{\tiL}			{\tibuttonnum{187}} % L (from List->Ops)
\newcommand{\tiN}			{\tibuttonnum{188}} % N
\newcommand{\tiF}			{\tibuttonnum{189}} % F
\newcommand{\tiFBold}		{\tibuttonnum{190}} % F (bold)
\newcommand{\tiE}			{\tibuttonnum{196}} % E
\newcommand{\tiEBold}		{\tibuttonnum{199}} % E (bold)
\newcommand{\tiQuote}		{\tibuttonnum{197}} % ''
\newcommand{\tiStar}		{\tibuttonnum{198}} % * (a star)
\newcommand{\tiIPerc}		{\tibuttonnum{200}} % I%
\newcommand{\tiEx}			{\tibuttonnum{220}} % E[x]
\newcommand{\tiEy}			{\tibuttonnum{221}} % E[y]
\newcommand{\tiExAlt}		{\tibuttonnum{222}} % E[x] (other font)
\newcommand{\tiEyAlt}		{\tibuttonnum{223}} % E[y] (other font)


% Cursor
\newcommand{\tiCursor}		{\tibuttonnum{7}} % Normal Cursor
\newcommand{\tiCursorAlpha}	{\tibuttonnum{186}} % Alpha Cursor
\newcommand{\tiCursorSecond}{\tibuttonnum{192}} % 2nd Cursor

% Other icons
\newcommand{\tiNewPage}		{\tibuttonnum{59}} % New page icon
\newcommand{\tiPNewPage}	{\tibuttonnum{79}} % New page icon with a P inside

\newcommand{\tiLandscape}	{\tibuttonnum{179}} % No idea. Looks like a landscape.

\newcommand{\tiCDot}		{\tibuttonnum{130}} % A round dot (Circular Dot)

\newcommand{\tiFourArrow}	{\tibuttonnum{210}} % Arrows in all four directions pointing outwards and a dot in the middle

\newcommand{\tiMatrixDots}	{\tibuttonnum{6}}

\newcommand{\tiArrow}		{\tibuttonnum{4}} % Arrow to right
\newcommand{\tiReturnArrow}	{\tibuttonnum{193}} % Return (arrow to bottom left at right angle)



%% Background alignment lines %%

\usebackgroundtemplate%
{%
\begin{tikzpicture}[overlay,remember picture,opacity=0.5]
\node[] (A) at (current page.south east){ };
\node[] (B) at (current page.north west){ };
\node[] (C) at (current page.south west){ };
\node[] (D) at (current page.north east){ };
\node[red] (CC) at (current page.center){*};
% \draw[step=.5cm,gray,thin] (A) grid (B);     % Use 0.5cm will be easier to find coordinates (xx,yy)
\draw[step=.5cm,gray,thin] (C) grid (D);     % Use 0.5cm will be easier to find coordinates (xx,yy)
\end{tikzpicture}
}