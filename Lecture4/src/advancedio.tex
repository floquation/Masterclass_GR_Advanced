\section{Advanced I/O}

\begin{frame}
\frametitle{Basic I/O}

\begin{itemize}
  \item<1-> We hebben al de volgende I/O (input/output) functies gezien:
  \begin{itemize}
    \item<2-> \tifonttxt{Disp} voor simpele output
    \item<3-> \tifonttxt{Prompt} voor simpele input
    \item<4-> \tifonttxt{Input} voor iets uitgebreidere input
  \end{itemize}
  \item<5-> Maar er zijn er nog veel meer!
\end{itemize}

\end{frame}

\subsection{Menu}

\begin{frame}
\frametitle{Advanced input: Menu's}

\hspace{-1cm}
\begin{minipage}{\textwidth}
\begin{itemize}
  \item<1-> Indien er voor \tifonttxt{Prompt} of \tifonttxt{Input} slechts een handjevol waardes is toegestaan
    (bijv. 1, 2 of 3, zoals bij de risk dobbelstenen),
    dan is \tifonttxt{Menu} zeer nuttig.
  \item<2-> De gebruiker kan dan tussen deze toegestane waardes kiezen.
  \item<3-> Dit hoeven overigens geen getallen te zijn, de gebruiker kan ook bijv. kiezen tussen \tifonttxt{AFLEIDEN} en \tifonttxt{INTEGREREN}.
  \begin{itemize}
    \item<4-> Afhankelijk van de keuze doet je \tifonttxt{prgm} wat anders.
  \end{itemize}
  \item<5-> \lenitem[0.8\linewidth]{De keuze die de gebruiker maakt werkt als een \inlineticalc{Goto} statement:}
  \begin{itemize}
    \item<6-> \lenitem[0.75\linewidth]{De code springt naar een \inlineticalc{Lbl} statement,
      zoals gegeven in het argument van \tifonttxt{Menu}}
  \end{itemize}
\end{itemize}
\end{minipage}

\begin{tikzpicture}[overlay,remember picture]
	\node[yshift=0.6cm] (BL) at (current page.south east){ };
	\node [anchor=south east,xshift=0.195cm] at (BL)
	{%
	\only<1-4>{%
		\begin{ticalc}
			\select{CTL}\,I/O\,EXEC \\%
			9\:Lbl \\%
			0\:Goto \\%
			A\:IS>( \\%
			B\:DS<( \\%
			\selectitem{C\:}Menu( \\%
			D\:prgm \\%
			E\arrowdown Return
		\end{ticalc}
	}%
	\only<5-6>{%
		\begin{ticalc}
			Menu(\\%
			\hline\\%
			(\qt title\qt\comma\qt text\one\qt\\%
			\hbox{\comma \only<-5>{label\one}\only<6->{\select{label\one}}[\comma \.\.\.\comma\qt}{ }
			text7\qt\comma label7])\\%
			\hfill\\%
			\hfill\\%
			\hline
		\end{ticalc}
	}%
	};
\end{tikzpicture}

\end{frame}


\begin{frame}
\frametitle{\tifonttxt{Menu} voorbeeld}

\begin{minipage}{0.48\textwidth}
\begin{ticalc}
	PROGRAM\:FRSTMENU\\%
	\:\hbox{Menu(\qt WHAT\,PIE?}{ }
	\hbox{\qt\comma\qt APPLE\qt\comma0\comma\qt ABR}{ }
	ICOT\qt\comma\one\comma\qt BLUEBER\\%
	RY\qt\comma2)\\%
	\:Lbl\,0\\%
	\:Disp\,\qt YOU\,CHOSE\,APPLE\qt\\%
	\:Stop\\%
	\:Lbl\,\one\\%
	\:Disp\,\qt YOU\,CHOSE\,ABRICOT\qt\\%
	\:Stop\\%
	\:Lbl\,2\\%
	\:Disp\,\qt YOU\,CHOSE\,BLUEBERRY\qt
\end{ticalc}
\end{minipage}
\begin{minipage}{0.48\textwidth}
\only<2>{%
\begin{ticalc}%
	\select{WHAT\,PIE?}\\%
	\selectitem{\one\:}APPLE\\%
	2\:ABRICOT\\%
	3\:BLUEBERRY%
\end{ticalc}%
}%

\only<1>{%
\begin{ticalc}%
	prgmFRSTMENU\\%
	\hfill\\%
	\hfill\\%
	\hfill%
\end{ticalc}%
}%

\only<3>{%
\begin{ticalc}%
	prgmFRSTMENU\\%
	YOU\,CHOSE\,APPLE\\%
	\hfill Done\\%
	\hfill
\end{ticalc}%
}%
\end{minipage}

\end{frame}






\subsection{Output}

\begin{frame}
\frametitle{Advanced output: formateer je scherm}




\end{frame}



\begin{frame}
\frametitle{Nuttig samen met \tifonttxt{Output}: \tifonttxt{ClrHome}}




\end{frame}