\section{Advanced I/O}

\begin{frame}
\frametitle{Basic I/O}

\begin{itemize}
  \item<1-> We hebben al de volgende I/O (input/output) functies gezien:
  \begin{itemize}
    \item<2-> \tifonttxt{Disp} voor simpele output
    \item<3-> \tifonttxt{Prompt} voor simpele input
    \item<4-> \tifonttxt{Input} voor iets uitgebreidere input
  \end{itemize}
  \item<5-> Maar er zijn er nog veel meer!
\end{itemize}

\end{frame}

\subsection{Menu}

\begin{frame}
\frametitle{Advanced input: Menu's}

\hspace{-1cm}
\begin{minipage}{\textwidth}
\begin{itemize}
  \item<1-> Indien er voor \tifonttxt{Prompt} of \tifonttxt{Input} slechts een handjevol waardes is toegestaan
    (bijv. 1, 2 of 3, zoals bij de risk dobbelstenen),
    dan is \tifonttxt{Menu} zeer nuttig.
  \item<2-> De gebruiker kan dan tussen deze toegestane waardes kiezen.
  \item<3-> Dit hoeven overigens geen getallen te zijn, de gebruiker kan ook bijv. kiezen tussen \tifonttxt{AFLEIDEN} en \tifonttxt{INTEGREREN}.
  \begin{itemize}
    \item<4-> Afhankelijk van de keuze doet je \tifonttxt{prgm} wat anders.
  \end{itemize}
  \item<5-> \lenitem[0.8\linewidth]{De keuze die de gebruiker maakt werkt als een \inlineticalc{Goto} statement:}
  \begin{itemize}
    \item<6-> \lenitem[0.75\linewidth]{De code springt naar een \inlineticalc{Lbl} statement,
      zoals gegeven in het argument van \tifonttxt{Menu}}
  \end{itemize}
\end{itemize}
\end{minipage}

\begin{tikzpicture}[overlay,remember picture]
	\node[yshift=0.6cm] (BL) at (current page.south east){ };
	\node [anchor=south east,xshift=0.195cm] at (BL)
	{%
	\only<1-4>{%
		\begin{ticalc}
			\select{CTL}\,I/O\,EXEC \\%
			9\:Lbl \\%
			0\:Goto \\%
			A\:IS>( \\%
			B\:DS<( \\%
			\selectitem{C\:}Menu( \\%
			D\:prgm \\%
			E\arrowdown Return
		\end{ticalc}
	}%
	\only<5-6>{%
		\begin{ticalc}
			Menu(\\%
			\hline\\%
			(\qt title\qt\comma\qt text\one\qt\\%
			\hbox{\comma \only<-5>{label\one}\only<6->{\select{label\one}}[\comma \.\.\.\comma\qt}{ }
			text7\qt\comma label7])\\%
			\hfill\\%
			\hfill\\%
			\hline
		\end{ticalc}
	}%
	};
\end{tikzpicture}

\end{frame}


\begin{frame}
\frametitle{\tifonttxt{Menu} voorbeeld}

\begin{minipage}{0.48\textwidth}
\begin{ticalc}
	PROGRAM\:FRSTMENU\\%
	\:\hbox{Menu(\qt WHAT\,PIE?}{ }
	\hbox{\qt\comma\qt APPLE\qt\comma0\comma\qt ABR}{ }
	ICOT\qt\comma\one\comma\qt BLUEBER\\%
	RY\qt\comma2)\\%
	\:Lbl\,0\\%
	\:Disp\,\qt YOU\,CHOSE\,APPLE\qt\\%
	\:Stop\\%
	\:Lbl\,\one\\%
	\:Disp\,\qt YOU\,CHOSE\,ABRICOT\qt\\%
	\:Stop\\%
	\:Lbl\,2\\%
	\:Disp\,\qt YOU\,CHOSE\,BLUEBERRY\qt
\end{ticalc}
\end{minipage}
\begin{minipage}{0.48\textwidth}
\only<2>{%
\begin{ticalc}%
	\select{WHAT\,PIE?}\\%
	\selectitem{\one\:}APPLE\\%
	2\:ABRICOT\\%
	3\:BLUEBERRY%
\end{ticalc}%
}%

\only<1>{%
\begin{ticalc}%
	prgmFRSTMENU\\%
	\hfill\\%
	\hfill\\%
	\hfill%
\end{ticalc}%
}%

\only<3>{%
\begin{ticalc}%
	prgmFRSTMENU\\%
	YOU\,CHOSE\,APPLE\\%
	\hfill Done\\%
	\hfill
\end{ticalc}%
}%
\end{minipage}

\end{frame}






\subsection{Output}

\begin{frame}
\frametitle{Advanced output: formateer je scherm met \tifonttxt{Output}}

\begin{itemize}
  \item<1-> \tifonttxt{Disp} weergeeft een string van links naar rechts op de eerstvolgende regel
  \item<2-> \tifonttxt{Output} weergeeft een string op een plek die je zelf kan kiezen
\end{itemize}

\begin{itemize}
  \item<3-> Het scherm van je rekenmachine bestaat uit 8 rijen en 16 kolommen
  \item<4-> Probeer het volgende voorbeeld zelf:
  \begin{itemize}
    \item<5-> \tifonttxt{ABC} verschijnt links-bovenin en overschrijft ``\tifonttxt{prgmOutput}''
    \item<6-> \tifonttxt{G} verschijnt rechts-in-het-midden; \tifonttxt{HI} op de volgende regel
    \item<7-> \tifonttxt{XY} verschijnt rechts-onderin;\\ \tifonttxt{Z} valt buiten het scherm
  \end{itemize} 
\end{itemize}

\vspace{2cm}

\begin{tikzpicture}[overlay,remember picture]
	\node[yshift=0.6cm] (BL) at (current page.south east){ };
	\node [anchor=south east,xshift=0.195cm] at (BL)
	{%
	\only<4->{%
		\begin{ticalc}
			\only<-4>{prgmOutput}\only<5->{pABCOutput}\\%
			\,\\%
			\,\\%
			\hfill \visible<6->{G}\\%
			\visible<6->{HI}\\%
			\,\\%
			\,\\%
			\hfill \visible<7->{XY}
		\end{ticalc}
	}%
	};
	\node [anchor=south east,xshift=-3.5cm] at (BL)
	{%
	\only<4->{%
		\begin{ticalc}[3.75cm]
			PROGRAM\:OUTPUT\\%
			\:Output(1\comma2\comma\qt ABC\qt)\\%
			\:Output(4\comma16\comma\qt GHI\qt)\\%
			\:Output(8\comma15\comma\qt XYZ\qt)
		\end{ticalc}
	}%
	};
\end{tikzpicture}


\end{frame}



\begin{frame}
\frametitle{Nuttig samen met \tifonttxt{Output}: \tifonttxt{ClrHome}}

\begin{itemize}
  \item<1-> ``With great power, comes great responsibility''
  \item<2-> Het is handig om het scherm eerst te legen, voordat je alles gaat overschrijven.
  \item<3-> Dit is precies wat \tifonttxt{ClrHome} doet
\end{itemize}

\vspace{1.5cm}

\begin{tikzpicture}[overlay,remember picture]
	\node[yshift=0.6cm] (BL) at (current page.south east){ };
	\node [anchor=south east,xshift=0.195cm] at (BL)
	{%
	\only<3->{%
		\begin{ticalc}
			CTL\,\select{I/O}\,EXEC \\
			2\:Prompt \\
			3\:Disp \\
			4\:DispGraph \\
			5\:DispTable \\
			6\:Output( \\
			7\:getKey \\
			\selectitem{8\arrowdown} ClrHome
		\end{ticalc}
	}%
	};
	\node [anchor=south east,xshift=-3.5cm] at (BL)
	{%
	\only<4->{%
		\begin{ticalc}
			PROGRAM\:CLRHOME\\%
			\:Disp\,\qt A\qt\\%
			\:Pause\\%
			\:ClrHome\\%
			\:Output(4,4,\qt HELL O\qt)
		\end{ticalc}
	}%
	};
	\node [anchor=south east,xshift=-7.195cm] at (BL)
	{%
	\only<4>{%
		\begin{ticalc}
			2+2\pausedots\\%
			\hfill 4\\%
			2*2\\%
			\hfill 4\\%
			prgmCLRHOME\\%
			A\\%
			\,\\%
			\,
		\end{ticalc}
	}%
	\only<5->{%
		\begin{ticalc}
			\,\\%
			\,\\%
			\,\\%
			\,\,\,HELLO\\%
			\,\\%
			\,\\%
			\,\\%
			\,
		\end{ticalc}
	}%
	};
\end{tikzpicture}


\end{frame}






\subsection{num2str}

\begin{frame}
\frametitle{num2str}

\begin{itemize}
  \item<1-> Om mooie output te krijgen, is het handig om getallen naar strings te converteren
  \item<2-> Zo kun je als antwoord \inlineticalc{\sqrt(2)} laten zien,\\%
  			i.p.v. op meerdere regels:
\end{itemize}
\visible<2->{
\begin{ticalc}
	\sqrt(\\%
	\hfill 2
\end{ticalc}
}
\begin{itemize}
  \item<3-> Op de TI84 gaat dit zeer ongemakkelijk\ldots
  \item<4-> Maar met behulp van een custom functie hoeven we het slechts 1 keer te maken!
\end{itemize}

\end{frame}



\begin{frame}
\frametitle{num2str}

\begin{itemize}
  \item \tifonttxt{\theta NUM2STR} zet het getal \tifonttxt{X} om in een string \tifonttxt{Str1}.
  \item \tifonttxt{DISPSQRT} weergeeft het getal \tifonttxt{X} in een wortel-teken.
  \item<2-> Nadeel is wel: de variabelen \tifonttxt{X}, \tifonttxt{Y\sub{0}} en \tifonttxt{Str1} worden overschreven door het \tifonttxt{prgm}.
  \item<3-> \lenitem[0.6\linewidth]{Merk op: de comments geven aan wat de functie overschrijft!\\%
  			Handig als je later terug kijkt!}
\end{itemize}

\vspace{3cm}


\begin{tikzpicture}[overlay,remember picture]
	\node[yshift=0.6cm] (BR) at (current page.south east){ };
	\node[yshift=0.6cm] (BL) at (current page.south west){ };
	\node [anchor=south east,xshift=0.195cm] at (BR)
	{%
	\only<1->{%
		\begin{ticalc}[4.2cm]
			PROGRAM\:\theta NUM2STR\\%
			\:\qt IN\:\,X\\%
			\:\qt OUT\:\,Str1\\%
			\:\qt USES\:\,Y\sub{0}\\%
			\:\{X\comma X\>\tiL NSY\\%
			\:\{0\comma 1\>\tiL NSX\\%
			\:LinReg(a+bx)\,\tiL NSX\comma\\%
				\tiL NSY\comma Y\sub{0}\\%
			\:Equ\ftriangleright String(Y\sub{0}\comma Str1)\\%
			\:sub(Str1\comma 1\comma length( Str1)-3)\>Str1\\%
			\:FnOff\,0 
		\end{ticalc}
	}%
	};
	\node [anchor=south east,xshift=-4.5cm] at (BR)
	{%
	\only<1->{%
		\begin{ticalc}[4cm]
			PROGRAM\:DISPSQRT\\%
			\:REQ\:prgm\theta NUM2STR\\%
			\:Prompt\,X\\%
			\:prgm\theta NUM2STR\\%
			\:Disp\,\qt\sqrt(\qt+ Str1+\qt)\qt
		\end{ticalc}
	}%
	};
	\node [anchor=south west,xshift=-0.1cm] at (BL)
	{%
	\only<1->{%
		\begin{ticalc}[3cm]
			prgmDISPSQRT\\%
			X=?2\\%
			\sqrt(2)
		\end{ticalc}
	}%
	};
\end{tikzpicture}


\end{frame}

