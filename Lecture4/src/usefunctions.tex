\section{Functions}

\subsection{Theorie}

\begin{frame}
\frametitle{Wat is een functie?}

\begin{itemize}
  \item<1-> In ``echte'' programmeertalen op de computer heb je functies/methodes
  \item<2-> Dit zijn kleine stukjes code die iets doen wat je heel vaak nodig hebt
  \item<3-> Het gebruik van een functie zorgt ervoor dat je die stukjes code makkelijk kunt herbruiken
  \item<4-> Op de rekenmachine zijn er voorgeprogrammeerde functies:
  \begin{enumerate}
    \item<5-> \tifonttxt{abs(} berekent de absolute waarde van een getal
    \item<6-> \tifonttxt{max(} geeft de grootste waarde van twee getallen
  \end{enumerate}
  \item<7-> \underline{Algemeen}: je stopt er getallen in, en je krijgt er berekende getallen uit
\end{itemize}

\vspace{2cm}

\begin{tikzpicture}[overlay,remember picture]
	\node[yshift=0.6cm] (BL) at (current page.south east){ };
	\node [anchor=south east,xshift=0.195cm] at (BL)
	{%
	\only<5>{%
		\begin{ticalc}
			abs(\min 5)\\%
			\hfill 5\\%
			abs(6)\\%
			\hfill 6
		\end{ticalc}
	}%
	\only<6>{%
		\begin{ticalc}
			max(\min10\comma10)\\%
			\hfill 10\\%
			max(42\comma3)\\%
			\hfill 42
		\end{ticalc}
	}%
	};
	\node [anchor=south east,xshift=-3.5cm] at (BL)
	{%
	\only<5>{%
		\begin{ticalc}
			\:If\,X>0\\%
			\:Then\\%
			\:Disp\,X\\%
			\:Else\\%
			\:Disp\,\min X\\%
			\:End
		\end{ticalc}
	}%
	\only<6>{%
		\begin{ticalc}
			\:If\,X>Y\\%
			\:Then\\%
			\:Disp\,X\\%
			\:Else\\%
			\:Disp\,Y\\%
			\:End
		\end{ticalc}
	}%
	};
\end{tikzpicture}

\end{frame}




\begin{frame}
\frametitle{Je eigen functie defini\"eren}

\begin{itemize}
  \item<1->  Het is zeer nuttig om je eigen functies te defini\"eren voor kleine stukjes code die je wenst te herhalen
\end{itemize}
\begin{itemize}
  \item<2->  Hoe doe je dat?
  \visible<3->{
  \begin{itemize}
    \item Kort antwoord: het \textbf{kan niet}.
    \begin{itemize}
      \item<4-> Ok, dat is deprimerend\ldots
    \end{itemize}
    \visible<5->{\item Lang antwoord: we kunnen iets soortgelijks maken!}
  \end{itemize}
  }
\end{itemize}

\end{frame}






\begin{frame}
\frametitle{Je eigen functie defini\"eren}


\begin{itemize}
  \item<1-> Dit doen we door een los \tifonttxt{prgm} te maken, die geen I/O heeft.
  \begin{itemize}
    \item<2-> Hij vraagt de gebruiker \underline{niet} om input (e.g. \tifonttxt{Prompt})
    \item<3-> Hij weergeeft ook \underline{geen} output (e.g. \tifonttxt{Disp})
  \end{itemize}
  \item<4-> Het \tifonttxt{prgm} communiceert met andere \tifonttxt{prgm}'s door variabelen aan te passen
  \item<5-> Onderstaand (nutteloos) voorbeeld illustreert hoe we dit kunnen doen
  \item<6-> De naam van het functie-\tifonttxt{prgm} begint met een \tifonttxt{\theta} om het \tifonttxt{prgm} te onderscheiden van een ``executable''. (Optioneel.)
\end{itemize}

\vspace{2.5cm}


\begin{tikzpicture}[overlay,remember picture]
	\node[yshift=0.6cm] (BL) at (current page.south east){ };
	\node [anchor=south east,xshift=0.195cm] at (BL)
	{%
	\only<5->{%
		\begin{ticalc}
			PROGRAM\:\theta ADD1\\%
			\:X+1\>X
		\end{ticalc}
	}%
	};
	\node [anchor=south east,xshift=-3.5cm] at (BL)
	{%
	\only<5->{%
		\begin{ticalc}
			PROGRAM\:COUNT10\\%
			\:Prompt\,X\\%
			\:While\,X\le10\\%
			\:prgm\theta ADD1\\%
			\:Disp\,X\\%
			\:End
		\end{ticalc}
	}%
	};
	\node [anchor=south east,xshift=-7.195cm] at (BL)
	{%
	\only<5->{%
		\begin{ticalc}
			prgmCOUNT10\\%
			X=?6\\%
			\hfill7\\%
			\hfill8\\%
			\hfill9\\%
			\hfill10\\%
			\hfill Done
		\end{ticalc}
	}%
	};
\end{tikzpicture}


\end{frame}


\subsection{Voorbeelden}


\begin{frame}
\frametitle{Functie voorbeeld}

\begin{itemize}
  \item De volgende functie eist dat de input waarde van \tifonttxt{N} positief is.
  \item Zo niet, dan crashed het \tifonttxt{prgm} met een error.
  \item Het voordeel van hiervoor een aparte functie-\tifonttxt{prgm} gebruiken, is dat het functie-\tifonttxt{prgm} herbruikt kan worden:
  \begin{itemize}
    \item Indien er bijv. twee input waarden zijn die positief moeten zijn.
    \item Of voor andere \tifonttxt{prgm}s!
  \end{itemize}
\end{itemize}

\vspace{2.5cm}



\begin{tikzpicture}[overlay,remember picture]
	\node[yshift=0.6cm] (BL) at (current page.south east){ };
	\node [anchor=south east,xshift=0.195cm] at (BL)
	{%
	\only<1->{%
		\begin{ticalc}
			PROGRAM\:\theta MUSTGT0\\%
			\:If\,X\le0\\%
			\:Then\\%
			\:Disp\,\qt VALUE\,MUST\,BE\,>0\comma\,BUT\,WAS\:\qt\comma X\\%
			\:Stop\\%
			\:Else\\%
			\:Return\\%
			\:End
		\end{ticalc}
	}%
	};
	\node [anchor=south east,xshift=-3.5cm] at (BL)
	{%
	\only<1->{%
		\begin{ticalc}
			PROGRAM\:THROWDIC\\%
			\:\qt THROW\,N\,DICE\,Disp\,SUM\\%
			\:Prompt\,N\\%
			\:N\>X\\%
			\:prgm\theta MUSTGT0\\%
			\:Disp\,sum(randInt(1\comma6 \comma N))
		\end{ticalc}
	}%
	};
	\node [anchor=south east,xshift=-7.195cm] at (BL)
	{%
	\only<1>{%
		\begin{ticalc}
			prgmTHROWDIC\\%
			N=?6\\%
			\hfill21\\%
			\hfill Done
		\end{ticalc}
	}%
	\only<2>{%
		\begin{ticalc}
			prgmTHROWDIC\\%
			N=?\min3\\%
			VALUE\,MUST\,BE\,>0\comma\,BUT\,WAS\:\\%
			\hfill \min3\\%
			\hfill Done
		\end{ticalc}
	}%
	};
\end{tikzpicture}

\end{frame}



\begin{frame}
\frametitle{Functie awesomerest voorbeeld}

\vspace{-2.5cm}
\hspace{-1cm}
\begin{minipage}{\textwidth}
\begin{itemize}
  \item Het volgende \tifonttxt{prgm} vraagt je naar een functie (e.g. $y=x^2$).
  \item Daarna vraagt hij wat jij denkt dat de afgeleide van die functie is (e.g. $y=2x$).
  \item \lenitem{Vervolgens plot hij de \textbf{echte} afgeleide, en jouw afgeleide tegelijkertijd.}
  \item Zo kun je jouw antwoord grafisch controleren!
\end{itemize}
\end{minipage}



\begin{tikzpicture}[overlay,remember picture]
	\node[yshift=0.6cm] (BL) at (current page.south east){ };
	\node [anchor=south east,xshift=0.195cm] at (BL)
	{%
	\only<1->{%
		\begin{ticalc}
			PROGRAM\:\theta PLTDERV\\%
			\:Disp\,\qt Y\sub{1}=FUNC\\%
			\:Disp\,\qt Y\sub{2}=YOUR\,DERIV\\%
			\:FnOff\,1\comma2\\%
			\:GraphStyle(8\comma7)\\%
			\:GraphStyle(9\comma5)\\%
			\:\qt nDeriv(Y\sub{1}\comma X\comma X) \qt\>Y\sub{8}\\%
			\:Equ\ftriangleright String(Y\sub{2} \comma Str0)\\%
			\:Str0\>Y\sub{9}\\%
			\:ZoomFit
			\:DispGraph
		\end{ticalc}
	}%
	};
	\node [anchor=south east,xshift=-3.5cm] at (BL)
	{%
	\only<1->{%
		\begin{ticalc}
			PROGRAM\:PLTDERIV\\%
			\:Disp\,\qt START\,WITH\,QUOTE\:\qt\\%
			\:Disp\,\qt ORIG\,FUNC\:\qt\\%
			\:Prompt\,Y\sub{1}\\%
			\:Disp\,\qt YOUR\,DERIVATIVE\:\qt\\%
			\:Prompt\,Y\sub{2}\\%
			\:prgm\theta PLTDERV
		\end{ticalc}
	}%
	};
\end{tikzpicture}




\end{frame}





