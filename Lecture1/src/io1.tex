\section{Basic IO}

\subsection{Disp}

\begin{frame}
\frametitle{Het \tifonttxt{PRGM-I/O} menu: input \& output.}

\visible<2>{\ticalcfig{\ticalcfigCircle{\ticalcfigCircleColThree}{0.615}}} % prgm
\visible<3>{\ticalcfig{\ticalcfigCircleRight\ticalcfigCircle{\ticalcfigCircleColThree}{0.615}}} % prgm + right
\visible<4->{\ticalcfig{\ticalcfigCircle{\ticalcfigCircleColThree}{0.615}}} % prgm

We gaan nu een programma tekst laten weergeven.
\begin{itemize}
  \item<2-> \lenitem{Maak zelf een nieuw programma aan: \tifonttxt{DISP1}.}
  \item<3-> \lenitem{Open het programma (\tiPRGM\tiRight\tifonttxt{DISP1}).}
  \item<4-> \lenitem{Druk op \tiPRGM\,om het programmeermenu te openen.}
  \item<5-> \lenitem{Bij menu \tifonttxt{I/O} staat alle Input en Output.}
  \item<6-> We zijn nu geinteresseerd in \tifonttxt{Disp}: ``Display''.
\end{itemize}

\only<4>{
	\begin{ticalc}
		\select{CTL}\,I/O\,EXEC \\
		\selectitem{1\+\:}If \\
		2\:Then \\
		3\:Else \\
		4\:For( \\
		5\:While \\
		6\:Repeat( \\
		7\arrowdown End
	\end{ticalc}
}
\visible<5->{
	\begin{ticalc}
		CTL\,\select{I/O}\,EXEC \\
		1\+\:Input \\
		2\:Prompt \\
		\selectitem{3\:}Disp \\
		4\:DispGraph \\
		5\:DispTable \\
		6\:Output( \\
		7\arrowdown getKey
	\end{ticalc}
}

\end{frame}

\begin{frame}
\frametitle{Hello World!!}

\visible<1-2>{\ticalcfig{\ticalcfigCircle{\ticalcfigCircleColThree}{0.615}}} % prgm
\visible<3>{\ticalcfig{\ticalcfigCircleAlpha\ticalcfigCircleSecond}} % Alpha + 2nd
\visible<4->{\ticalcfig{\ticalcfigCircleAlpha\ticalcfigCircle{\ticalcfigCircleColFive}{0.227}}} % Quote / Mem

We maken nu een ``Hello World'' programma:
\begin{itemize}
  \item<2-> \lenitem{Kies de \tifonttxt{Disp} functie uit het \tifonttxt{I/O} menu.}
  \item<3-> \lenitem{Zet je rekenmachine op \tiALOCK, zodat je letters kunt typen,
  			zonder herhaaldelijk op \tiALPHA\, te drukken.}
  \item<4-> \lenitem{Typ vervolgens \tifonttxt{\qt HELLO\,WORLD\qt} achter \tifonttxt{Disp}.}
  \item<5-> \lenitem{Done! Execute het programma om te testen!}
\end{itemize}

\visible<2->{
	\begin{ticalc}
		CTL\,\select{I/O}\,EXEC \\
		1\+\:Input \\
		2\:Prompt \\
		\selectitem{3\:}Disp \\
		4\:DispGraph \\
		5\:DispTable \\
		6\:Output( \\
		7\arrowdown getKey
	\end{ticalc}
}
\visible<1->{
	\begin{ticalc}[5cm]
		PROGRAM\:DISP1 \\%
		\:\visible<2->{Disp\,\only<4->{\qt HELLO\,WORLD\qt}\only<3->{\tiCursorAlpha}}%
	\end{ticalc}
}

\end{frame}


\subsection{Prompt}

\begin{frame}
\frametitle{Input: \tifontbigtxt{Prompt}}

\begin{itemize}
  \item<1-> Nice\ldots Maar het is veel leuker als de rekenmachine iets meer kan dan alleen ``Hello World'' zeggen.
  \item<2-> \tifonttxt{Prompt} vertelt de rekenmachine iets: Input.
  \item<3-> Bijvoorbeeld, \inlineticalc{Prompt\,A} vraagt de gebruiker om de waarde van de variabele \tifonttxt{A}: een getal.
  \item<4-> Bijvoorbeeld, \inlineticalc{Prompt\,Str1} vraagt de gebruiker om de waarde van de variabele \tifonttxt{Str1}: een string.
\end{itemize}

\visible<2->{
	\begin{ticalc}
		CTL\,\select{I/O}\,EXEC \\
		1\+\:Input \\
		\selectitem{2\:}Prompt \\
		3\:Disp \\
		4\:DispGraph \\
		5\:DispTable \\
		6\:Output( \\
		7\arrowdown getKey
	\end{ticalc}
}

\end{frame}

\begin{frame}
\frametitle{Input: \tifontbigtxt{Prompt}}
\framesubtitle{Meerdere argumenten}

\begin{itemize}
  \item<1-> Merk op dat je meerdere argumenten aan \tifonttxt{Prompt} kunt geven.
  \item<2-> \inlineticalc{Prompt\,X\comma Y\comma Z} vraagt de gebruiker om de waarde van de variabelen \tifonttxt{X}, \tifonttxt{Y} en \tifonttxt{Z}.
  \item<3-> Dit geeft een kleiner, overzichtelijker programma dan drie aparte \tifonttxt{Prompt} commando's.
  \item<4-> Hetzelfde geldt voor \tifonttxt{Disp}.
\end{itemize}

\begin{ticalc}
	PROGRAM\:BAD \\%
	\:Prompt\,X \\%
	\:Prompt\,Y \\%
	\:Prompt\,Z
\end{ticalc}
\begin{ticalc}
	PROGRAM\:GOOD \\%
	\:Prompt\,X\comma Y\comma Z
\end{ticalc}

\end{frame}


\begin{frame}
\frametitle{``Hello World'' is zo onpersoonlijk\ldots}

We breiden ``Hello World'' uit: ``Hello {`naam'}''
\begin{itemize}
  \item<2-> Maak een nieuw programma aan: \tifonttxt{DISP2}.
  \item<3-> Vraag de gebruiker om zijn naam en sla deze op in \tifonttxt{Str1}.
  \item<4-> Print vervolgens de \tifonttxt{Str1} variabele met \tifonttxt{Disp}, samen met de tekst ``\tifonttxt{HELLO\,}''.
  \item<5-> Run het programma en laat het jouw naam weergeven. \\ Let op: Zet je naam tussen quotes (\tifonttxt{\qt}).
\end{itemize}

\visible<1->{
	\begin{ticalc}%
		CTL\,\select{I/O}\,EXEC \\%
		1\+\:Input \\%
		\selectitem{2\:}Prompt \\%
		3\:Disp \\%
		4\:DispGraph \\%
		5\:DispTable \\%
		6\:Output( \\%
		7\arrowdown getKey%
	\end{ticalc}
}
\visible<2->{
	\begin{ticalc}[5cm]
		PROGRAM\:DISP2 \\%
		\:\visible<3->{Prompt\,Str1}\\%
		\:\visible<4->{Disp\,\qt HELLO\,\qt+Str1}
	\end{ticalc}
}

\end{frame}




