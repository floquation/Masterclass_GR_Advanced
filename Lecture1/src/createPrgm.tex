\section{Hoe open je een programma?}

\begin{frame}
\frametitle{\tifontbigtxt{NEW} \tiPRGM}

\visible<1-2>{\ticalcfig{\ticalcfigCircle{\ticalcfigCircleColThree}{0.615}}} % prgm
\visible<3>{\ticalcfig{\ticalcfigCircleEnter\ticalcfigCircleRight}}
\visible<4>{\ticalcfig{\ticalcfigCircleEnter \ticalcfigCircleAlpha\ticalcfigCircleSecond}}
\visible<5>{\ticalcfig{\ticalcfigCircleSecond\ticalcfigCircle{\ticalcfigCircleColTwo}{0.81}}}

We gaan ons eerste programma aanmaken.
\begin{itemize}
  \item Druk op \tiPRGM.
  \pause %1
  \item \lenitem{Tenzij je eerder een programma hebt gemaakt, zie je alleen
  		\inlineticalc{\only<1-2>{\select{EXEC}}\only<3->{EXEC}\,EDIT\,\only<3->{\select{NEW}}\only<1-2>{NEW}}.}
  \pause %2
  \item Blader met \tiRight\,naar \tifonttxt{NEW} en druk op \tiENTER.
  \pause %3
  \item \lenitem{Typ een naam in. Een prgm naam is maximaal 8 karakters. Druk vervolgens op \tiENTER.}
	  \begin{ticalc}[3.25cm]
	  	PROGRAM\\
	  	NAME=MYPRGM01
	  \end{ticalc}
  
  Merk op dat je een \tiCursorAlpha-cursor hebt: \tiALOCK\,is geactiveerd.
  \pause %4
  \item Sluit het programma nu met \tiQUIT.
\end{itemize}
\end{frame}

\begin{frame}
\frametitle{\tifontbigtxt{EDIT} \tiPRGM}

\visible<1-2>{\ticalcfig{\ticalcfigCircle{\ticalcfigCircleColThree}{0.615}}}
\visible<3>{\ticalcfig{\ticalcfigCircleRight}}
\visible<4>{\ticalcfig{\ticalcfigCircleDown\ticalcfigCircleEnter}}
\visible<5>{\ticalcfig{}}
\visible<6>{\ticalcfig{\ticalcfigCircleAlpha}}

Om het programma nu weer te openen, doe:
\begin{itemize}
  \item Druk op \tiPRGM.
  \pause %1
  \item \lenitem{Je ziet nu een lijst met alle programma's die je kunt uitvoeren.}
	  \begin{ticalc}[3.25cm]
	  	\only<1-2>{\select{EXEC}}\only<3->{EXEC}\,\only<3->{\select{EDIT}}\only<1-2>{EDIT}\,NEW
	  	\selectitem{1\:}MYPRGM01
	  \end{ticalc}
  \pause %2
  \item Blader met \tiRight\,naar \tifonttxt{EDIT}.
  \pause %3
  \item Hier zie je dezelfde lijst. Gebruik \tiDown\, om naar je programma te bladeren en druk op \tiENTER.
  \pause %4
  \item Als alternatief, kun je ook het nummer intoetsen wat voor je programma staat. Dit is een hotkey om je programma te openen.
  \pause %5
  		Ook kun je met \tiALPHA\, de eerste letter van je programmanaam intoetsen om snel je programma te vinden
  		(indien je later een grotere lijst met programma's hebt dan 1).
\end{itemize}
\end{frame}




\begin{frame}
\frametitle{Deleten/Archiveren van een \tiPRGM}
\visible<2-4,6-7>{\ticalcfig{\ticalcfigCircleSecond\ticalcfigCircle{\ticalcfigCircleColFive}{0.228}}}
\visible<5>{\ticalcfig{\ticalcfigCircle{\ticalcfigCircleColThree}{0.802}}}
\visible<8>{\ticalcfig{\ticalcfigCircleEnter}}

Indien je een \tifonttxt{prgm} wilt verwijderen:
\begin{itemize}
  \item<2-> Druk op \tiSecond\tiPlus=\tiMEM
  \item<3-> Kies \inlineticalc{Mem\,Mgmt/Del\ldots}
  \item<4-> Kies \inlineticalc{Prgm\ldots}
  \item<5-> Blader naar het te-deleten-\tifonttxt{prgm} en druk op \tiDEL.
  \item<6-> Yes\ldots Of course we are sure\ldots
  \item<7-> En weg is 'ie!
\end{itemize}
  
\begin{itemize}
  \item<8-> Om te archiveren druk je op \tiENTER\\
  	i.p.v. \tiDEL.
\end{itemize}

\vspace{1cm}

\begin{tikzpicture}[overlay,remember picture]
	\node[yshift=0.6cm] (BL) at (current page.south east){ };
	\node [anchor=south east,xshift=0.195cm] at (BL)
	{%
	\only<2-3>{%
		\begin{ticalc}%
			\select{MEMORY} \\%
			\only<2>{\selectitem{\one\:}}\only<3->{\one\:}About \\%
			\only<3->{\selectitem{2\:}}\only<2>{2\:}Mem\,Mgmt/Del\ldots \\%
			3\:Clear\,Entries \\%
			4\:ClrAllLists \\%
			5\:Archive \\%
			6\:UnArchive \\%
			7\arrowdown Reset\ldots%
		\end{ticalc}%
	}%
	\only<4>{%
		\begin{ticalc}%
			RAM\,FREE\hfill5144 \\%
			ARC\,FREE\hfill24861 \\%
			2\arrowup Real\ldots \\%
			3\:Complex\ldots \\%
			4\:List\ldots \\%
			5\:Matrix\ldots \\%
			6\:Y-Vars\ldots \\%
			\selectitem{7\arrowdown} Prgm\ldots%
		\end{ticalc}%
	}%
	\only<5>{%
		\begin{ticalc}%
			RAM\,FREE\hfill5144 \\%
			ARC\,FREE\hfill24861 \\%
			\,\,ABC\hfill530\\%
			\,\,AFGELYDN\hfill724\\%
			\,\,EXACTOR\hfill352\\%
			\ftriangleright\,JUNK\hfill34\\%
			\,\,SKRKNSGM\hfill1238\\%
			\,\,\theta NUM2FRA\hfill502%
		\end{ticalc}%
	}%
	\only<6>{%
		\begin{ticalc}%
			\select{Are\,You\,Sure?}\\%
			\one\:No\\%
			\selectitem{2\:}Yes%
		\end{ticalc}%
	}%
	\only<7>{%
		\begin{ticalc}%
			RAM\,FREE\hfill5110 \\%
			ARC\,FREE\hfill24861 \\%
			\,\,ABC\hfill530\\%
			\,\,AFGELYDN\hfill724\\%
			\,\,EXACTOR\hfill352\\%
			\ftriangleright\,SKRKNSGM\hfill1238\\%
			\,\,\theta NUM2FRA\hfill502\\%
			\,\,\theta NUM2SQR\hfill275%
		\end{ticalc}%
	}%
	\only<8>{%
		\begin{ticalc}%
			RAM\,FREE\hfill6348 \\%
			ARC\,FREE\hfill23623 \\%
			\,\,ABC\hfill530\\%
			\,\,AFGELYDN\hfill724\\%
			\,\,EXACTOR\hfill352\\%
			\ftriangleright*SKRKNSGM\hfill1238\\%
			\,\,\theta NUM2FRA\hfill502\\%
			\,\,\theta NUM2SQR\hfill275%
		\end{ticalc}%
	}%
	};
\end{tikzpicture}

\end{frame}



%% END %%